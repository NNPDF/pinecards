\documentclass[a4paper,11pt]{article}
\pdfoutput=1

\usepackage[UKenglish]{babel}
\usepackage{booktabs}
\usepackage{csquotes}
\usepackage{jheppub}
\usepackage[T1]{fontenc}
\usepackage{listings}
\usepackage{lmodern}
\usepackage[binary-units=true,detect-weight=true]{siunitx}
\usepackage{xcolor}
\usepackage[percent]{overpic}
\usepackage[section]{placeins}


\newcommand{\alphas}{\alpha_\mathrm{s}}

\title{PineAPPL: EW and mixed QCD--EW corrections for PDF fits}

\author[a]{S. Carrazza,}
\author[b]{E. Nocera,}
\author[a]{C. Schwan}
\author[a]{and M. Zaro}

\affiliation[a]{Tif Lab, Dipartimento di Fisica, 
Universit\`a di Milano and INFN, Sezione di Milano, 20133 Milano, Italy}
\affiliation[b]{Nikhef Theory Group, Science Park 105, 1098 XG Amsterdam, 
The Netherlands}

\emailAdd{stefano.carrazza@mi.infn.it}
\emailAdd{e.nocera@nikhef.nl}
\emailAdd{christopher.schwan@mi.infn.it}
\emailAdd{marco.zaro@mi.infn.it}

\abstract{We introduce {\sc PineAPPL}, a computer library that makes it possible to produce fast interpolation 
grids for fitting parton distribution functions (PDFs)
including corrections of strong and electroweak origin. After presenting
the features and performances of such a code, we show how to interface {\sc PineAPPL} with general-purpose
softwares that allow the user to compute such corrections, providing an explicit example with
{\sc MadGraph5\_aMC@NLO}. We showcase some phenomenological result focusing on processes and observables
relevant for PDF fits, and 
we discuss what steps need to be undertaken in the presentation of experimental data in order to have them
consistently employed in a fit including electroweak corrections.}


\begin{document}

\maketitle
\flushbottom

\section{Introduction}
\label{sec:introduction}

The Large Hadron Collider (LHC) 



ERN

The impact of EW corrections to LHC data.\\
How to do that consistently, include systematically EW corrections, blabla.\\
Outline a possible strategy to deal with EW correction in LHC observables (FSR, PHOTOS, dressed/born, ...).\\
Pinappl kills amcfast and mcgrid, replace applgrid and fastnlo.\\

\section{PDF-independent binning of phase-space weights with \texorpdfstring{\textsc{PineAPPL}}{PineAPPL}}
\label{sec:pineappl}

In this paper we introduce a new library called \textsc{PineAPPL}, which bins phase-space weights independently from the chosen PDF set.
The files produced in this way, generally called grids, can be used to quickly evaluate several observables and also to assess the impact of different PDF sets and their PDF uncertainties.
Finally they are the main theoretical input to a PDF determination.
In that sense \textsc{PineAPPL} is similar to \textsc{APPLgrid}~\cite{Carli:2010rw} and \textsc{fastNLO}~\cite{Kluge:2006xs,Wobisch:2011ij,Britzger:2012bs}, but it also understands EW, which are the main interest in this paper.
The following features distinguish it:
\begin{itemize}
\item Support for arbitrary fixed-order calculations in powers of $\alpha$, $\alphas$ or combinations thereof, e.g.\ in mixed QCD-EW corrections.
Furthermore, variations of the renormalisation and factorisation scale are supported, if needed.
For each needed combination of the couplings and logarithms of renormalisation and factorisation scale a separate subgrid is created (see section~\ref{sec:multi-coupling-expansion} for more details);
\item Support for all-order predictions coming from a resummation calculation or a photon-/parton-shower, which are important for some observables (see section~\ref{sec:results}),
\item A simple \textsc{C}-interface, with a wrapper for \textsc{Fortran} and \textsc{Python} (see appendix~\ref{app:example-program} for examples and documentation), which is needed for Monte Carlos and programs to read and write \textsc{PineAPPL} grids.
\textsc{PineAPPL} itself is written in Rust (see appendix~\ref{app:installation} for installation instructions).
\end{itemize}
For \textsc{mg5\_aMC@NLO}~\cite{Alwall:2014hca,Frederix:2018nkq} the interfacing code is already implemented in the most recent version, which replaces the \textsc{aMCfast}~\cite{Bertone:2014zva} interface.
The interfacing code for other Monte Carlo generators should be easy to write, see appendix~\ref{app:example-program} for a small example program.
Finally, \textsc{PineAPPL} provides programs to convert \textsc{APPLgrids} and \textsc{fastNLO} tables to \textsc{PineAPPL} grids.

\subsection{Cross sections in a multi-coupling expansion}
\label{sec:multi-coupling-expansion}
{\bf MZ comments:}
\begin{itemize}
    \item I would prefer to keep the notation similar to the aMCfast paper, or at least to start with that, and then generalise it
    \item drop eq 2.2, start already by 2.1 with a symmetric treatment of $\alpha$ and $\alphas$ (plus, the symbol $\sigma$ is overloaded there)
    \item $Q$ is the Ellis-Sexton scale, a (unphysical) scale introduced to have all logs in the form $\log(\mu_R/F / Q)$, and hence to separate
        the dependence of $\mu_R$ and $\mu_F$
    \item the business of the different kinematics (born, resolved, counterterms, etc) is not mentioned at all
    \item eq 2.1 assumes $\sigma_{ab}$ integrated over phase-space?
\end{itemize}
{\bf end MZ comments:}

Fixed-order partonic cross sections supported by \textsc{PineAPPL} are expansions in powers of the strong coupling $\alphas$, the electromagnetic coupling $\alpha$, and, if scale variations are desired, also in the logarithms of $\xi_\mathrm{R} = \mu_\mathrm{R}^2 / Q^2$ and $\xi_\mathrm{F} = \mu_\mathrm{F}^2 / Q^2$,
\begin{equation}
\sigma_{ab} (x_1, x_2, Q^2; \xi_\mathrm{R}, \xi_\mathrm{F}) = \sum_{k,l,m,n} \alphas^k \left( \xi_\mathrm{R} Q^2 \right) \log^m ( \xi_\mathrm{R} ) \log^n ( \xi_\mathrm{F} ) w_{ab}^{(k,l,m,n)} \left( x_1, x_2, Q^2 \right) \text{,}
\label{eq:expansion}
\end{equation}
with the phase-space weights $w$ defined as
\begin{equation}
w_{ab}^{(k,l,m,n)} \left( x_1, x_2, Q^2 \right) = \alpha^l \sigma_{ab}^{(k,l,m,n)} \left( x_1, x_2, Q^2 \right) \text{.}
\label{eq:phase-space-weight}
\end{equation}
The left-hand side of eq.~\eqref{eq:expansion} shows the partonic cross section for a process $a + b \to X$, where the parton $a$ has momentum fraction $x_1$ and $b$ has momentum fraction $x_2$.
The central value of the renormalisation and factorisation scale, $Q^2$, is treated as an independent variable; the parameters $\xi_\mathrm{R}$ and $\xi_\mathrm{F}$ allow varying the scales around the central value.

Eq.~\eqref{eq:phase-space-weight} contains the phase-space weights that we assume are calculated numerically, e.g.\ with a Monte Carlo generator, and then passed to \textsc{PineAPPL}, which will store this information either approximately or exactly, depending on the chosen format (see section~\ref{sec:grid-representation}).
In particular, this phase-space weight is the product of
\begin{itemize}
\item the electroweak coupling $\alpha$, which we assume does not depend on either $x_1$, $x_2$, or $Q^2$ --- this is the case for the most prominent choices of $\alpha$, namely $\alpha (0)$, $\alpha (M_\mathrm{Z})$, and $\alpha_{G_\mu}$, but is not the case when $\alpha (\mu)$ is a (dynamic) scale-dependent coupling --- and
\item the rest of the partonic cross section, denoted using a multi index $(k,l,m,n)$, where each index is the exponent of either a coupling or a logarithm.
\end{itemize}
Having stored eq.~\eqref{eq:phase-space-weight}, this allows \textsc{PineAPPL} to quickly calculate
\begin{equation}
\sigma (\xi_\mathrm{R}, \xi_\mathrm{R}) = \sum_{a,b} \int_0^1 \mathrm{d} x_1 \int_0^1 \mathrm{d} x_2 \int_0^1 \mathrm{d} y \, f_a (x_1, \xi_\mathrm{F} Q^2) f_b (x_2, \xi_\mathrm{F} Q^2) \sigma_{ab} (x_1, x_2, Q^2) \text{.}
\label{eq:pineappl-convolution}
\end{equation}
where $Q^2 = Q^2_\mathrm{min} + (Q^2_\mathrm{max} - Q^2_\mathrm{min}) y$, to obtain the hadronic cross sections for arbitrarily many PDF sets and scale variations.

\subsubsection{Example}
\label{sec:pineappl-example}

To give an example of eq.~\eqref{eq:expansion}, the following shows Drell--Yan lepton-pair production up to terms at NLO:
\begin{equation}
\begin{split}
\sigma_{ab} (\xi_\mathrm{R}, \xi_\mathrm{F})
    &= \left[ \alpha^2 \sigma_{ab}^{(0,2,0,0)} \right] \\
    &+ \alphas \left( \xi_\mathrm{R} Q^2 \right) \left[ \alpha^2 \sigma_{ab}^{(1,2,0,0)} \right] + \alphas \left( \xi_\mathrm{R} Q^2 \right) \log (\xi_\mathrm{F}) \left[ \alpha^2 \sigma_{ab}^{(1,2,0,1)} (Q^2) \right] \\
    &+ \left[ \alpha^3 \sigma_{ab}^{(0,3,0,0)} \right] + \log (\xi_\mathrm{F}) \left[ \alpha^3 \sigma_{ab}^{(0,3,0,1)} (Q^2) \right] \text{.}
\end{split}
\end{equation}
The first term with index $(0,2,0,0)$ is the LO term, the next line shows the NLO QCD correction, and the final line the NLO EW correction.
Note that the dependence on the renormalisation scale is only indirectly through $\alphas$, because the LO does not have any gluons in the initial or final state.
Terms proportional to $\log (\xi_\mathrm{R})$ vanish, because they require an explicit dependence on the renormalisation scale, which only enters through counterterms with more than two gluons.

\subsubsection{General characteristics}

In general, we define as leading order all terms for which the sum of the coupling exponents in eq.~\eqref{eq:expansion} is smallest, i.e.\ $k + l = p$, where $p = \min (k+l)$.
This number is process dependent and usually determined by the number of external particles.
For many processes there is only one LO, but when a process has multiple quark lines, colourless (photons, \dots) and coloured particles (gluons, \ldots) can be exchanged between them, which allows for more than one leading order.
To each leading order a higher-order correction with an additional power of $\alphas$ and $\alpha$ can be calculated, which in general leads to at least two next-to-leading order corrections.
Sometimes there are higher orders that cannot directly be understood as a correction to a leading order, e.g.\ $\mathrm{p} \mathrm{p} \to \ell \bar{\ell} + \mathrm{jet} + X$, which has one leading order, $\mathcal{O} (\alphas \alpha^2)$, but three next-to-leading orders, one of which is $\mathcal{O} (\alpha^4)$~TODO.

Due to typical size of the couplings $\alphas^2 \sim \alpha$, it is naively expected that within the same order, i.e.\ for fixed $k + l$, terms with larger powers $\alphas^k$ dominate over those with smaller powers; however, in practise this naive expectation is not always true due to dynamic effects.
Some prominent examples are vector-boson scattering processes and TODO.

\subsection{Grid representation and accuracy}
\label{sec:grid-representation}

So far we did not explain how the phase space-weights, eq.~\ref{eq:phase-space-weight}, are represented.
An obvious choice are $n$-tuples, which for each observable and for each combination $(a, b, k, l, m, n)$ saves a list of $N$ 4-tuples,
\begin{equation}
\left\{ x_1^i, x_2^i, Q^2_i, \frac{\mathrm{d}}{\mathrm{d} \phi_i} w^{(k,l,m,n)}_{ab} (x_1^i, x_2^i, Q^2_i) \right\}_{i=1}^N \text{.}
\end{equation}
The last element in the tuple is not an integrated cross section as in eq.~\eqref{eq:phase-space-weight}, but rather fully differential in the phase space $\phi$, evaluated at a specific phase-space point $\phi_i$.
This is due to the fact that Monte Carlo integrators do not perform the phase space integrals and the convolution with the PDFs separately, but together at the same time.

In case of $n$-tuples, the reconstruction of the integrated cross section, eq.~\eqref{eq:pineappl-convolution}, is very straightforward, for example for the central scale choice,
\begin{equation}
w_{ab}^{(k,l,m,n)} = \sum_{i=1}^N f_a (x_1^i, Q^2_i) f_b (x_2^i, Q^2) \alphas (Q^2) \frac{\mathrm{d}}{\mathrm{d} \phi_i} w^{(k,l,m,n)}_{ab} (x_1^i, x_2^i, Q^2_i) \text{,} \label{eq:n-tuple-integration}
\end{equation}
given a proper normalization of the integral measures; this equation is the same approximation that a Monte Carlo integrator evaluates.
For the full hadronic cross section it is only required to sum eq.~\eqref{eq:n-tuple-integration} over all open indices.

This method has the clear advantage of being able to reproduce exactly the numerical value of the generator.
However, the price to pay for is in large storage requirements.
In the case of the NLO Drell--Yan (section~\ref{sec:pineappl-example}) the $n$-tuples need roughly \SI{7030}{\giga\byte} of storage (see appendix~\ref{app:drell-yan-storage} for an explanation of this number).

A different choice is to build an interpolation grid, which basically partitions the space
\begin{equation}
H = [x_\mathrm{min},x_\mathrm{max}]^2 \times [Q^2_\mathrm{min}, Q^2_\mathrm{max}] \ni (x_1, x_2, Q^2)
\end{equation}
along each axis into a small numbers of bins, which allows one to approximately recover the result of the cross section.

\section{Results}
\label{sec:results}

In this section we demonstrate the capabilities of \textsc{PineAPPL} by
computing fast interpolation grids, accurate to NLO QCD and NLO QCD+EW,
for a representative set of processes in which EW corrections are expected 
to be sizeable. In order to consider some realistic kinematics for these
processes, we resort to measurements commonly devised for inclusion in PDF
fits. Our aim is twofold. On the one hand, we want to validate the results
obtained with \textsc{PineAPPL}; on the other hand, we want to assess the
impact of the EW corrections for usual experimental setups. We describe first
the processes and measurements that we consider, then the computational
settings that we adopt, and finally the results that we obtain.

\subsection{Processes and measurements}
\label{subsec:processes_and_measurements}

We focus on the following three processes: DY lepton-pair production, top-quark
pair production, and Z-boson (lepton-pair) production with non-zero transverse
momentum at the LHC. For each of these processes, we consider the measurments
described below.

\paragraph{DY lepton pair production.}
We select the distribution, single-differential in the invariant mass of the
lepton pair, $M_{\ell \bar\ell}$, measured by the ATLAS experiment at a centre-of
mass energy of 7~TeV in the high-mass region
($M_{\ell\bar\ell}>116$~GeV)~\cite{Aad:2013iua}.
We also select the distribution, double-differential in the rapidity and in
the invariant mass of the lepton pair, $y_{\ell\bar\ell}$ and $M_{\ell\bar\ell}$,
measured by the CMS experiment at 7~TeV~\cite{Chatrchyan:2013tia}.
These measurements are currently included as standard in the
NNPDF3.1~\cite{Ball:2017nwa} and MMHT2014~\cite{Harland-Lang:2014zoa} PDF sets,
although with appropriate kinematic cuts that remove the bins at the largest
values of invariant mass, where EW corrections become sizeable.

\paragraph{Top-quark pair production.}
We select the distributions, single-differential in either the transverse
momentum of the top quark, $p_T^t$, or the invariant mass of the top-quark
pair, $m_{t\bar t}$, measured by the ATLAS and CMS experiments at a centre-of-mass
energy of 8~TeV~\cite{Aad:2015mbv,Khachatryan:2015oqa}. These measurements have
been extensively studied in the context of PDF fits in
Refs.~\cite{Czakon:2016olj,Bailey:2019yze,Amoroso:2020lgh,Kadir:2020yml}.
Because EW corrections are significantly smaller for distributions differential
in the rapidity of either the top quark or the top-quark
pair~\cite{Czakon:2017wor}, these were preferred for inclusion in the NNPDF3.1
set~\cite{Ball:2017nwa}.

\paragraph{$Z$-boson production with non-zero transverse momentum.}
We select the distribution, single-differential in the transverse momentum of
the $Z$ boson, $p_T^Z$, measured by the CMS experiment at a centre-of-mass
energy of 13~TeV~\cite{Sirunyan:2019bzr}. This measurement has not been
included in a PDF determination yet. Given that it has sub-percent
uncertainties, it is expected that EW corrections will be essential to
achieve a good description of it, and to constrain accurately the PDFs.
A similar challenge was observed in similar measurements, from the
ATLAS~\cite{Aad:2015auj} and CMS~\cite{Khachatryan:2015oaa} experiment at a
centre-of-mass energy of 8~TeV, in Ref.~\cite{Boughezal:2017nla}, which lead to
their partial inclusion (upon the selection of an appropriate kinematic cut)
in the NNDPF3.1 PDF set~\cite{Ball:2017nwa} and in variants of the CT18 PDF
set~\cite{Hou:2019efy}.

\subsection{Computational settings}
\label{subsec:computational_settings}



We discuss,
\begin{itemize}
\item in section~\ref{sec:atlas-high-mass-dy}, ATLAS high-mass DY lepton-pair production at \SI{7}{\tera\electronvolt} \cite{Aad:2013iua}, measuring $\mathrm{d} \sigma / \mathrm{d} M_{\ell \bar{\ell}}$ for the lepton-pair invariant mass $M_{\ell \bar{\ell}} > \SI{116}{\giga\electronvolt}$,
\item in section~\ref{sec:cms-dy}, CMS DY lepton-pair production at \SI{7}{\tera\electronvolt} \cite{Chatrchyan:2013tia}, measuring $\mathrm{d} \sigma / \mathrm{d} y_{\ell \bar{\ell}}$ for six slices in the range $\SI{20}{\giga\electronvolt} < M_{\ell \bar{\ell}} < \SI{1500}{\giga\electronvolt}$,
\item in section~\ref{sec:atlas-top-pair-production}, ATLAS top-pair production at \SI{8}{\tera\electronvolt} \cite{Aad:2015mbv}, measuring
\begin{itemize}
\item the transverse momentum of the reconstructed top, $\mathrm{d} \sigma / \mathrm{d} p_\mathrm{T}^\mathrm{t}$,
\item its rapidity, $\mathrm{d} \sigma / \mathrm{d} y_\mathrm{t}$,
\item the invariant mass of the reconstructed top pair, $\mathrm{d} \sigma / \mathrm{d} M_{\mathrm{t} \bar{\mathrm{t}}}$, and
\item its rapidity, $\mathrm{d} \sigma / \mathrm{d} M_{\mathrm{t} \bar{\mathrm{t}}}$,
\end{itemize}
\item in section~\ref{sec:cms-transverse-momentum}, CMS transverse momentum of the Z boson at \SI{13}{\tera\electronvolt} \cite{Sirunyan:2019bzr}, measuring $\mathrm{d} \sigma / \mathrm{d} p_\mathrm{Z}$ for $p_\mathrm{Z} > \SI{20}{\giga\electronvolt}$.
\end{itemize}
We used the MC \texttt{mg5\_aMC@NLO}~\cite{} and the PDF set \texttt{NNPDF31\_as\_0118\_luxqed}~\cite{} to generate the predictions.
The chosen PDF set contains a photon PDF calculated from the LUXQED method~\cite{}.
For each process---except top-pair production, which has stable tops in the final state---we use a complex-mass scheme~\cite{} as described in ref.~\cite{}.
The values of the most important parameters are
\begin{equation}
\begin{aligned}
M_\mathrm{W} &= \SI{80}{\giga\electronvolt} \text{,} \quad &
M_\mathrm{Z} &= \SI{90}{\giga\electronvolt} \text{,} \quad &
m_\mathrm{t} &= \SI{170}{\giga\electronvolt} \text{,} \\
\Gamma_\mathrm{W} &= \SI{2}{\giga\electronvolt} \text{,} &
\Gamma_\mathrm{Z} &= \SI{2}{\giga\electronvolt} \text{,} &
G_\mu &= \text{.}
\end{aligned}
\end{equation}

\noindent
TODO for each of the following subsections:
\begin{itemize}
\item size of the photon-initiated contributions,
\item largest partonic channel,
\item most important $x$ region,
\item PDF uncertainty,
\end{itemize}

\subsection{ATLAS high-mass DY lepton-pair production at \SI{7}{\tera\electronvolt}}
\label{sec:atlas-high-mass-dy}

\begin{equation}
\begin{gathered}
p_\mathrm{T}^\ell > \SI{25}{\giga\electronvolt} \text{,} \quad |\eta_\ell| < 2.5 \text{,} \\
\SI{116}{\giga\electronvolt} < M_{\ell \bar{\ell}} < \SI{1500}{\giga\electronvolt} \text{,}
\end{gathered}
\end{equation}

\subsection{CMS DY lepton-pair production at \SI{7}{\tera\electronvolt}}
\label{sec:cms-dy}

\begin{equation}
\begin{gathered}
p_\mathrm{T}^{\ell_1} > \SI{14}{\giga\electronvolt} \text{,} \quad p_\mathrm{T}^{\ell_2} > \SI{9}{\giga\electronvolt} \text{,} \quad |\eta_\ell| < 2.4 \text{,} \\
|\eta_{\ell \bar{\ell}}| < 2.4 \text{,} \quad \SI{20}{\giga\electronvolt} < M_{\ell \bar{\ell}} < \SI{1500}{\giga\electronvolt} \text{,}
\end{gathered}
\end{equation}

\subsection{ATLAS top-pair production at \SI{8}{\tera\electronvolt}}
\label{sec:atlas-top-pair-production}

\subsection{CMS transverse momentum of the Z boson at \SI{13}{\tera\electronvolt}}
\label{sec:cms-transverse-momentum}

\begin{equation}
\begin{gathered}
p_\mathrm{T}^\ell > \SI{25}{\giga\electronvolt} \text{,} \quad |\eta_\ell| < 2.4 \text{,} \quad M_\mathrm{Z} - \SI{15}{\giga\electronvolt} < M_{\ell \bar{\ell}} < M_\mathrm{Z} + \SI{20}{\giga\electronvolt} \text{,} \\
|\eta_{\ell \bar{\ell}}| < 2.4 \text{,} \quad \SI{20}{\giga\electronvolt} < p_\mathrm{T}^{\ell \bar{\ell}} < \SI{1500}{\giga\electronvolt} \text{,}
\end{gathered}
\end{equation}

\begin{figure}
    \centering
    \includegraphics[width=0.5\textwidth]{figures/pineappl_ATLASZHIGHMASS49FB}
    \caption{PineAPPL comparison for ATLAS high-mass Drell--Yan at $\sqrt{s}=7$ TeV.}
    \label{fig:atlaszhighmass49fb}
\end{figure}

\begin{figure}
    \centering
    \includegraphics[width=0.5\textwidth]{figures/pineappl_CMSDY2D11_bin1}%
    \includegraphics[width=0.5\textwidth]{figures/pineappl_CMSDY2D11_bin2}
    \caption{PineAPPL comparison for CMS 2D Drell--Yan.}
    \label{fig:cmsdy2d11_bins12}
\end{figure}

\begin{figure}
    \centering
    \includegraphics[width=0.5\textwidth]{figures/pineappl_CMSDY2D11_bin3}%
    \includegraphics[width=0.5\textwidth]{figures/pineappl_CMSDY2D11_bin4}
    \caption{PineAPPL comparison for CMS 2D Drell--Yan.}
    \label{fig:cmsdy2d11_bins34}
\end{figure}


\begin{figure}
    \centering
    \includegraphics[width=0.5\textwidth]{figures/pineappl_CMSDY2D11_bin5}%
    \includegraphics[width=0.5\textwidth]{figures/pineappl_CMSDY2D11_bin6}
    \caption{PineAPPL comparison for CMS 2D Drell--Yan.}
    \label{fig:cmsdy2d11_bins56}
\end{figure}


\begin{figure}
    \centering
    \includegraphics[width=0.5\textwidth]{figures/pineappl_ATLAS_TTB_DIFF_8TEV_LJ_TPT}%
    \includegraphics[width=0.5\textwidth]{figures/pineappl_ATLAS_TTB_DIFF_8TEV_LJ_TTM}
    \caption{PineAPPL comparison for ATLAS top pair.}
    \label{fig:cmsdy2d11_bins56}
\end{figure}

\begin{figure}
    \centering
    \includegraphics[width=0.5\textwidth]{figures/pineappl_CMS_Z_13_TEV}
    \caption{PineAPPL comparison for CMS $Z$ $p_T$ distribution.}
    \label{fig:cmsdy2d11_bins56}
\end{figure}

%ERN 7 Apr: there is consensus on the fact that we should present results for the
%following data sets:
%\begin{itemize}
%\item ATLAS high mass DY distributions, 7 TeV~\cite{Aad:2013iua} (CS);
%\item CMS 2D DY distributions, 7 TeV~\cite{Chatrchyan:2013tia} (CS);
%\item ATLAS top pair differential distributions ($m_{t\bar{t}}$ and $p_T^t$),
%8 TeV~\cite{Aad:2015mbv} (ERN);
%\item CMS $Z$ $pT$ distributions, 13 TeV~\cite{Sirunyan:2019bzr} (ERN).
%\end{itemize}
%
%We agree not to display any LHCb measurement, given that they won't add
%further value to our discussion.
%Note added: we might also want to have a look at the ATLAS 2D and 3D DY
%distributions, 8 TeV~\cite{Aad:2016zzw,Aaboud:2017ffb}, if time allows
%us to do so.
%
%CS 25 Jun: We've agreed to show basically two types of plots: 1) technical plots showing the good agreement between the MC compared to the results from the grids, and 2) phenomenological results showing larger EW corrections, for example.
%In the aMCfast paper both is shown in single plot, but since we have more to show, I suggest the following: for the technical plots we show a 2x2 matrix of plots, each showing the difference of the grid result compared to the MC result, in the following fashion:
%\begin{itemize}
%\item NLO QCD with low statistics,
%\item NLO QCD+EW with low statistics,
%\item NLO QCD with high statistics, and finally
%\item NLO QCD+EW with high statistics,
%\end{itemize}
%each showing a few scale variations.
%These plots then clearly show that no matter what corrections you choose, no matter the statistics, and no matter the scale variation, the agreement is always excellent.
%In any case I would like to avoid showing a plot with absolute numbers and low statistics, which looks a bit ridiculous in my opinion (look at figure 1, left side, top plot of the aMCfast paper).
%
%Finally we can show another series of plots, which in my opinion should be very similar to the usual pheno paper plots: absolute numbers with a scale variation band, maybe a few corrections shown in the same plot and then in the bottom relative corrections.

\section{On the presentation of experimental data}
\label{sec:doublecounting}

As it has been mentioned in the introduction, the ability to consistently include electro-weak correction in
PDF fits is not enough to make a fully-fledged fit possible in a consistent manner. Indeed, experimental data should
be provided in  a format that allows PDF collaborations to employ them in such fits in a theoretically sound manner. Before concluding the paper,
we will try to provide some guidelines, to facilitate a consistent presentation of experimental data. Common sources of inconsistency in the extraction of data
typically lead to accounting multiple times for EW effects (so-called double counting), or to not accounting at
all for such effects.

\begin{figure}[ht!]
    \centering
    \begin{overpic}[width=0.6\textwidth, trim=0.cm 11cm 0.cm 10cm, clip=True]{figures/dy-pi.pdf}
        \put (5, 33) {\large $q$}
        \put (20, 37) {\large $q$}
        \put (38, 5) {\large $\ell^+$}
        \put (38, 31) {\large $\ell^-$}
        \put (11, 6) {\large $\gamma$}
        %
        \put (65, 9) {\large $q$}
        \put (73, 30) {\large $q$}
        \put (96, 25) {\large $\ell^-$}
        \put (96, 8) {\large $\ell^+$}
        \put (55, 26) {\large $\gamma$}
    \end{overpic}
    \caption{\label{fig:dy-pi}
    Photon-induced (left) and quark-induced (right) contributions to the Drell-Yan process. In black, the LO process is shown.
    In red, the initial-state splitting leading to the real-emission $q \gamma \to \ell^+ \ell^- q$ is highlighted. Such a 
    real emission enters in the NLO EW corrections.}
\end{figure}
An example is the subtraction of (irreducible) background processes which must not be considered as such. A very blatant case
is neutral-current Drell--Yan, where the signal process is an opposite-sign lepton pair, which starts
at $\mathcal O(\alpha^2)$. Because this process is usually thought
as a quark-initiated, $s$-channel mechanism ($q\bar q \to \gamma^*/Z \to \ell^+ \ell^-$), in many analyses the photon-induced component,
$\gamma \gamma \to \ell^+ \ell^-$ in the $t$ channel, is considered as a different process, and it 
subtracted, possibly evaluated with different parton densities and/or including (unphysical) higher-order 
corrections. For example, in refs.~\cite{Aaboud:2017ffb,Aad:2016zzw} (a similar statement appears also in an older analysis~\cite{Aad:2013iua}), one reads:
\begin{quote}
The photon-induced process, $\gamma\gamma \to \ell \ell$, is simulated at LO using Pythia 8 
and the MRST2004qed PDF set~\cite{Martin:2004dh}. The expected yield for this process also accounts for 
NLO QED/EW corrections from references~\cite{Bardin:2012jk,Bondarenko:2013nu}, which decrease the yield by approximately 30\%.
\end{quote}
Such a distinction, which is unphysical and incorrect in quantum mechanics, may be somehow justified at LO. Beyond this order, it is simply wrong.
Indeed, at $\mathcal O(\alpha^3)$, the reaction $q \gamma \to \ell^+ \ell^- q$ becomes possible, which
includes both kind of topologies discussed above, see Fig.~\ref{fig:dy-pi}. While it is common sense that the QCD counterpart of this subtraction should never 
be performed---nobody would ever
consider to \enquote{subtract} the gluon-initiated contribution to top-pair production in top analyses---seemingly it is not so
when EW corrections are considered.

{\bf It looks like CMS does things in a pretty consistent manner~\cite{Sirunyan:2018owv, CMS:2014jea}. Shall we mention? We should
not make ATLAS appear as the bad guys, and CMS as the good ones.} 

A second example is related to removing EW effects from data. These can be either the full EW corrections
or just a part, for example deconvolving multiple-photon radiation from light particles in the final state. This applies mostly
to processes such as neutral- or charge-current Drell--Yan, specially when electrons are considered. The problem lie in the fact that
 electrons, and to a lesser extent muons, tend to radiate photons, and such photons are typically not accounted
for in hard matrix-elements. Thus, leptons that are measured in the detector are less energetic, and this fact is compensated for
by undoing the photon shower before publishing data, which are referred to e.g.\ as \emph{Born-level electrons} (see e.g. Ref.~\cite{Aad:2015auj} for
its definition). There are at least two 
problems related with this. The first, a quite obvious one, is that when EW corrections
are included at fixed order, the first photon emission is included exactly at the matrix-element level. The inclusion of 
subsequent emission would require the matching with the QED shower, which is not yet available for general processes. It is interesting
to note that one can tune the QED parton shower in order to mimick NLO EW effects, so that a prediction only accurate at NLO
QCD displays a remarkable agreement with another at NLO QCD+EW when the photonic shower is included (see example the behaviour of predictions showered with
\textsc{Photos}~\cite{Barberio:1990ms,Barberio:1993qi,Golonka:2005pn}in Ref.~\cite{CarloniCalame:2016ouw}). However, this kind of agreement
always comes \emph{a posteriori}, and cannot be ensured in general. The second, which
applies for electrons, is that they are never measured as bare particles, because of the finite resolution 
of the electromagnetic calorimeter. Since collinear photonic emission cannot be resolved, data 
involving electrons should also be published
in terms of dressed particles, after applying some recombination scheme. This has the further advantage of being 
inclusive on the effect of further collinear emissions. For what concerns muons, while in principle
the concept of bare muons is a physical one, it should be kept in mind that modern, general-purpose codes employed to
compute EW corrections treat leptons as massless. This fact encourages to explore the possibility of employing dressed
muons, on the same footing as their electron counterpart. It is reassuring to acknowledge that this practice is already quite common in experimental analyses:
indeed, to mention two examples of analyses discussed in this paper, in Ref.~\cite{Aad:2015auj} data for dressed leptons are published, together with the pre- and post-ISR ones, while Refs.~\cite{Sirunyan:2019bzr} employs a dressed-lepton definition.

\clearpage
\noindent
\textbf{Notes from CS}:
\begin{itemize}
\item overall we should also point out the positive (rewrite the last part of this section a bit): for leptonic observables there always seems to be a pre-/post-FSR dataset with born/dressed leptons
\begin{itemize}
\item is this really the case?
\item is this also the case for non-leptonic observables? Specifically: Do ATLAS/CMS subtract photon radiation for observables of jet, top, or any other reconstructed objects (for instance: Z pT)?
\end{itemize}
\item double-photon initiated processes:
\begin{itemize}
\item how do they subtract it exactly (one analysis uses a concrete PDF set and subtracts the theory prediction; point out first how they do it, then that this is not a good idea, even if it would make sense: mixing theory predictions with measurements)
\item explain why they do it and point that this is not good
\end{itemize}
\item stress subtraction of \emph{irreducible} backgrounds
\end{itemize}

\section{Conclusions and Outlook}
\label{sec:conclusion}

The systematic inclusion of EW corrections in accurate theoretical computations
for several LHC processes is becoming more and more important in order to
match the increasing precision of the data. In this paper we simplified the
computational aspect of this task, building upon the automation of QCD and EW
computations pioneered in recent years~\cite{Kallweit:2014xda,Biedermann:2017yoi,Frederix:2018nkq}.
Specifically we developed \textsc{PineAPPL}, a new library that stores perturbative calculations from an external Monte Carlo generator in a PDF-independent way using interpolation grids.
This offers the advantage of fast a posteriori convolutions with PDFs, for example to study the uncertainties coming from different PDF sets and/or the strong coupling $\alphas$, and to determine the PDFs themselves, a task for
which fast-interpolation grids are fundamental.
We tested \textsc{PineAPPL} together with \textsc{mg5\_aMC} and found a precision of \numrange{e-4}{e-5} relative to the MC result, which is excellent for all
foreseeable practical purposes.
Although we used \textsc{mg5\_aMC}, we note that \textsc{PineAPPL} is not tied in any way to a specific Monte Carlo generator, and can be easily interfaced with any of them.

We emphasise that a distinguishing feature of \textsc{PineAPPL} is the support for arbitrary coupling orders not only in the strong, but also in the electroweak coupling.
This enables us to generate, for the first time, NLO EW and NLO combined QCD--EW interpolation grids.
Using \textsc{mg5\_aMC} we calculated and showcased the impact of these corrections for specific measurements of some representative LHC processes: Drell--Yan lepton-pair production, top-pair production, and Z-boson production with non-zero transverse momentum. 

Finally, we discussed the issue of subtracting EW corrections in experimental data, which becomes important when theoretical predictions including EW corrections are compared to experimental data. In particular, with the
development of \textsc{PineAPPL}, all technical requirements are fulfilled for producing the first PDF fit of LHC data including EW and combined QCD--EW corrections.
This will have at least two advantages: in PDF fits phase-space regions are usually cut away if they exhibit large EW corrections; including them therefore increases the number of data points in a fit and therefore indirectly enlarges a PDF set's interpolation region.
Secondly, this makes it possible to use experimental data that are closer to the actual measurement, without the need to compensate for missing EW corrections.
We plan to address this task in a future work.

\vspace{0.5cm}
\hrule
\begin{center}
The \textsc{PineAPPL} library is available at \url{https://n3pdf.github.io/pineappl}.
\end{center}


\appendix

\acknowledgments
We are grateful to Rikkert Frederix for discussions and explanations about
\textsc{aMCfast}. We acknowledge discussions with 
Florencia Canelli, Stefano Camarda, Paolo Francavilla, Abideh Jafari, Andreas Jung, Elizaveta Shabalina, Wolfgang Wagner about the 
treatment of EW effects in experimental data. \\
%Alessandro Vicini\\
E.R.N. is supported by the European Commission through the Marie
Sk\l odowska-Curie Action ParDHonS FFs.TMDs (grant number 752748).
C.S.\ is supported by the European Research Council under the European Union's
Horizon 2020 research and innovation Programme (grant agreement no.\ 740006).
MZ thanks the Nikhef institute in Amsterdam,
where he was employed during the early stages of this paper.

\section{Installation and usage of \texorpdfstring{\textsc{PineAPPL}}{PineAPPL}}
\label{app:pineappl}

%\subsection{Efficiency of the grid representation}
%\label{app:pineappl-representation-efficiency}
%
%Using a similar notation introduced in eq.~\eqref{eq:expansion}, we single out a specific NLO correction:
%\begin{equation}
%\begin{split}
%\alphas^k \bigl( \mu_\mathrm{R}^2 \bigr) \alpha^l \Bigl[ \log \bigl( \mu_\mathrm{R}^2 \bigr) &W_{ab}^{(k,l,1,0)} (x_1, x_2) \\
%+ \log \bigl( \mu_\mathrm{F}^2 \bigr) &W_{ab}^{(k,l,0,1)} (x_1, x_2) \\
%&W_{ab}^{(k,l,0,0)} (x_1, x_2) \Bigr] \text{.}
%\end{split}
%\end{equation}
%If we assume the case of a dynamic scale choice, i.e.\ $\mu_\mathrm{R}^2 \equiv \mu_\mathrm{R}^2 \bigl( x_1, x_2 \bigr)$ and $\mu_\mathrm{F}^2 \equiv \mu_\mathrm{F}^2 \bigl( x_1, x_2 \bigr)$,

\subsection{Installation}
\label{app:installation}

\textsc{PineAPPL} currently consists of three parts: 1) the library itself, which is a dependency for the other parts, 2) the helper program \texttt{pineappl}, which allows to read \textsc{PineAPPL} grids from the command line and make predictions with it, and finally 3) the C interface, which is intended to be used in Monte Carlo integrators to generate the grids.

\subsubsection*{Installation of Rust}

\lstset{
  basicstyle=\ttfamily\small,
  showstringspaces=false,
  commentstyle=\color{red},
  keywordstyle=\color{blue}
}

All parts are written in Rust and therefore need a Rust compiler and other tools.
On computers with a \texttt{bash} shell the installation is as simple as
\begin{lstlisting}[language=bash]
 $ curl --proto '=https' --tlsv1.2 -sSf https://sh.rustup.rs | sh
\end{lstlisting}
which downloads the compiler \texttt{rustc}, the package manager \texttt{cargo}, and a few other helpful tools.
When the installation has completed make sure to read and follow the instructions printed before.
See also \url{https://www.rust-lang.org/tools/install} for more details and for installation instructions for other systems.

\subsubsection*{Installation of the command-line program \texorpdfstring{\texttt{pineappl}}{pineappl}}

The command-line program \texttt{pineappl} is compiled and installed using
\begin{lstlisting}[language=bash]
 $ cargo install pineappl_cli
\end{lstlisting}
This program also needs \texttt{LHAPDF} \cite{Buckley:2014ana} installed.
For usage instructions simply type \texttt{pineappl} in your shell and read the help message.

\subsubsection*{Installation of the C-language interface (optional)}

For the C interface you need to first install \texttt{cargo-c},
\begin{lstlisting}[language=bash]
 $ cargo install cargo-c
\end{lstlisting}
and then download the \textsc{PineAPPL} repository, compile and finally install into it into a directory \texttt{\$prefix} as follows:
\begin{lstlisting}[language=bash]
 $ git clone https://github.com/N3PDF/pineappl/
 $ cd pineappl_capi/
 $ cargo cinstall --release --prefix=DIRECTORY
\end{lstlisting}
The last line will install the C header \texttt{pineappl\_capi.h}, the library, and a pkg-config\footnote{A standard way on Linux to express how dependencies are compiled/linked against, see \url{https://www.freedesktop.org/wiki/Software/pkg-config/}} file (\texttt{pineappl\_capi.pc}) into the directory specified as \texttt{DIRECTORY}.
Make sure that the environment variables \texttt{PATH}, \texttt{LD\_LIBRARY\_PATH}, and \texttt{PKG\_CONFIG\_PATH} are properly set.
The latter is needed for \texttt{pkg-config -{}-cflags -{}-libs pineappl\_capi} to work, which prints the necessary compiler/linker flags.

After being installed, you can compile and link against the library.
See App.~\ref{app:example-program} for an example.

Updated installation instructions are kept in the file \texttt{README.md} in \textsc{PineAPPL}'s repository at \url{https://github.com/N3PDF/pineappl/}.

\subsection{Example Monte Carlo program}
\label{app:example-program}

\definecolor{mygreen}{rgb}{0,0.6,0}
\definecolor{mygray}{rgb}{0.5,0.5,0.5}
\definecolor{mymauve}{rgb}{0.58,0,0.82}

\lstset{
  belowcaptionskip=1\baselineskip,
  breaklines=true,
  frame=L,
  numbers=left,
  xleftmargin=\parindent,
  showstringspaces=false,
  basicstyle=\footnotesize\ttfamily,
  keywordstyle=\bfseries\color{green!40!black},
  commentstyle=\itshape\color{purple!40!black},
  identifierstyle=\color{blue},
  stringstyle=\color{orange},
}

The following listing shows how to setup \textsc{PineAPPL} using its C interface in a simple Monte Carlo integrator for calculating the double-photon contribution to Drell--Yan lepton-pair production at the LHC.
All \textsc{PineAPPL} functions have the prefix \texttt{pineappl\_}.
The full example can be found at \url{https://github.com/N3PDF/pineappl/tree/master/examples/capi-dy-aa}.

\begin{lstlisting}[language=C++,mathescape=true]
// create a new luminosity function for the $\gamma\gamma$ initial state
auto* lumi = pineappl_lumi_new();
int32_t pdg_ids[] = { 22, 22 };
double ckm_factors[] = { 1.0 };
pineappl_lumi_add(lumi, 1, pdg_ids, ckm_factors);

// only LO $\alpha_\mathrm{s}^0 \alpha^2 \log^0(\xi_\mathrm{R}) \log^0(\xi_\mathrm{F})$
uint32_t orders[] = { 0, 2, 0, 0 };

// we bin in rapidity from 0 to 2.4 in steps of 0.1
double bins[] = {
    0.1, 0.2, 0.3, 0.4, 0.5, 0.6, 0.7, 0.8, 0.9, 1.0, 1.1, 1.2,
    1.3, 1.4, 1.5, 1.6, 1.7, 1.8, 1.9, 2.0, 2.1, 2.2, 2.3, 2.4
};

// create the PineAPPL grid with default interpolation and binning parameters
auto* keyval = pineappl_keyval_new();
auto* grid = pineappl_grid_new(lumi, 1, orders, 24, bins, keyval);

// now we no longer need `keyval` and `lumi`
pineappl_keyval_delete(keyval);
pineappl_lumi_delete(lumi);

// fill the grid with phase-space points
fill_grid(grid, 10000000);

// perform a convolution of the grid with PDFs
auto* pdf = LHAPDF::mkPDF("NNPDF31_nlo_as_0118_luxqed", 0);
auto xfx = [](int32_t id, double x, double q2, void* pdf) {
    return static_cast <LHAPDF::PDF*> (pdf)->xfxQ2(id, x, q2);
};
auto alphas = [](double q2, void* pdf) {
    return static_cast <LHAPDF::PDF*> (pdf)->alphasQ2(q2);
};

std::vector<double> dxsec(24);
pineappl_grid_convolute(grid, xfx, xfx, alphas, pdf, nullptr,
    nullptr, 1.0, 1.0, dxsec.data());

// print the results
for (std::size_t i = 0; i != 24; ++i) {
    std::printf("%.1f %.1f %.3e\n", bins[i], bins[i + 1], dxsec[i]);
}

// write the grid to disk
pineappl_grid_write(grid, "DY-LO-AA.pineappl");

// destroy the object
pineappl_grid_delete(grid);
\end{lstlisting}

\subsection{Demonstration of \texorpdfstring{\texttt{pineappl}}{pineappl}}
\label{app:pineappl-demo}

The program \texttt{pineappl} can be used to perform quick convolutions and other calculations with existing grids on the command line.
If started without any arguments, it prints its help and lists all supported subcommands:
\begin{lstlisting}[language=bash]
 $ pineappl
pineappl 0.1.0
Read, write, and query PineAPPL grids

USAGE:
    pineappl <SUBCOMMAND>

FLAGS:
    -h, --help       Prints help information
    -V, --version    Prints version information

SUBCOMMANDS:
    channels           Shows the contribution for each partonic channel
    convolute          Convolutes a PineAPPL grid with a PDF set
    help               Prints this message or the help of the given
                       subcommand(s)
    luminosity         Shows the luminosity function
    merge              Merges one or more PineAPPL grids together
    orders             Shows the predictions for all bins for each order
                       separately
    pdf_uncertainty    Calculates PDF uncertainties
\end{lstlisting}
As an example we show the grid produced for the ATLAS DY high-mass lepton-pair production (from section~\ref{subsec:numerical_results}).
The output shows all 13 bins with lower (\texttt{xmin}) and upper limit (\texttt{xmax}) of the invariant mass $M_{\ell \bar{\ell}}$ of the lepton pair, with the differential cross section $\mathrm{d} \sigma / \mathrm{d} M_{\ell \bar{\ell}}$ (\texttt{diff}), integrated cross section $(M_{\ell \bar{\ell}}^\mathrm{max} - M_{\ell \bar{\ell}}^\mathrm{min}) \mathrm{d} \sigma / \mathrm{d} M_{\ell \bar{\ell}}$ (\texttt{integ}), together with the perturbative uncertainty estimated from a 7-point scale variation (\texttt{neg unc} and \texttt{pos unc}).
\begin{lstlisting}[language=bash]
 $ pineappl convolute ATLASZHIGHMASS49FB.pineappl \
 > NNPDF31_nlo_as_0118_luxqed
bin xmin xmax     diff        integ     neg unc pos unc
---+----+----+------------+------------+-------+-------
  0  116  130 2.0630698e-1  2.8882978e0  -2.08%   1.69%
  1  130  150 9.1818985e-2  1.8363797e0  -1.79%   1.86%
  2  150  170 4.5306370e-2 9.0612740e-1  -1.60%   1.98%
  3  170  190 2.5894856e-2 5.1789712e-1  -1.66%   2.06%
  4  190  210 1.6075267e-2 3.2150534e-1  -1.70%   2.10%
  5  210  230 1.0526659e-2 2.1053317e-1  -1.71%   2.12%
  6  230  250 7.1928162e-3 1.4385632e-1  -1.71%   2.13%
  7  250  300 4.0776555e-3 2.0388277e-1  -1.70%   2.44%
  8  300  400 1.4775481e-3 1.4775481e-1  -1.94%   2.87%
  9  400  500 4.5473785e-4 4.5473785e-2  -2.30%   3.19%
 10  500  700 1.2164277e-4 2.4328555e-2  -2.41%   3.12%
 11  700 1000 1.9792340e-5 5.9377019e-3  -2.05%   2.12%
 12 1000 1500 2.0228761e-6 1.0114381e-3  -1.29%   0.47%
\end{lstlisting}

\subsection{Sample runcard for \texorpdfstring{\textsc{mg5\_aMC@NLO}}{mg5\_aMC@NLO}}
\label{app:sample-runcard}

The following run card was used to produce the results shown in section~\ref{subsec:numerical_results}.
The only difference to a normal \textsc{mg5\_aMC@NLO} run is the switch \texttt{set iappl 1}, which enables filling a \textsc{PineAPPL} grid.
For a complete set of runcards and patches see TODO.

\lstset{
  basicstyle=\ttfamily\small,
  showstringspaces=false,
  commentstyle=\ttfamily\small,
  keywordstyle=\ttfamily\small,
  identifierstyle=\color{black},
  stringstyle=\color{black},
}

\begin{lstlisting}
set complex_mass_scheme True
import model loop_qcd_qed_sm_Gmu
define p = p b b~
define j = p
generate p p > e+ e- [QCD QED]
output @OUTPUT@
launch @OUTPUT@
fixed_order = ON
set mz @MZ@
set ymt @YMT@
set ebeam1 3500
set ebeam2 3500
set pdlabel lhapdf
set lhaid 324900
set fixed_ren_scale True
set fixed_fac_scale True
set mur_ref_fixed @MZ@
set muf_ref_fixed @MZ@
set reweight_scale True
set ptl = 25.0
set etal = 2.5
set mll = 116
#user_defined_cut set mmllmax = 1500.0
set req_acc_FO 0.0001
set iappl 1
done
quit
\end{lstlisting}


\bibliographystyle{JHEP}
\bibliography{paper}

\end{document}
