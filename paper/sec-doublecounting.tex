\section{Subtraction of EW effects from data}
\label{sec:doublecounting}

The ability to include theoretical calculations of EW corrections in
PDF determinations is not sufficient to make a \emph{consistent} fit possible. Indeed, experimental data should
be provided in a format that allows PDF collaborations to easily employ them in these fits in a theoretically sound manner. To this purpose, in this section
we formulate some
guidelines to facilitate a consistent presentation of the experimental data. We focus on the problem of data with (partially) subtracted EW effects, which, in PDF fits with theory predictions including them, leads to a double counting issue.

\begin{figure}[!t]
    \centering
    \begin{overpic}[width=0.6\textwidth, trim=0.cm 11cm 0.cm 10cm, clip=True]{figures/dy-pi}
        \put (5, 33) {\large $\mathrm{q}$}
        \put (20, 37) {\large $\mathrm{q}$}
        \put (38, 5) {\large $\bar{\ell}$}
        \put (38, 31) {\large $\ell$}
        \put (11, 6) {\large $\gamma$}
        %
        \put (65, 9) {\large $\mathrm{q}$}
        \put (73, 30) {\large $\mathrm{q}$}
        \put (96, 25) {\large $\ell$}
        \put (96, 8) {\large $\bar{\ell}$}
        \put (55, 26) {\large $\gamma$}
    \end{overpic}
    \caption{\label{fig:dy-pi}
    Photon-induced (left) and quark-induced (right) contributions to the Drell-Yan process. In black, the LO process is shown.
    In red, the initial-state splitting leading to the real-emission $\mathrm{q} \gamma \to \ell \bar{\ell} \mathrm{q}$ is highlighted. Such a
    real emission enters in the NLO EW corrections.}
\end{figure}

A first example is the subtraction of (irreducible) background processes which must not be considered as such. A very blatant case
is neutral-current Drell--Yan, where the signal process is the production of an opposite-sign lepton pair, which starts
at $\mathcal O(\alpha^2)$. Because this process is usually thought
as a quark-initiated $s$-channel mechanism ($\mathrm{q} \bar{\mathrm{q}} \to \gamma^*/\mathrm{Z} \to \ell \bar{\ell}$), in many analyses the PI component,
$\gamma \gamma \to \ell \bar{\ell}$ in the $t$ channel, is considered a different process, and therefore as a background and subtracted.
The subtraction from the measured data is done by calculating the theoretical predictions of the double-photon initiated contribution, possibly including (ill-defined) higher-order
corrections. For example, in refs.~\cite{Aaboud:2017ffb,Aad:2016zzw} (a similar statement appears also in an older analysis~\cite{Aad:2013iua}), one reads:
\begin{quote}
The photon-induced process, $\gamma\gamma \to \ell \bar{\ell}$, is simulated at LO using Pythia 8
and the MRST2004qed PDF set~\cite{Martin:2004dh}. The expected yield for this process also accounts for 
NLO QED/EW corrections from references~\cite{Bardin:2012jk,Bondarenko:2013nu}, which decrease the yield by approximately \SI{30}{\percent}.
\end{quote}
Such a distinction, which is unphysical and incorrect in quantum mechanics, may be somehow justified at LO. Beyond this order, it is simply wrong.
Indeed, at $\mathcal O(\alpha^3)$, the reaction $\mathrm{q} \gamma \to \ell \bar{\ell} \mathrm{q}$ becomes possible, which
includes both kind of topologies discussed above, and needs both in order to yield an IR-finite result, see Fig.~\ref{fig:dy-pi} (as a consequence, one cannot speak of EW corrections to $\gamma \gamma \to \ell \bar{\ell}$). While it is common sense that the QCD counterpart of this subtraction should never 
be performed --- no one would
consider to \enquote{subtract} the gluon-initiated contribution to top-pair production in top analyses --- seemingly it is not so
when EW corrections are considered.

A second example is related to removing EW effects from data. These can be either the full EW corrections
or just a part of them. In either case, a comparison between these data and a NLO-EW accurate simulation aimed at the extraction of some parameter would be meaningless, as some effects included in the latter
have been removed from the former. The typical example relevant for the LHC is the deconvolution of effects due to multiple-photon radiation
from light particles in the final state. This applies mostly
to processes such as neutral- or charged-current Drell--Yan, especially when electrons are considered. The problem lies in the fact that
 electrons, and to a lesser extent muons, tend to radiate photons, which are not accounted
for in QCD-only matrix elements. Thus, leptons that are measured in the detector are less energetic, and this fact is compensated for
by inverting a photon shower. The resulting dataset is e.g.\ referred to as \emph{pre-FSR} with observables defined in terms of \emph{Born-level electrons} (see e.g.\ ref.~\cite{Aad:2015auj} for its definition).
These datasets are needed for and correctly used in QCD-only PDF determinations, since the EW corrections to some DY observables can be significant, and excluding them would therefore degrade the quality of the fit.
In fits including fixed-order EW corrections the problem with this definition, besides double counting, is that the first photon emission is included exactly at the matrix-element level. The inclusion of
subsequent emission would require the matching with the QED shower, which is not yet available for general processes.

It is interesting
to note that one can tune the QED parton shower to mimic NLO EW effects for specific processes and observables, so that a prediction only accurate at NLO
QCD displays a remarkable agreement with another at NLO QCD+EW when the photonic shower is included (see for example the behaviour of predictions showered with
\textsc{Photos}~\cite{Barberio:1990ms,Barberio:1993qi,Golonka:2005pn} in ref.~\cite{CarloniCalame:2016ouw}). However, this kind of agreement
always comes \emph{a posteriori}, and cannot be ensured in general.
Furthermore, the deconvolution of QED effects in data introduces a dependence on the program (and possibly on 
the specific version) employed for the shower inversion. 
This fact is especially problematic if deconvolved datasets are the only ones which are published, since undoing the exact deconvolution can be very difficult, or practically impossible.
(\textbf{TODO insert comment about LEP ISR and DIS EW effects removal}).

A more physical definition of leptonic observables would be one making use of either bare leptons (the leptons as they emerge after FSR)
or of dressed leptons (leptons and photons are clustered together and their momenta are combined, in analogy with jets in QCD). The problem with the former is that 
electrons are never measured as bare particles, because of the finite resolution 
of the electromagnetic calorimeter. For what concerns muons, while in principle the concept of a bare muon is physical, it should be kept in
mind that modern, general-purpose codes employed to
compute EW corrections treat leptons as massless, to ensure numerical stability of the matrix elements. In this case, using bare leptons is not collinear safe.
Dressed leptons avoid all these shortcomings, with the further advantage of being
inclusive on the effect of extra collinear emissions. This fact encourages to explore the possibility of employing a dressed-lepton
definition, regardless of the leptonic flavour. We acknowledge that this practice is already being followed in (some) experimental analyses:
indeed, to mention two examples discussed in this paper, in ref.~\cite{Aad:2015auj} data for dressed leptons are published, together with the Born-level
and bare ones, while refs.~\cite{Sirunyan:2019bzr} employs a dressed-lepton definition. We therefore recommend that these ways of presenting the experimental
data become standard in the future. 
