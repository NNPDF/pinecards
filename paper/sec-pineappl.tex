\section{PDF-independent binning of phase-space weights with \texorpdfstring{\textsc{PineAPPL}}{PineAPPL}}
\label{sec:pineappl}

In this paper we introduce a new library called \textsc{PineAPPL}, which bins phase-space weights independently from the chosen PDF set.
The files produced in this way, generally called grids, can be used to quickly evaluate several observables and also to assess the impact of different PDF sets and their PDF uncertainties.
Finally they are the main theoretical input to a PDF determination.
In that sense \textsc{PineAPPL} is similar to \textsc{APPLgrid}~\cite{Carli:2010rw} and \textsc{fastNLO}~\cite{Kluge:2006xs,Wobisch:2011ij,Britzger:2012bs}, but it also understands EW, which are the main interest in this paper.
The following features distinguish it:
\begin{itemize}
\item Support for arbitrary fixed-order calculations in powers of $\alpha$, $\alphas$ or combinations thereof, e.g.\ in mixed QCD-EW corrections.
Furthermore, variations of the renormalisation and factorisation scale are supported, if needed.
For each needed combination of the couplings and logarithms of renormalisation and factorisation scale a separate subgrid is created (see section~\ref{sec:multi-coupling-expansion} for more details);
\item Support for all-order predictions coming from a resummation calculation or a photon-/parton-shower, which are important for some observables (see section~\ref{sec:results}),
\item A simple \textsc{C}-interface, with a wrapper for \textsc{Fortran} and \textsc{Python} (see appendix~\ref{app:example-program} for examples and documentation), which is needed for Monte Carlos and programs to read and write \textsc{PineAPPL} grids.
\textsc{PineAPPL} itself is written in Rust (see appendix~\ref{app:installation} for installation instructions).
\item Superior efficiency for datasets with more then 100 observables.
\end{itemize}
For \textsc{mg5\_aMC@NLO}~\cite{Alwall:2014hca,Frederix:2018nkq} the interfacing code is already implemented in the most recent version, which replaces the \textsc{aMCfast}~\cite{Bertone:2014zva} interface.
The interfacing code for other Monte Carlo generators should be easy to write, see appendix~\ref{app:example-program} for a small example program.
Finally, \textsc{PineAPPL} provides programs to convert \textsc{APPLgrids} and \textsc{fastNLO} tables to \textsc{PineAPPL} grids.

\subsection{Cross sections in a multi-coupling expansion}
\label{sec:multi-coupling-expansion}
{\bf MZ comments:}
\begin{itemize}
    \item I would prefer to keep the notation similar to the aMCfast paper, or at least to start with that, and then generalise it
    \item the business of the different kinematics (born, resolved, counterterms, etc) is not mentioned at all
\end{itemize}
{\bf end MZ comments:}

Fixed-order partonic cross sections $a + b \to X$ supported by \textsc{PineAPPL} are expansions in powers of the strong coupling $\alphas$, the electromagnetic coupling $\alpha$, and the logarithms of $\xi_\mathrm{R} = \mu_\mathrm{R}^2 / Q^2$ and $\xi_\mathrm{F} = \mu_\mathrm{F}^2 / Q^2$,
\begin{equation}
\sigma_{ab} (x_1, x_2, \phi) = \sum_{k,l,m,n} \alphas^k \left( \xi_\mathrm{R} Q^2 \right) \alpha^l \log^m ( \xi_\mathrm{R} ) \log^n ( \xi_\mathrm{F} ) w_{ab}^{(k,l,m,n)} \left( x_1, x_2, Q^2, \phi \right) \text{.}
\label{eq:expansion}
\end{equation}
Above $Q^2$ denotes the Ellis--Sexton scale, which, if chosen dynamically, depends on the phase space $\phi$, $Q^2 = Q^2 (\phi)$.
If this is the case, then we assume the fractions $\xi_\mathrm{R}$ and $\xi_\mathrm{F}$ to be constants of phase space, however.
For the central scale choice, $\xi_\mathrm{R} = \xi_\mathrm{F} = 1$, the terms $m > 0$ and $n > 0$ are clearly not needed; they are only required if scale variations are desired.
For example, a seven-point scale variation sets the fractions to following values,
\begin{equation}
(\xi_\mathrm{R}, \xi_\mathrm{F}) \in \left\{ \bigl( 1, 1 \bigr), \bigl( \tfrac{1}{2}, \tfrac{1}{2} \bigr), \bigl( 2, 2 \bigr), \bigl( \tfrac{1}{2}, 1 \bigr), \bigl( 1, \tfrac{1}{2} \bigr), \bigl( 2, 1 \bigr), \bigl( 1, 2 \bigr) \right\} \text{,}
\end{equation}
and estimates the perturbative QCD uncertainty---no EW uncertainty is covered by this method---as the envelope of the cross section evaluated with the previous values.

As is clear from eq.~\eqref{eq:expansion}, the EW coupling $\alpha$ is assumed not to be a dynamically varying coupling, but instead a constant over phase space.
This, however, includes the most common choices of the coupling, which are (not necessarily in this order), $\alpha (0)$, $\alpha (M_\mathrm{Z})$, and $\alpha_{G_\mu}$.

The task that \textsc{PineAPPL} solves is to (approximately) reconstruct the function
\begin{equation}
w_{ab}^{(k,l,m,n)} \left( x_1, x_2, Q^2, \phi \right)
\end{equation}
from a set of $N$ function evaluations for specific phase-space points and momentum fractions
\begin{equation}
\left\{ \phi_i, x_1^{(i)}, x_2^{(i)}, Q^2_i \right\}_{i=1}^N \text{,}
\end{equation}
given by the Monte Carlo integrator.
Using eq.~\eqref{eq:expansion} and
\begin{equation}
\sigma (\xi_\mathrm{R}, \xi_\mathrm{R}) = \sum_{a,b} \int_0^1 \mathrm{d} x_1 \int_0^1 \mathrm{d} x_2 \int \mathrm{d} \phi \, f_a (x_1, \xi_\mathrm{F} Q^2) f_b (x_2, \xi_\mathrm{F} Q^2) \sigma_{ab} (x_1, x_2, \phi) \text{,}
\label{eq:pineappl-convolution}
\end{equation}
\textsc{PineAPPL} can calculate hadronic cross sections for arbitrarily many PDF sets and scale variations quickly.

\subsubsection{Example: Drell--Yan lepton-pair production}
\label{sec:pineappl-example}

To give an example of eq.~\eqref{eq:expansion}, the following shows Drell--Yan lepton-pair production up to terms at NLO (with arguments $x_1$, $x_2$, and $\phi$ suppressed for the phase-space weights):
\begin{equation}
\begin{split}
\sigma_{ab}
    &= \alpha^2 w_{ab}^{(0,2,0,0)} \\
    &+ \alphas \left( \xi_\mathrm{R} Q^2 \right) \alpha^2 w_{ab}^{(1,2,0,0)} + \alphas \left( \xi_\mathrm{R} Q^2 \right) \log (\xi_\mathrm{F}) \alpha^2 w_{ab}^{(1,2,0,1)} (Q^2) \\
    &+ \alpha^3 w_{ab}^{(0,3,0,0)} + \log (\xi_\mathrm{F}) \alpha^3 w_{ab}^{(0,3,0,1)} (Q^2) \text{.}
\end{split}
\end{equation}
The first term with index $(0,2,0,0)$ is the LO term, the following line shows the NLO QCD correction, and the final line the NLO EW correction.
Note that for all terms the dependence on the renormalisation scale is only indirectly through $\alphas$, because 1) higher-order terms in $\alpha$ never generate a renormalisation scale dependence (in the $\alpha$ schemes that are valid according to section~\ref{sec:multi-coupling-expansion}) and 2) higher-order QCD corrections only introduce an explicit renormalisation scale dependence in counterterms with vertices with more than two gluons.
These terms, however, are not present in this process at NLO.
Both NLOs, however, have a contribution from a collinear counterterm that depends on the factorisation scale.

% TODO: add graphical representation of the multi-coupling expansion

\subsubsection{Example: top-pair production}

TODO

\subsubsection{General characteristics}

In general, we define as leading order all terms for which the sum of the coupling exponents in eq.~\eqref{eq:expansion} is smallest, i.e.\ $k + l = p$, where $p = \min (k+l)$.
This number is process dependent and usually determined by the number of external particles.
For many processes there is only one LO, but when a process has multiple quark lines, colourless (photons, \dots) and coloured particles (gluons, \ldots) can be exchanged between them, which allows for more than one leading order.
To each leading order a higher-order correction with an additional power of $\alphas$ or $\alpha$ can be calculated, which in general leads to at least two next-to-leading order corrections.
Sometimes there are higher orders that cannot directly be understood as a correction to a leading order, e.g.\ $\mathrm{p} \mathrm{p} \to \ell \bar{\ell} + \mathrm{jet} + X$, which has one leading order, $\mathcal{O} (\alphas \alpha^2)$, but three next-to-leading orders, one of which is $\mathcal{O} (\alpha^4)$~TODO.

Due to typical size of the couplings $\alphas^2 \sim \alpha$, it is naively expected that within the same order, i.e.\ for fixed $k + l$, terms with larger powers $\alphas^k$ dominate over those with smaller powers; however, in practise this naive expectation is not always true due to dynamic effects.
Some prominent examples are vector-boson scattering processes and TODO.

\subsection{Grid representation and accuracy}
\label{sec:grid-representation}

In this subsection we explain the details of how the phase-space weights $w_{ab}$ in eq.~\ref{eq:expansion} are represented.

\subsubsection{4-tuples}

One very simple representation are 4-tuples, which for each phase-space point and combination $(a, b, k, l, m, n)$ saves a list of the parton fractions $x_1$ and $x_2$, the scale $Q^2$, and the phase-space weight $w$; this is sufficient for a total cross section.
If, however, a differential cross section shall be reconstructed, we first need to generalise eq.~\eqref{eq:expansion}:
\begin{equation}
\frac{\mathrm{d} \sigma_{ab}}{\mathrm{d} \mathcal{O}} (x_1, x_2, \phi, \mathcal{O}) = \sum_{k,l,m,n} \alphas^k \left( \xi_\mathrm{R} Q^2 \right) \alpha^l \log^m ( \xi_\mathrm{R} ) \log^n ( \xi_\mathrm{F} ) \frac{\mathrm{d} w_{ab}^{(k,l,m,n)}}{\mathrm{d} \mathcal{O}} \left( x_1, x_2, Q^2, \phi \right) \\
\end{equation}
where, because of a finite number of phase-space points given by the Monte Carlo integrator, we numerically approximates the derivative as
\begin{equation}
\begin{split}
\frac{\mathrm{d} w_{ab}^{(k,l,m,n)}}{\mathrm{d} \mathcal{O}} \left( x_1, x_2, Q^2, \phi \right) &\approx \sum_{o=1}^M \frac{\Theta (\mathcal{O}_o^\mathrm{min} \le \mathcal{O} < \mathcal{O}_o^\mathrm{max})}{\mathcal{O}_o^\mathrm{max} - \mathcal{O}_o^\mathrm{min}} w_{ab}^{(k,l,m,n)} \left( x_1, x_2, Q^2, \phi \right) \\
&= \sum_{o=1}^M w_{ab}^{(k,l,m,n,o)} \left( x_1, x_2, Q^2, \phi, \mathcal{O} \right) \text{,}
\end{split}
\end{equation}
using $M$ discrete bins with boundaries $\{ \mathcal{O}_o^\mathrm{min}, \mathcal{O}_o^\mathrm{max} \}_{o=1}^M$.

This means that for each bin $o$ of the observable $\mathcal{O}$ and therefore for each combination $(a,b,k,l,m,n,o)$ we save the following 4-tuples:
\begin{equation}
\left\{ x_1^i, x_2^i, Q^2_i, w^{(k,l,m,n,o)}_{ab} (x_1^i, x_2^i, Q^2_i, \phi_i, \mathcal{O}_i) \right\}_{i=1}^N \text{,} \label{eq:four-tuples}
\end{equation}
The reconstruction of the differential cross section for bin $o$ is then straightforwardly done by multiplying the phase-space weights $w$ with PDFs evaluated with the correct arguments given in the 4-tuple and summing over all indices $a$, $b$, $k$, $l$, $m$, $n$, and $i$.

Using 4-tuples has the clear advantage of being very easy to implement and therefore also easy to test.
Furthermore, they reproduce the exact numerical value that is also calculated by the Monte Carlo integrator.
However, the price one has to pay is the size of the 4-tuples, which scales linearly with the number of phase-space weights $N$.
Because the uncertainties of Monte Carlo integrators shrink as $1/\sqrt{N}$, this typically means \numrange{e7}{e9} phase-space points.
In the case of the NLO Drell--Yan (section~\ref{sec:pineappl-example}) the 4-tuples need roughly TODO~\si{\giga\byte} of storage (see appendix~\ref{app:drell-yan-storage} for an explanation of this number).
With increasing size also the speed of the convolution degrades, because it basically becomes bound by the speed with which the 4-tuples can be read from disk.

\subsubsection{Interpolation grids}

A different choice partition the $(x_1, x_2, Q^2)$ space,
\begin{equation}
H = [x_\mathrm{min},x_\mathrm{max}]^2 \times [Q^2_\mathrm{min}, Q^2_\mathrm{max}] \ni (x_1, x_2, Q^2)
\end{equation}
along each axis into a small numbers of bins and to insert the phase-space weights $w$ into the corresponding discrete bin.
This method is straightforward, but given a finite number of bins this usually yields a bad approximation for the cross section.
This can be remedied by increasing the number of bins, which would also increase the space requirements of the grid.
There are, however, more refined methods, which keep the number of bins constant, while improving the quality of the approximation: Interpolation grids.

TODO
