In this paper we introduce a new library called \textsc{PineAPPL}, which is similar to \textsc{APPLgrid}~\cite{Carli:2010rw} and \textsc{fastNLO}~\cite{Kluge:2006xs,Wobisch:2011ij,Britzger:2012bs}.
Like to those two libraries, also \textsc{PineAPPL} bins phase-space weights in a PDF-independent manner, but it understands EW corrections, which are the main interest in this paper.
The following features distinguish it:
\begin{itemize}
\item Support for arbitrary fixed-order calculations in powers of $\alpha$, $\alpha_\mathrm{s}$ or combinations thereof, e.g.\ in mixed QCD-EW corrections.
Furthermore variations of the renormalization and factorization scale are supported, if needed.
For each needed combination of the couplings and logarithms of renormalization and factorization scale a separate subgrid is created (see Sec.~\ref{sec:grid-representation} for more details);
\item Support for all-order predictions coming from a resummation calculation or a photon-/parton-shower, which are important for some observables (see Sec.~\ref{sec:results}),
\item A simple \textsc{C}-interface, with a wrapper for \textsc{Fortran} and \textsc{Python} (see App.~\ref{app:pineappl-interface} for examples and documentation), which is needed for Monte Carlos and programs to read and write \textsc{PineAPPL} grids.
\textsc{PineAPPL} itself is written in Rust.
\end{itemize}
For \textsc{mg5\_aMC@NLO}~\cite{Alwall:2014hca,Frederix:2018nkq} the interfacing code is already implemented in the most recent version, which replaces the \textsc{aMCfast}~\cite{Bertone:2014zva} interface.
The interfacing code for other Monte Carlo generators should be easy to write, see App.~\ref{app:pineappl-interface} for a small example program.
Finally, \textsc{PineAPPL} provides programs to convert \textsc{APPLgrids} and \textsc{fastNLO} tables to \textsc{PineAPPL} grids.

\subsubsection{Cross sections in a multi-coupling expansion}
The structure of the cross-section weights needed by \textsc{PineAPPL} follows the scheme outlined in
Refs.~\cite{Frederix:2011ss, Bertone:2014zva}, embedded in a mixed-coupling expansion, see Ref.~\cite{Frederix:2018nkq} and Refs.CITE for
specific examples. Starting
from the latter, an observable $\Sigma(\alphas, \alpha)$ is written as
\begin{equation}
    \Sigma(\alphas, \alpha) = \alpha^c \alphas^{c_S} \sum_{a_S, a} \alphas^{a_S} \alpha^a \Sigma_{a_S, a}\,.
\end{equation}
$c$ and $c_S$ are process-dependent; the contributions $\Sigma_{a_S, a}$ are in general non-zero for $a_S, a \ge 0$ and $a_S + a > q$, where also $q$ is a process-dependent
uantity. For example,
for (stable) top-pair production, $c=c_S=0$ and $q=2$; for Drell-Yan production, $c=2$, $c_S=0$, and $q=0$. Terms
with  $a_S + a = q + k$ correspond to different N$^k$LO contributions to $\Sigma$, usually
labeled as N$^k$LO$_i$, $i =1,2, \ldots$, with $i=1$ assigned to the term with the largest power of $\alphas$. Given the hierarchy of the couplings,
one expects that
\begin{equation}
 \textrm{N}^k\textrm{LO}_1 \gg \textrm{N}^k\textrm{LO}_2 \gg\ldots \, ,
\end{equation}
however such a relation is not always respected, and sometimes blatantly violated~CITE.

Given the perturbative order (LO, NLO, \ldots), different weights $\cal W$, with different kinematics expressed by the set $\cal P$, enter the various
terms $\Sigma_{a_S, a}$. One can write
\begin{equation}
    \Sigma_{a_S, a}= \sum_{l\in \mathcal P} f_1(x_1^{(l)},\mu_F^{(l)}) \,f_2(x_2^{(l)},\mu_F^{(l)}) \mathcal W^{(l)}_{a_S, a}
    d \textrm{PS}\,,
\end{equation}
where we have introduced the parton distributions $f_{1,2}$ and the phase-space measure $d$PS. The structure of the weights $ \mathcal W^{(l)}_{a_S, a}$ is 
identical to the case when only correction of QCD origin are treated. In particular:
\begin{itemize}
    \item At LO, $\mathcal P = \{B\}$, $B$ being the Born kinematics, and the weight structure is trivial,
    \begin{equation}
        \mathcal W^{(l)}_{a_S, a} = {\mathcal W^{(l),0}_{a_S, a}}\,.
    \end{equation}
    \item At NLO, $\mathcal P = \{E, S, C, SC\}$ with, $E$ being the event (or resolved) kinematics, and $S$, $C$, $SC$ being
        respectively the soft, collinear, and soft-collinear kinematics. The weight structure includes three terms at NLO:
    \begin{equation}
        \mathcal W^{(l)}_{a_S, a} = {{\mathcal W}^{(l),0}_{a_S, a}} +
                                {\mathcal W^{(l),R}_{a_S, a}} \log\left(\frac{{\mu_R^{(l)}}^2}{Q^2}\right) +
                                {\mathcal W^{(l),F}_{a_S, a}} \log\left(\frac{{\mu_F^{(l)}}^2}{Q^2}\right)
    \end{equation}
    where, besides the renormalisation and factorisation scales $\mu_{R,F}$, we have introduced the Ellis-Sexton scale $Q$
\end{itemize}

\subsection{Grid representation and accuracy}
\label{sec:grid-representation}

\subsection{Accuracy and performance}
\label{sec:accuracy-and-performance}

CS, MZ

How to integrate in MC codes, how this works in aMCblast (setup, MG5 version, urls).\\
Appendix: how to create and fill grids, + simple code.
