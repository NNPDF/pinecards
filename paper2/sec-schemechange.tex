\section{Scheme-change operator}
\label{sec:scheme-change-operator}

At LO in $\alpha$, the coupling $\alpha$ in the PDF (as used in the evolution equations) and in the fixed-order calculations can be chosen independently, if no initial-state photons are present.
\textbf{TODO}: Marco insert discussion here, and change the next sentence accordingly.

This is important, because in the PDFs the coupling is typically assumed to be renormalised in the $\overline{\mathrm{MS}}$ scheme, while in fixed-order calculations $\alpha$ is usually chosen according to the $G_\mu$ scheme.

At NLO in $\alpha$, this is---at least conceptually---no longer trivially possible: in the fixed-order calculation a collinear counter term is present, which implements the PDF redefinition and, in particular, subtracts initial-state QED singularities.
From the PDF redefinition the evolution equations can be derived, which makes clear that there is a non-trivial connection in the choice of $\alpha$ in 1) the evolution equations and 2) the collinear counter term.
Note this only concerns the additional power in $\alpha$ in the counter term, as this corresponds to the $\alpha$ in the evolution equations.
The remaining powers in $\alpha$ coming from the LO are disconnected from the DGLAP equations, as argued above, and can therefore still be chosen independently.

For the additional power in $\alpha$ we therefore have a mismatch between the evolution equations and the collinear counter term in the fixed-order calculation.
We can correct for this mismatch by introducing a new term in the fixed-order calculation, which is the difference between the collinear counter term with a running $\overline{\mathrm{MS}}$ $\hat{\alpha} (\mu_\mathrm{R})$, and the counter term actually used.
If we assume for the following that we use a $G_\mu$ input scheme for $\alpha$, this term is the collinear counter term, with the additional power in $\alpha$ replaced by the difference of $\alpha_{G_\mu}$ and the $\overline{\text{MS}}$ coupling $\hat{\alpha} (\mu_\mathrm{R}^2)$:
\begin{equation}
\hat{\alpha} (\mu_\mathrm{R}^2) - \alpha_{G_\mu} = \alpha_{G_\mu} \left( \Delta \hat{\alpha} (\mu_\mathrm{R}^2) - \Delta r^{(1)} + \ldots \right) \text{,} \label{eq:consistency-condition}
\end{equation}
which uses the following definitions,
\begin{equation}
\hat{\alpha} (\mu_\mathrm{R}^2) = \frac{\alpha(0)}{1 - \Delta \hat{\alpha} (\mu_\mathrm{R}^2)} \text{,} \qquad \alpha_{G_\mu} = \alpha(0) \left( 1 + \Delta r^{(1)} \right) + \mathcal{O} (\alpha(0)^3) \text{,}
\end{equation}
which in make use of the fine-structure constant $\alpha(0)$; all quantities are given e.g.\ in ref.~\cite{Denner:2019vbn} and references therein.
Since all $\Delta$ terms are $\mathcal{O} (\alpha)$, the difference in eq.~\eqref{eq:consistency-condition} is $\mathcal{O} (\alpha^2)$ and therefore zero at NLO EW in the full perturbative calculation.
This must be the case, otherwise the scheme-change operator we defined above would introduce additional singularities.
