\section{PDF-independent binning of phase-space weights with \texorpdfstring{\textsc{PineAPPL}}{PineAPPL}}
\label{sec:pineappl}

In this paper we introduce a new library called \textsc{PineAPPL}, which bins phase-space weights independently from the chosen PDF set.
The files produced in this way, generally called grids, can be used to quickly evaluate several observables and also to assess the impact of different PDF sets and their PDF uncertainties.
Finally they are the main theoretical input to a PDF determination.
In that sense \textsc{PineAPPL} is similar to \textsc{APPLgrid}~\cite{Carli:2010rw} and \textsc{fastNLO}~\cite{Kluge:2006xs,Wobisch:2011ij,Britzger:2012bs}, but it also understands EW, which are the main interest in this paper.
The following features distinguish it:
\begin{itemize}
\item Support for arbitrary fixed-order calculations in powers of $\alpha$, $\alphas$ or combinations thereof, e.g.\ in mixed QCD-EW corrections.
Furthermore, variations of the renormalisation and factorisation scale are supported, if needed.
For each needed combination of the couplings and logarithms of renormalisation and factorisation scale a separate subgrid is created (see section~\ref{sec:multi-coupling-expansion} for more details);
\item Support for all-order predictions coming from a resummation calculation or a photon-/parton-shower, which are important for some observables (see section~\ref{sec:results}),
\item A simple \textsc{C}-interface, with a wrapper for \textsc{Fortran} and \textsc{Python} (see appendix~\ref{app:example-program} for examples and documentation), which is needed for Monte Carlos and programs to read and write \textsc{PineAPPL} grids.
\textsc{PineAPPL} itself is written in Rust (see appendix~\ref{app:installation} for installation instructions).
\end{itemize}
For \textsc{mg5\_aMC@NLO}~\cite{Alwall:2014hca,Frederix:2018nkq} the interfacing code is already implemented in the most recent version, which replaces the \textsc{aMCfast}~\cite{Bertone:2014zva} interface.
The interfacing code for other Monte Carlo generators should be easy to write, see appendix~\ref{app:example-program} for a small example program.
Finally, \textsc{PineAPPL} provides programs to convert \textsc{APPLgrids} and \textsc{fastNLO} tables to \textsc{PineAPPL} grids.

\subsection{Cross sections in a multi-coupling expansion}
\label{sec:multi-coupling-expansion}
{\bf MZ comments:}
\begin{itemize}
    \item I would prefer to keep the notation similar to the aMCfast paper, or at least to start with that, and then generalise it
    \item drop eq 2.2, start already by 2.1 with a symmetric treatment of $\alpha$ and $\alphas$ (plus, the symbol $\sigma$ is overloaded there)
    \item $Q$ is the Ellis-Sexton scale, a (unphysical) scale introduced to have all logs in the form $\log(\mu_R/F / Q)$, and hence to separate
        the dependence of $\mu_R$ and $\mu_F$
    \item the business of the different kinematics (born, resolved, counterterms, etc) is not mentioned at all
    \item eq 2.1 assumes $\sigma_{ab}$ integrated over phase-space?
\end{itemize}
{\bf end MZ comments:}

Fixed-order partonic cross sections supported by \textsc{PineAPPL} are expansions in powers of the strong coupling $\alphas$, the electromagnetic coupling $\alpha$, and, if scale variations are desired, also in the logarithms of $\xi_\mathrm{R} = \mu_\mathrm{R}^2 / Q^2$ and $\xi_\mathrm{F} = \mu_\mathrm{F}^2 / Q^2$,
\begin{equation}
\sigma_{ab} (x_1, x_2, Q^2; \xi_\mathrm{R}, \xi_\mathrm{F}) = \sum_{k,l,m,n} \alphas^k \left( \xi_\mathrm{R} Q^2 \right) \log^m ( \xi_\mathrm{R} ) \log^n ( \xi_\mathrm{F} ) w_{ab}^{(k,l,m,n)} \left( x_1, x_2, Q^2 \right) \text{,}
\label{eq:expansion}
\end{equation}
with the phase-space weights $w$ defined as
\begin{equation}
w_{ab}^{(k,l,m,n)} \left( x_1, x_2, Q^2 \right) = \alpha^l \sigma_{ab}^{(k,l,m,n)} \left( x_1, x_2, Q^2 \right) \text{.}
\label{eq:phase-space-weight}
\end{equation}
The left-hand side of eq.~\eqref{eq:expansion} shows the partonic cross section for a process $a + b \to X$, where the parton $a$ has momentum fraction $x_1$ and $b$ has momentum fraction $x_2$.
The central value of the renormalisation and factorisation scale, $Q^2$, is treated as an independent variable; the parameters $\xi_\mathrm{R}$ and $\xi_\mathrm{F}$ allow varying the scales around the central value.

Eq.~\eqref{eq:phase-space-weight} contains the phase-space weights that we assume are calculated numerically, e.g.\ with a Monte Carlo generator, and then passed to \textsc{PineAPPL}, which will store this information either approximately or exactly, depending on the chosen format (see section~\ref{sec:grid-representation}).
In particular, this phase-space weight is the product of
\begin{itemize}
\item the electroweak coupling $\alpha$, which we assume does not depend on either $x_1$, $x_2$, or $Q^2$ --- this is the case for the most prominent choices of $\alpha$, namely $\alpha (0)$, $\alpha (M_\mathrm{Z})$, and $\alpha_{G_\mu}$, but is not the case when $\alpha (\mu)$ is a (dynamic) scale-dependent coupling --- and
\item the rest of the partonic cross section, denoted using a multi index $(k,l,m,n)$, where each index is the exponent of either a coupling or a logarithm.
\end{itemize}
Having stored eq.~\eqref{eq:phase-space-weight}, this allows \textsc{PineAPPL} to quickly calculate
\begin{equation}
\sigma (\xi_\mathrm{R}, \xi_\mathrm{R}) = \sum_{a,b} \int_0^1 \mathrm{d} x_1 \int_0^1 \mathrm{d} x_2 \int_0^1 \mathrm{d} y \, f_a (x_1, \xi_\mathrm{F} Q^2) f_b (x_2, \xi_\mathrm{F} Q^2) \sigma_{ab} (x_1, x_2, Q^2) \text{.}
\label{eq:pineappl-convolution}
\end{equation}
where $Q^2 = Q^2_\mathrm{min} + (Q^2_\mathrm{max} - Q^2_\mathrm{min}) y$, to obtain the hadronic cross sections for arbitrarily many PDF sets and scale variations.

\subsubsection{Example}
\label{sec:pineappl-example}

To give an example of eq.~\eqref{eq:expansion}, the following shows Drell--Yan lepton-pair production up to terms at NLO:
\begin{equation}
\begin{split}
\sigma_{ab} (\xi_\mathrm{R}, \xi_\mathrm{F})
    &= \left[ \alpha^2 \sigma_{ab}^{(0,2,0,0)} \right] \\
    &+ \alphas \left( \xi_\mathrm{R} Q^2 \right) \left[ \alpha^2 \sigma_{ab}^{(1,2,0,0)} \right] + \alphas \left( \xi_\mathrm{R} Q^2 \right) \log (\xi_\mathrm{F}) \left[ \alpha^2 \sigma_{ab}^{(1,2,0,1)} (Q^2) \right] \\
    &+ \left[ \alpha^3 \sigma_{ab}^{(0,3,0,0)} \right] + \log (\xi_\mathrm{F}) \left[ \alpha^3 \sigma_{ab}^{(0,3,0,1)} (Q^2) \right] \text{.}
\end{split}
\end{equation}
The first term with index $(0,2,0,0)$ is the LO term, the next line shows the NLO QCD correction, and the final line the NLO EW correction.
Note that the dependence on the renormalisation scale is only indirectly through $\alphas$, because the LO does not have any gluons in the initial or final state.
Terms proportional to $\log (\xi_\mathrm{R})$ vanish, because they require an explicit dependence on the renormalisation scale, which only enters through counterterms with more than two gluons.

\subsubsection{General characteristics}

In general, we define as leading order all terms for which the sum of the coupling exponents in eq.~\eqref{eq:expansion} is smallest, i.e.\ $k + l = p$, where $p = \min (k+l)$.
This number is process dependent and usually determined by the number of external particles.
For many processes there is only one LO, but when a process has multiple quark lines, colourless (photons, \dots) and coloured particles (gluons, \ldots) can be exchanged between them, which allows for more than one leading order.
To each leading order a higher-order correction with an additional power of $\alphas$ and $\alpha$ can be calculated, which in general leads to at least two next-to-leading order corrections.
Sometimes there are higher orders that cannot directly be understood as a correction to a leading order, e.g.\ $\mathrm{p} \mathrm{p} \to \ell \bar{\ell} + \mathrm{jet} + X$, which has one leading order, $\mathcal{O} (\alphas \alpha^2)$, but three next-to-leading orders, one of which is $\mathcal{O} (\alpha^4)$~TODO.

Due to typical size of the couplings $\alphas^2 \sim \alpha$, it is naively expected that within the same order, i.e.\ for fixed $k + l$, terms with larger powers $\alphas^k$ dominate over those with smaller powers; however, in practise this naive expectation is not always true due to dynamic effects.
Some prominent examples are vector-boson scattering processes and TODO.

\subsection{Grid representation and accuracy}
\label{sec:grid-representation}

So far we did not explain how the phase space-weights, eq.~\ref{eq:phase-space-weight}, are represented.
An obvious choice are $n$-tuples, which for each observable and for each combination $(a, b, k, l, m, n)$ saves a list of $N$ 4-tuples,
\begin{equation}
\left\{ x_1^i, x_2^i, Q^2_i, \frac{\mathrm{d}}{\mathrm{d} \phi_i} w^{(k,l,m,n)}_{ab} (x_1^i, x_2^i, Q^2_i) \right\}_{i=1}^N \text{.}
\end{equation}
The last element in the tuple is not an integrated cross section as in eq.~\eqref{eq:phase-space-weight}, but rather fully differential in the phase space $\phi$, evaluated at a specific phase-space point $\phi_i$.
This is due to the fact that Monte Carlo integrators do not perform the phase space integrals and the convolution with the PDFs separately, but together at the same time.

In case of $n$-tuples, the reconstruction of the integrated cross section, eq.~\eqref{eq:pineappl-convolution}, is very straightforward, for example for the central scale choice,
\begin{equation}
w_{ab}^{(k,l,m,n)} = \sum_{i=1}^N f_a (x_1^i, Q^2_i) f_b (x_2^i, Q^2) \alphas (Q^2) \frac{\mathrm{d}}{\mathrm{d} \phi_i} w^{(k,l,m,n)}_{ab} (x_1^i, x_2^i, Q^2_i) \text{,} \label{eq:n-tuple-integration}
\end{equation}
given a proper normalization of the integral measures; this equation is the same approximation that a Monte Carlo integrator evaluates.
For the full hadronic cross section it is only required to sum eq.~\eqref{eq:n-tuple-integration} over all open indices.

This method has the clear advantage of being able to reproduce exactly the numerical value of the generator.
However, the price to pay for is in large storage requirements.
In the case of the NLO Drell--Yan (section~\ref{sec:pineappl-example}) the $n$-tuples need roughly \SI{7030}{\giga\byte} of storage (see appendix~\ref{app:drell-yan-storage} for an explanation of this number).

A different choice is to build an interpolation grid, which basically partitions the space
\begin{equation}
H = [x_\mathrm{min},x_\mathrm{max}]^2 \times [Q^2_\mathrm{min}, Q^2_\mathrm{max}] \ni (x_1, x_2, Q^2)
\end{equation}
along each axis into a small numbers of bins, which allows one to approximately recover the result of the cross section.
