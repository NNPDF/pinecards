\documentclass[a4paper,11pt]{article}
\pdfoutput=1

\usepackage{jheppub}
\usepackage[T1]{fontenc}



\title{Including EW and mixed QCD-EW corrections into PDF fits}

\author[a,b]{S. Carrazza,}
\author[b]{E. Nocera,}
\author[a]{C. Schwan}
\author[a,b]{and M. Zaro}

\affiliation[a]{Tif Lab, Dipartimento di Fisica, Universit\`a di Milano and INFN, Sezione di Milano, 20133 Milano, Italy}
\affiliation[b]{Nikhef Theory Group, Science Park 105, 1098 XG Amsterdam, The Netherlands}

\emailAdd{stefano.carrazza@mi.infn.it}
\emailAdd{e.nocera@nikhef.nl}
\emailAdd{christopher.schwan@mi.infn.it}
\emailAdd{marco.zaro@mi.infn.it}

\abstract{}

\begin{document}

\maketitle
\flushbottom

\section{Introduction}
\label{sec:introduction}

ERN

The impact of EW corrections to LHC data.\\
How to do that consistently, include systematically EW corrections, blabla.\\
Outline a possible strategy to deal with EW correction in LHC observables (FSR, PHOTOS, dressed/born, ...).\\
Pinappl kills amcfast and mcgrid, replace applgrid and fastnlo.\\

\section{Technical}

CS, MZ

Describe pineappl, representation of grids, accuracy and performance.\\
We may provide converters from applgrid.\\
How to integrate in MC codes, how this works in aMCblast (setup, MG5 version, urls).\\
Appendix: how to create and fill grids, + simple code.\\

\section{Results}

SC

For datasets in 3.1qed (+DY2), compute pure NLO QCD (born) vs NLO QCD + NLO EW (dressed) FKtables and:\\
- compute data/prediction plots\\
- compute chi2\\

\section{Double counting}

Count von Count, ALL

Explain the problem to experimentalists.\\

\section{Outlook}

ALL

\cite{Carli:2010rw}
\cite{Bertone:2014zva}

\appendix

\acknowledgments

C.S. is supported by the European Research Council under the European Unions Horizon 2020 research and innovation Programme (grant agreement no. 740006).

\bibliographystyle{JHEP}
\bibliography{paper}

\end{document}
