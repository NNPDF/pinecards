\section{Datasets}
\label{sec:datasets}

We will now discuss potential datasets for an NLO EW PDF fit.

\subsection{Drell--Yan lepton-pair production}
\label{sec:dy-at-the-lhc}

The inclusive production of a single same-flavour opposite-sign (SFOS) lepton pair, $\mathrm{p}\mathrm{p} \to \ell \bar{\ell} + \mathrm{X}$, is an important ingredient in PDF fitting.
On the experimental side this process features large cross sections, which translate to small experimental uncertainties.
On the other hand, the comparative simplicity of the $2 \to 2$ process allows for the calculation of NNLO QCD corrections, which are needed to match the experimental precision.
Finally, we do not expect any new physics in a very large phase space covered by Drell--Yan (DY) measurements.
This makes DY an ideal candidate for the inclusion in PDF determinations.

Observables often used in DY datasets are the invariant mass, $M_{\ell\bar{\ell}}$, and the rapidity, $y_{\ell\bar{\ell}}$, both for the lepton pair.
At LO they are directly related to the parton fractions $x_1$ and $x_2$:
\begin{equation}
M_{\ell\bar{\ell}} = \sqrt{s x_1 x_2} \text{,} \quad y_{\ell\bar{\ell}} = \frac{1}{2} \log \left( \frac{x_1}{x_2} \right) \text{.}
\end{equation}
Less-used observables include the Collins--Soper angle, $\cos \theta^*$, see e.g.\ ref.~\cite{Aaboud:2017ffb}, and see the discussion in app.~\ref{app:the-collins-soper-angle}.

In the context of EW PDF fits, the following features of are new with respect to a purely QCD fit:
\begin{enumerate}
\item Non-zero photon PDFs.
DY lepton-pair production features a contribution with two photon initial states.
These contributions are typically neglected in NNLO QCD PDF fits, and, also subtracted in data from the experimental collaborations.
The size of this contribution depends---obviously---on the PDF set used in the calculation, and can be as high as TODO in the region of $\SI{1}{\tera\electronvolt} < M_{\ell\bar{\ell}} < \SI{1.5}{\tera\electronvolt}$ for $\sqrt{s} = \SI{7}{\tera\electronvolt}$.
\item Final-state radiation (FSR).
The radiation of photons off the final-state leptons induce very large corrections, typically in the invariant mass distributions $M_{\ell\bar{\ell}}$ in the vicinity of the Z-boson mass.
For example, for the CMS \SI{13}{\tera\electronvolt} measurement~\cite{Sirunyan:2018owv}, the NLO EW corrections (which include a perturbative approximation of the FSR corrections) are as large as \SI{60}{\percent}.
However, the size depends very strongly on the chosen bin limits; often bin limits are chosen symmetrically around the Z-boson mass, in which positive and negative corrections almost cancel each other.
The remainder of large FSR corrections are then only seen in the bins below the Z peak.

Since FSR corrections are potentially large, but not covered by NNLO QCD corrections, they are often subtracted in data.
The problems of this practice are discussed e.g.\ in ref.~\cite{Carrazza:2020gss}.
\item Weak corrections.
Above a certain invariant mass of the lepton pair, say at \SI{200}{\giga\electronvolt}, the electroweak corrections show the typical negative corrections, which can be as large as TODO around \SI{1}{\tera\electronvolt}.
In NNLO QCD PDF fits this regions is therefore usually excluded.
\end{enumerate}


