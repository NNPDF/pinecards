\section{Installation and usage of \texorpdfstring{\textsc{PineAPPL}}{PineAPPL}}
\label{app:pineappl}

%\subsection{Efficiency of the grid representation}
%\label{app:pineappl-representation-efficiency}
%
%Using a similar notation introduced in eq.~\eqref{eq:expansion}, we single out a specific NLO correction:
%\begin{equation}
%\begin{split}
%\alphas^k \bigl( \mu_\mathrm{R}^2 \bigr) \alpha^l \Bigl[ \log \bigl( \mu_\mathrm{R}^2 \bigr) &W_{ab}^{(k,l,1,0)} (x_1, x_2) \\
%+ \log \bigl( \mu_\mathrm{F}^2 \bigr) &W_{ab}^{(k,l,0,1)} (x_1, x_2) \\
%&W_{ab}^{(k,l,0,0)} (x_1, x_2) \Bigr] \text{.}
%\end{split}
%\end{equation}
%If we assume the case of a dynamic scale choice, i.e.\ $\mu_\mathrm{R}^2 \equiv \mu_\mathrm{R}^2 \bigl( x_1, x_2 \bigr)$ and $\mu_\mathrm{F}^2 \equiv \mu_\mathrm{F}^2 \bigl( x_1, x_2 \bigr)$,

\subsection{Installation}
\label{app:installation}

\textsc{PineAPPL} currently consists of three parts: 1) the library itself, which is a dependency for the other parts, 2) the helper program \texttt{pineappl}, which allows to read \textsc{PineAPPL} grids from the command line and make predictions with it, and finally 3) the C interface, which is intended to be used in Monte Carlo integrators to generate the grids.

\subsubsection*{Installation of Rust}

\lstset{
  basicstyle=\ttfamily\small,
  showstringspaces=false,
  commentstyle=\color{red},
  keywordstyle=\color{blue}
}

All parts are written in Rust and therefore need a Rust compiler and other tools.
On computers with a \texttt{bash} shell the installation is as simple as
\begin{lstlisting}[language=bash]
 $ curl --proto '=https' --tlsv1.2 -sSf https://sh.rustup.rs | sh
\end{lstlisting}
which downloads the compiler \texttt{rustc}, the package manager \texttt{cargo}, and a few other helpful tools.
When the installation has completed make sure to read and follow the instructions printed before.
See also \url{https://www.rust-lang.org/tools/install} for more details and for installation instructions for other systems.

\subsubsection*{Installation of the command-line program \texorpdfstring{\texttt{pineappl}}{pineappl}}

The command-line program \texttt{pineappl} is compiled and installed using
\begin{lstlisting}[language=bash]
 $ cargo install pineappl_cli
\end{lstlisting}
This program also needs \texttt{LHAPDF} \cite{Buckley:2014ana} installed.
For usage instructions simply type \texttt{pineappl} in your shell and read the help message.

\subsubsection*{Installation of the C-language interface (optional)}

For the C interface you need to first install \texttt{cargo-c},
\begin{lstlisting}[language=bash]
 $ cargo install cargo-c
\end{lstlisting}
and then download the \textsc{PineAPPL} repository, compile and finally install into it into a directory \texttt{\$prefix} as follows:
\begin{lstlisting}[language=bash]
 $ git clone https://github.com/N3PDF/pineappl/
 $ cd pineappl_capi/
 $ cargo cinstall --release --prefix=DIRECTORY
\end{lstlisting}
The last line will install the C header \texttt{pineappl\_capi.h}, the library, and a pkg-config\footnote{A standard way on Linux to express how dependencies are compiled/linked against, see \url{https://www.freedesktop.org/wiki/Software/pkg-config/}} file (\texttt{pineappl\_capi.pc}) into the directory specified as \texttt{DIRECTORY}.
Make sure that the environment variables \texttt{PATH}, \texttt{LD\_LIBRARY\_PATH}, and \texttt{PKG\_CONFIG\_PATH} are properly set.
The latter is needed for \texttt{pkg-config -{}-cflags -{}-libs pineappl\_capi} to work, which prints the necessary compiler/linker flags.

After being installed, you can compile and link against the library.
See App.~\ref{app:example-program} for an example.

Updated installation instructions are kept in the file \texttt{README.md} in \textsc{PineAPPL}'s repository at \url{https://github.com/N3PDF/pineappl/}.

\subsection{Example Monte Carlo program}
\label{app:example-program}

\definecolor{mygreen}{rgb}{0,0.6,0}
\definecolor{mygray}{rgb}{0.5,0.5,0.5}
\definecolor{mymauve}{rgb}{0.58,0,0.82}

\lstset{
  belowcaptionskip=1\baselineskip,
  breaklines=true,
  frame=L,
  numbers=left,
  xleftmargin=\parindent,
  showstringspaces=false,
  basicstyle=\footnotesize\ttfamily,
  keywordstyle=\bfseries\color{green!40!black},
  commentstyle=\itshape\color{purple!40!black},
  identifierstyle=\color{blue},
  stringstyle=\color{orange},
}

The following listing shows how to setup \textsc{PineAPPL} using its C interface in a simple Monte Carlo integrator for calculating the double-photon contribution to Drell--Yan lepton-pair production at the LHC.
All \textsc{PineAPPL} functions have the prefix \texttt{pineappl\_}.
The full example can be found at \url{https://github.com/N3PDF/pineappl/tree/master/examples/capi-dy-aa}.

\begin{lstlisting}[language=C++,mathescape=true]
// create a new luminosity function for the $\gamma\gamma$ initial state
auto* lumi = pineappl_lumi_new();
int32_t pdg_ids[] = { 22, 22 };
double ckm_factors[] = { 1.0 };
pineappl_lumi_add(lumi, 1, pdg_ids, ckm_factors);

// only LO $\alpha_\mathrm{s}^0 \alpha^2 \log^0(\xi_\mathrm{R}) \log^0(\xi_\mathrm{F})$
uint32_t orders[] = { 0, 2, 0, 0 };

// we bin in rapidity from 0 to 2.4 in steps of 0.1
double bins[] = {
    0.1, 0.2, 0.3, 0.4, 0.5, 0.6, 0.7, 0.8, 0.9, 1.0, 1.1, 1.2,
    1.3, 1.4, 1.5, 1.6, 1.7, 1.8, 1.9, 2.0, 2.1, 2.2, 2.3, 2.4
};

// create the PineAPPL grid with default interpolation and binning parameters
auto* keyval = pineappl_keyval_new();
auto* grid = pineappl_grid_new(lumi, 1, orders, 24, bins, keyval);

// now we no longer need `keyval` and `lumi`
pineappl_keyval_delete(keyval);
pineappl_lumi_delete(lumi);

// fill the grid with phase-space points
fill_grid(grid, 10000000);

// perform a convolution of the grid with PDFs
auto* pdf = LHAPDF::mkPDF("NNPDF31_nlo_as_0118_luxqed", 0);
auto xfx = [](int32_t id, double x, double q2, void* pdf) {
    return static_cast <LHAPDF::PDF*> (pdf)->xfxQ2(id, x, q2);
};
auto alphas = [](double q2, void* pdf) {
    return static_cast <LHAPDF::PDF*> (pdf)->alphasQ2(q2);
};

std::vector<double> dxsec(24);
pineappl_grid_convolute(grid, xfx, xfx, alphas, pdf, nullptr,
    nullptr, 1.0, 1.0, dxsec.data());

// print the results
for (std::size_t i = 0; i != 24; ++i) {
    std::printf("%.1f %.1f %.3e\n", bins[i], bins[i + 1], dxsec[i]);
}

// write the grid to disk
pineappl_grid_write(grid, "DY-LO-AA.pineappl");

// destroy the object
pineappl_grid_delete(grid);
\end{lstlisting}

\subsection{Demonstration of \texorpdfstring{\texttt{pineappl}}{pineappl}}
\label{app:pineappl-demo}

The program \texttt{pineappl} can be used to perform quick convolutions and other calculations with existing grids on the command line.
If started without any arguments, it prints its help and lists all supported subcommands:
\begin{lstlisting}[language=bash]
 $ pineappl
pineappl 0.1.0
Read, write, and query PineAPPL grids

USAGE:
    pineappl <SUBCOMMAND>

FLAGS:
    -h, --help       Prints help information
    -V, --version    Prints version information

SUBCOMMANDS:
    channels           Shows the contribution for each partonic channel
    convolute          Convolutes a PineAPPL grid with a PDF set
    help               Prints this message or the help of the given
                       subcommand(s)
    luminosity         Shows the luminosity function
    merge              Merges one or more PineAPPL grids together
    orders             Shows the predictions for all bins for each order
                       separately
    pdf_uncertainty    Calculates PDF uncertainties
\end{lstlisting}
As an example we show the grid produced for the ATLAS DY high-mass lepton-pair production (from section~\ref{subsec:numerical_results}).
The output shows all 13 bins with lower (\texttt{xmin}) and upper limit (\texttt{xmax}) of the invariant mass $M_{\ell \bar{\ell}}$ of the lepton pair, with the differential cross section $\mathrm{d} \sigma / \mathrm{d} M_{\ell \bar{\ell}}$ (\texttt{diff}), integrated cross section $(M_{\ell \bar{\ell}}^\mathrm{max} - M_{\ell \bar{\ell}}^\mathrm{min}) \mathrm{d} \sigma / \mathrm{d} M_{\ell \bar{\ell}}$ (\texttt{integ}), together with the perturbative uncertainty estimated from a 7-point scale variation (\texttt{neg unc} and \texttt{pos unc}).
\begin{lstlisting}[language=bash]
 $ pineappl convolute ATLASZHIGHMASS49FB.pineappl \
 > NNPDF31_nlo_as_0118_luxqed
bin xmin xmax     diff        integ     neg unc pos unc
---+----+----+------------+------------+-------+-------
  0  116  130 2.0630698e-1  2.8882978e0  -2.08%   1.69%
  1  130  150 9.1818985e-2  1.8363797e0  -1.79%   1.86%
  2  150  170 4.5306370e-2 9.0612740e-1  -1.60%   1.98%
  3  170  190 2.5894856e-2 5.1789712e-1  -1.66%   2.06%
  4  190  210 1.6075267e-2 3.2150534e-1  -1.70%   2.10%
  5  210  230 1.0526659e-2 2.1053317e-1  -1.71%   2.12%
  6  230  250 7.1928162e-3 1.4385632e-1  -1.71%   2.13%
  7  250  300 4.0776555e-3 2.0388277e-1  -1.70%   2.44%
  8  300  400 1.4775481e-3 1.4775481e-1  -1.94%   2.87%
  9  400  500 4.5473785e-4 4.5473785e-2  -2.30%   3.19%
 10  500  700 1.2164277e-4 2.4328555e-2  -2.41%   3.12%
 11  700 1000 1.9792340e-5 5.9377019e-3  -2.05%   2.12%
 12 1000 1500 2.0228761e-6 1.0114381e-3  -1.29%   0.47%
\end{lstlisting}

\subsection{Sample runcard for \texorpdfstring{\textsc{mg5\_aMC@NLO}}{mg5\_aMC@NLO}}
\label{app:sample-runcard}

The following run card was used to produce the results shown in section~\ref{subsec:numerical_results}.
The only difference to a normal \textsc{mg5\_aMC@NLO} run is the switch \texttt{set iappl 1}, which enables filling a \textsc{PineAPPL} grid.
For a complete set of runcards and patches see TODO.

\lstset{
  basicstyle=\ttfamily\small,
  showstringspaces=false,
  commentstyle=\ttfamily\small,
  keywordstyle=\ttfamily\small,
  identifierstyle=\color{black},
  stringstyle=\color{black},
}

\begin{lstlisting}
set complex_mass_scheme True
import model loop_qcd_qed_sm_Gmu
define p = p b b~
define j = p
generate p p > e+ e- [QCD QED]
output @OUTPUT@
launch @OUTPUT@
fixed_order = ON
set mz @MZ@
set ymt @YMT@
set ebeam1 3500
set ebeam2 3500
set pdlabel lhapdf
set lhaid 324900
set fixed_ren_scale True
set fixed_fac_scale True
set mur_ref_fixed @MZ@
set muf_ref_fixed @MZ@
set reweight_scale True
set ptl = 25.0
set etal = 2.5
set mll = 116
#user_defined_cut set mmllmax = 1500.0
set req_acc_FO 0.0001
set iappl 1
done
quit
\end{lstlisting}
