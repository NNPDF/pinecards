\documentclass[a4paper,11pt]{article}
\pdfoutput=1

\usepackage{jheppub}
\usepackage[T1]{fontenc}

\newcommand{\alphas}{\alpha_{S}}

\title{Including EW and mixed QCD-EW corrections into PDF fits}

\author[a,b]{S. Carrazza,}
\author[b]{E. Nocera,}
\author[a]{C. Schwan}
\author[a,b]{and M. Zaro}

\affiliation[a]{Tif Lab, Dipartimento di Fisica, Universit\`a di Milano and INFN, Sezione di Milano, 20133 Milano, Italy}
\affiliation[b]{Nikhef Theory Group, Science Park 105, 1098 XG Amsterdam, The Netherlands}

\emailAdd{stefano.carrazza@mi.infn.it}
\emailAdd{e.nocera@nikhef.nl}
\emailAdd{christopher.schwan@mi.infn.it}
\emailAdd{marco.zaro@mi.infn.it}

\abstract{}

\begin{document}

\maketitle
\flushbottom

\section{Introduction}
\label{sec:introduction}

ERN

The impact of EW corrections to LHC data.\\
How to do that consistently, include systematically EW corrections, blabla.\\
Outline a possible strategy to deal with EW correction in LHC observables (FSR, PHOTOS, dressed/born, ...).\\
Pinappl kills amcfast and mcgrid, replace applgrid and fastnlo.\\

\section{Technical aspects}
\label{sec:technical-aspects}

\subsection{PineAPPL}
\label{sec:pineappl}

In this paper we introduce a new program called \textsc{PineAPPL}, which solves the same problem as \textsc{APPLgrid}~\cite{} and \textsc{fastNLO}~\cite{}, but also supports EW corrections, which are the main interest here.
In particular, the following features distinguish it from the two aforementioned programs:
\begin{itemize}
\item support for arbitrary fixed-order calculations in $\alpha$ and $\alpha_\mathrm{s}$ with
\item support for scale variations in multiples of the central renormalization and factorization scale,
\item support for all-order predictions from resummation and photon-/parton-showers,
\item a simple \textsc{C}-interface, with a wrapper for \textsc{Fortran}
\item and support for it implemented in \textsc{mg5\_aMC@NLO},
\item converters to convert \textsc{APPLgrids} and \textsc{fastNLO} tables.
\end{itemize}

\subsection{Grid representation}
\label{sec:grid-representation}

\subsection{Accuracy and performance}
\label{sec:accuracy-and-performance}

CS, MZ

How to integrate in MC codes, how this works in aMCblast (setup, MG5 version, urls).\\
Appendix: how to create and fill grids, + simple code.


\section{Results}

SC

For datasets in 3.1qed (+DY2), compute pure NLO QCD (born) vs NLO QCD + NLO EW (dressed) FKtables and:\\
- compute data/prediction plots\\
- compute chi2\\

\section{Double counting}
Count von Count, ALL

Explain the problem to experimentalists (the bee and the flower).\\

As it has been mentioned in the introduction, the ability to consistently include electro-weak correction in
PDF fits is not enough to make a fully-fledged fit possible in a consistent manner. Indeed, experimental data should
be provided in  a format that allows PDF collaborations to employ them in such fits in a theoretically sound manner. In this
section, we will try to provide some guidelines, by highlighting some sources of inconsistency commonly occurring in experimental 
analyses. These will either lead to accounting multiple times for EW effects (so-called double counting), or to not accounting at
all for such effects.\footnote{By no means we intend to blame on our beloved experimental colleagues for what they write or do. However,
    we think that by highlighting errors and making everybody more cautious about them is the only way to avoid repeating such
    errors in the future.}

A common source of inconsistency is the subtraction of backgrounds processes which must not be considered as such. The typical example
is neutral-current Drell-Yan, where the signal process is an opposite-sign lepton pair, which starts
at $\mathcal O(\alpha^2)$. Because this process is usually thought
as a quark-initiated, $s$-channel mechanism ($q\bar q \to \gamma^*/Z \to \ell^+ \ell^-$), in many analyses the photon-induced component,
$\gamma \gamma \to \ell^+ \ell^-$ in the $t$ channel, is considered as a different process, and it 
subtracted (see e.g. Refs.~\cite{}). However, such a distinction cannot be physically
justified beyond LO. Indeed, at $\mathcal O(\alpha^3)$, the reaction $q \gamma \to \ell^+ \ell^- q$ becomes possible, which
includes both kind of topologies. {\bf MZ add Feynman diags} The problem here derives from the fact 
that identifing a process based only on its 
initial-state partons is plain wrong in quantum mechanics. While this fact is well established in QCD --nobody would ever 
consider to ``subtract'' the gluon-initiated contribution to top-pair production in top analyses-- seemingly it is not so
when EW corrections are considered.

A second example is related to removing EW effects from data. These can be either the full EW corrections (Refs??) 
or just a part, for example deconvolving multiple-photon radiation from light particles in the final state. This applies mostly
to processes such as neutral- or charge-current Drell-Yan, specially when electrons are considered. The problem lie in the fact that
 electrons, and to a lesser extent muons, tend to radiate photons, and such photons are typically not accounted
for in hard matrix-elements. Thus, leptons that are measured in the detector are less energetic, and this fact is compensated for
by undoing the photon shower before publishing data, which are referred to e.g. as \emph{Born-level electrons}~\cite{}. There are 
at least two 
problems related with this. The first, a quite obvious one, is that when EW corrections
are included at fixed order, the first photon emission is included exactly at the matrix-element level. The second, which
applies for electrons, is that they are never measured as bare particles, because of the finite resolution 
of the electromagnetic calorimeter. Since collinear photonic emission cannot be resolved, at least data 
involving electrons should also be published
in terms of dressed particles, after applying some recombination scheme. This has the further advantage of being 
inclusive on the effect of further collinear emissions. For what concerns muons, while in principle
the concept of bare muons is a physical one, it should be kept in mind that modern, general-purpose codes employed to
compute EW corrections treat leptons as massless. This fact encourages to explore the possibility of employing dressed
muons, on the same footing as their electron counterpart.

\section{Outlook}

ALL

\cite{Carli:2010rw}
\cite{Bertone:2014zva}

\appendix

\acknowledgments

C.S. is supported by the European Research Council under the European Unions Horizon 2020 research and innovation Programme (grant agreement no. 740006).

\bibliographystyle{JHEP}
\bibliography{paper}

\end{document}
