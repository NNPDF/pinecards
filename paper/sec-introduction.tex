\section{Introduction}
\label{sec:introduction}

With the recent completion of Run II, the Large Hadron Collider (LHC) has 
accumulated data from an integrated luminosity of approximately 
\SI{150}{\per\femto\barn}~\cite{Mangano:2020icy}. This represents only a small fraction of
the anticipated \SI{3000}{\per\femto\barn} that will eventually be recorded in the
forthcoming twenty years of LHC operation. Nevertheless the statistical 
uncertainty of the data has already shrunk to unprecedentedly small values,
typically \SI{1}{\percent} or less, a fact that will allow for precision tests of the
Standard Model (SM) and for indirect searches of New Physics only if 
theoretical predictions become comparatively precise. This entails the 
computation of additional higher-order contributions to the fixed-order 
perturbative expansion, on the one hand, and an increasingly
sophisticated determination of the Parton Distribution Functions (PDFs) of the 
proton~\cite{Gao:2017yyd,Ethier:2020way}, on the other hand. 

In the first respect, because Quantum Chromodynamics (QCD) dominates the 
interactions occurring within colliding protons at the LHC, much effort 
has been devoted to the computation of higher-order QCD corrections: 
fully-differential next-to-leading order (NLO) results, possibly matched to a
parton shower, are currently automated in various general-purpose
Monte Carlo generators~\cite{Gleisberg:2008ta,Alwall:2014hca,Bellm:2015jjp}
(see also ref.~\cite{Buckley:2011ms} for a review),
while an increasing number of next-to-next-to-leading order (NNLO) predictions 
are becoming available for processes with various degrees of inclusiveness
(see e.g.\ ref.~\cite{Amoroso:2020lgh} and references therein). The computation
of higher-order corrections in the electroweak (EW) and combined QCD+EW theory has
also witnessed a comprehensive progress. Frameworks 
were developed in which the QCD and EW couplings are simultaneously treated as 
small parameters in the perturbative expansion, and the computation of 
theoretical predictions, accurate to NLO in both (including multi-coupling
QCD--EW terms), is automated~\cite{Kallweit:2014xda,Biedermann:2017yoi,Frederix:2018nkq}. For an extensive and recent review, see ref.~\cite{Denner:2019vbn}.

In the second respect, contemporary PDF
sets~\cite{Harland-Lang:2014zoa,Ball:2017nwa,Hou:2019efy}
incorporate a significant amount of LHC data, which is analysed with NNLO QCD 
theory by default. No EW corrections are systematically included in the 
theoretical description of the experimental observables to which PDFs are 
optimised, except for QED effects if a photon PDF is 
determined~\cite{Schmidt:2015zda,Manohar:2016nzj,Manohar:2017eqh,Bertone:2017bme,Harland-Lang:2019pla}.
The resulting relative PDF uncertainty --- which accounts only for the
uncertainty of the data and of residual methodological inefficiencies inherent to 
each PDF determination --- can be as low as \SI{1}{\percent} at the EW
scale~\cite{Ball:2017nwa}. Theoretical uncertainties, possibly of comparable 
size (e.g.\ from missing higher-order terms in the QCD perturbative
expansion), have started to be represented 
into PDF uncertainties only very 
recently~\cite{AbdulKhalek:2019bux,AbdulKhalek:2019ihb}.

The two respects are intertwined, as they both concur to determine the accuracy
of the theoretical predictions that are matched to the precision of the data.
In particular, taking advantage of the automation pioneered
in refs.~\cite{Kallweit:2014xda,Biedermann:2017yoi,Frederix:2018nkq},
perturbative corrections that arise from the simultaneous expansion in both the
QCD and EW couplings should start to be
incorporated in calculations for LHC processes as standard. The reason is
twofold. First, one expects NNLO QCD and NLO EW corrections to be of
comparable size because, at the EW scale, the QCD and EW running couplings 
become similar, $\alphas^2\sim \alpha$. If NNLO QCD corrections are included
by default in the computations, NLO EW corrections should be
taken into account as well. Second, the virtual exchange of soft or 
collinear weak bosons leads to Sudakov 
logarithms~\cite{Denner:2000jv,Denner:2001gw}
(see also ref.~\cite{Denner:2019vbn} and references therein),
which can make the coefficients of the EW series grow faster than 
their QCD counterparts. This behaviour is relevant in
phase-space regions associated with large mass scales (roughly of the order
of a \si{\tera\electronvolt}), where several LHC data sets (e.g.\ Z-boson
transverse-momentum distributions) enter both the validation of the SM and the search
for new physics.

The consistent inclusion of QCD and EW corrections in precision computations for
LHC processes entails
the solution of two separate problems. First, a problem of efficiency:
fast-interpolation grids should be constructed, whereby partonic matrix 
elements, accurate to NLO QCD+EW, are precomputed in such a way that the 
numerical convolution with generic input PDFs can be efficiently approximated
by means of interpolation techniques. Such grids are essential
whenever the evaluation of the hadronic cross section needs to 
be performed a large number of times, as is the case in the evaluation of
scale variations or of PDF fits. While two formats already exist for
these grids, \textsc{APPLgrid}~\cite{Carli:2010rw} and
\textsc{fastNLO}~\cite{Kluge:2006xs,Wobisch:2011ij,Britzger:2012bs}, none of them supports the inclusion of EW
corrections nor the interface to a Monte Carlo generator accurate to 
NLO QCD+EW\@. Second, a problem of consistency: the way in which EW effects may
(or may not) be folded into the data varies across different experimental 
analyses, a fact that challenges their consistent theoretical interpretation. 
Examples are the subtraction of background processes which should not be 
considered as such (e.g.\ the $t$-channel photon-induced component in
neutral-current Drell--Yan, which is not a separate process beyond leading
order) or of just a part of the EW effects (e.g.\ multiple-photon
radiation from light particles in the final state of neutral- or charged-current
Drell--Yan, especially with electrons). Be that as it may, if EW effects are
systematically included in theoretical predictions, they should not be subtracted
from experimental results, otherwise they will be double counted.

In this paper we address the first of these two problems. Specifically, 
we develop \textsc{PineAPPL}, a library that allows any user to generate
fast-interpolation grids, accurate to any fixed order in the QCD and
EW couplings. The library supports variations of the factorisation
and renormalisation scales, and can be extended to include
resummation, and matching with a photon- and/or parton-shower.
The grids in the new \textsc{PineAPPL} format complement those
(accurate to fixed order in the strong coupling only) that can be generated
in the \textsc{APPLgrid} and \textsc{FastNLO} formats. The \textsc{PineAPPL}
library is interfaced to \textsc{MadGraph5\_aMC@NLO} (\textsc{mg5\_aMC}
henceforth), with which it has been developed and tested. In this respect,
\textsc{PineAPPL} supersedes
\textsc{APPLgrid}+\textsc{aMCfast}~\cite{Bertone:2014zva}.
However, \textsc{PineAPPL} can also be easily interfaced to any Monte Carlo generator,
e.g.\ \textsc{SHERPA}~\cite{Biedermann:2017yoi}, where it would complement
\textsc{MCgrid}~\cite{DelDebbio:2013kxa,Bothmann:2015dba}.

The paper is organised as follows. In section~\ref{sec:pineappl} we introduce
\textsc{PineAPPL}, we describe its features, we illustrate its operation, and we
assess its performance. In section~\ref{sec:results} we validate
\textsc{PineAPPL} and demonstrate its capabilities by computing fast-interpolation grids, accurate to NLO QCD and NLO QCD+EW, for a representative
set of LHC processes for which EW corrections may
have a sizeable effect on the accuracy of the theoretical predictions.
In section~\ref{sec:doublecounting} we try to detail in a more
comprehensive manner the double-counting problem sketched above, the solution of which,
however, remains beyond the scope of the current work. We provide our
conclusions and an outlook in section~\ref{sec:conclusion}. Examples of
usage and the installation of \textsc{PineAPPL} are provided in
appendix~\ref{app:pineappl}; appendix~\ref{app:lumis} collects the parton
luminosities for each process considered in section~\ref{sec:results}; and
appendix~\ref{app:add_plots} complements some of the results presented
in section~\ref{sec:results}.

