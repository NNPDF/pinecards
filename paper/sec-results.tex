\section{Results}
\label{sec:results}

In this section we demonstrate the capabilities of \textsc{PineAPPL} by
computing fast interpolation grids, accurate to NLO QCD and NLO QCD+EW,
for a representative set of processes in which EW corrections are expected 
to be sizeable. In order to consider some realistic kinematics for these
processes, we resort to measurements commonly devised for inclusion in PDF
fits. Our aim is twofold. On the one hand, we want to validate the results
obtained with \textsc{PineAPPL}; on the other hand, we want to assess the
impact of the EW corrections for usual experimental setups. We describe first
the processes and measurements that we consider, then the computational
settings that we adopt, and finally the results that we obtain.

\subsection{Processes and measurements}
\label{subsec:processes_and_measurements}

We focus on the following three processes: DY lepton-pair production, top-quark
pair production, and Z-boson (lepton-pair) production with non-zero transverse
momentum at the LHC. For each of these processes, we consider the measurments
described below.

\paragraph{DY lepton pair production.}
We select the distribution, single-differential in the invariant mass of the
lepton pair, $M_{\ell \bar\ell}$, measured by the ATLAS experiment at a centre-of
mass energy of 7~TeV in the high-mass region
($M_{\ell\bar\ell}>116$~GeV)~\cite{Aad:2013iua}.
We also select the distribution, double-differential in the rapidity and in
the invariant mass of the lepton pair, $y_{\ell\bar\ell}$ and $M_{\ell\bar\ell}$,
measured by the CMS experiment at 7~TeV~\cite{Chatrchyan:2013tia}.
These measurements are currently included as standard in the
NNPDF3.1~\cite{Ball:2017nwa} and MMHT2014~\cite{Harland-Lang:2014zoa} PDF sets,
although with appropriate kinematic cuts that remove the bins at the largest
values of invariant mass, where EW corrections become sizeable.

\paragraph{Top-quark pair production.}
We select the distributions, single-differential in either the transverse
momentum of the top quark, $p_T^t$, or the invariant mass of the top-quark
pair, $m_{t\bar t}$, measured by the ATLAS and CMS experiments at a centre-of-mass
energy of 8~TeV~\cite{Aad:2015mbv,Khachatryan:2015oqa}. These measurements have
been extensively studied in the context of PDF fits in
Refs.~\cite{Czakon:2016olj,Bailey:2019yze,Amoroso:2020lgh,Kadir:2020yml}.
Because EW corrections are significantly smaller for distributions differential
in the rapidity of either the top quark or the top-quark
pair~\cite{Czakon:2017wor}, these were preferred for inclusion in the NNPDF3.1
set~\cite{Ball:2017nwa}.

\paragraph{$Z$-boson production with non-zero transverse momentum.}
We select the distribution, single-differential in the transverse momentum of
the $Z$ boson, $p_T^Z$, measured by the CMS experiment at a centre-of-mass
energy of 13~TeV~\cite{Sirunyan:2019bzr}. This measurement has not been
included in a PDF determination yet. Given that it has sub-percent
uncertainties, it is expected that EW corrections will be essential to
achieve a good description of it, and to constrain accurately the PDFs.
A similar challenge was observed in similar measurements, from the
ATLAS~\cite{Aad:2015auj} and CMS~\cite{Khachatryan:2015oaa} experiment at a
centre-of-mass energy of 8~TeV, in Ref.~\cite{Boughezal:2017nla}, which lead to
their partial inclusion (upon the selection of an appropriate kinematic cut)
in the NNDPF3.1 PDF set~\cite{Ball:2017nwa} and in variants of the CT18 PDF
set~\cite{Hou:2019efy}.

\subsection{Computational settings}
\label{subsec:computational_settings}



We discuss,
\begin{itemize}
\item in section~\ref{sec:atlas-high-mass-dy}, ATLAS high-mass DY lepton-pair production at \SI{7}{\tera\electronvolt} \cite{Aad:2013iua}, measuring $\mathrm{d} \sigma / \mathrm{d} M_{\ell \bar{\ell}}$ for the lepton-pair invariant mass $M_{\ell \bar{\ell}} > \SI{116}{\giga\electronvolt}$,
\item in section~\ref{sec:cms-dy}, CMS DY lepton-pair production at \SI{7}{\tera\electronvolt} \cite{Chatrchyan:2013tia}, measuring $\mathrm{d} \sigma / \mathrm{d} y_{\ell \bar{\ell}}$ for six slices in the range $\SI{20}{\giga\electronvolt} < M_{\ell \bar{\ell}} < \SI{1500}{\giga\electronvolt}$,
\item in section~\ref{sec:atlas-top-pair-production}, ATLAS top-pair production at \SI{8}{\tera\electronvolt} \cite{Aad:2015mbv}, measuring
\begin{itemize}
\item the transverse momentum of the reconstructed top, $\mathrm{d} \sigma / \mathrm{d} p_\mathrm{T}^\mathrm{t}$,
\item its rapidity, $\mathrm{d} \sigma / \mathrm{d} y_\mathrm{t}$,
\item the invariant mass of the reconstructed top pair, $\mathrm{d} \sigma / \mathrm{d} M_{\mathrm{t} \bar{\mathrm{t}}}$, and
\item its rapidity, $\mathrm{d} \sigma / \mathrm{d} M_{\mathrm{t} \bar{\mathrm{t}}}$,
\end{itemize}
\item in section~\ref{sec:cms-transverse-momentum}, CMS transverse momentum of the Z boson at \SI{13}{\tera\electronvolt} \cite{Sirunyan:2019bzr}, measuring $\mathrm{d} \sigma / \mathrm{d} p_\mathrm{Z}$ for $p_\mathrm{Z} > \SI{20}{\giga\electronvolt}$.
\end{itemize}
We used the MC \texttt{mg5\_aMC@NLO}~\cite{} and the PDF set \texttt{NNPDF31\_as\_0118\_luxqed}~\cite{} to generate the predictions.
The chosen PDF set contains a photon PDF calculated from the LUXQED method~\cite{}.
For each process---except top-pair production, which has stable tops in the final state---we use a complex-mass scheme~\cite{} as described in ref.~\cite{}.
The values of the most important parameters are
\begin{equation}
\begin{aligned}
M_\mathrm{W} &= \SI{80}{\giga\electronvolt} \text{,} \quad &
M_\mathrm{Z} &= \SI{90}{\giga\electronvolt} \text{,} \quad &
m_\mathrm{t} &= \SI{170}{\giga\electronvolt} \text{,} \\
\Gamma_\mathrm{W} &= \SI{2}{\giga\electronvolt} \text{,} &
\Gamma_\mathrm{Z} &= \SI{2}{\giga\electronvolt} \text{,} &
G_\mu &= \text{.}
\end{aligned}
\end{equation}

\noindent
TODO for each of the following subsections:
\begin{itemize}
\item size of the photon-initiated contributions,
\item largest partonic channel,
\item most important $x$ region,
\item PDF uncertainty,
\end{itemize}

\subsection{ATLAS high-mass DY lepton-pair production at \SI{7}{\tera\electronvolt}}
\label{sec:atlas-high-mass-dy}

\begin{equation}
\begin{gathered}
p_\mathrm{T}^\ell > \SI{25}{\giga\electronvolt} \text{,} \quad |\eta_\ell| < 2.5 \text{,} \\
\SI{116}{\giga\electronvolt} < M_{\ell \bar{\ell}} < \SI{1500}{\giga\electronvolt} \text{,}
\end{gathered}
\end{equation}

\subsection{CMS DY lepton-pair production at \SI{7}{\tera\electronvolt}}
\label{sec:cms-dy}

\begin{equation}
\begin{gathered}
p_\mathrm{T}^{\ell_1} > \SI{14}{\giga\electronvolt} \text{,} \quad p_\mathrm{T}^{\ell_2} > \SI{9}{\giga\electronvolt} \text{,} \quad |\eta_\ell| < 2.4 \text{,} \\
|\eta_{\ell \bar{\ell}}| < 2.4 \text{,} \quad \SI{20}{\giga\electronvolt} < M_{\ell \bar{\ell}} < \SI{1500}{\giga\electronvolt} \text{,}
\end{gathered}
\end{equation}

\subsection{ATLAS top-pair production at \SI{8}{\tera\electronvolt}}
\label{sec:atlas-top-pair-production}

\subsection{CMS transverse momentum of the Z boson at \SI{13}{\tera\electronvolt}}
\label{sec:cms-transverse-momentum}

\begin{equation}
\begin{gathered}
p_\mathrm{T}^\ell > \SI{25}{\giga\electronvolt} \text{,} \quad |\eta_\ell| < 2.4 \text{,} \quad M_\mathrm{Z} - \SI{15}{\giga\electronvolt} < M_{\ell \bar{\ell}} < M_\mathrm{Z} + \SI{20}{\giga\electronvolt} \text{,} \\
|\eta_{\ell \bar{\ell}}| < 2.4 \text{,} \quad \SI{20}{\giga\electronvolt} < p_\mathrm{T}^{\ell \bar{\ell}} < \SI{1500}{\giga\electronvolt} \text{,}
\end{gathered}
\end{equation}

\begin{figure}
    \centering
    \includegraphics[width=0.5\textwidth]{figures/pineappl_ATLASZHIGHMASS49FB}
    \caption{PineAPPL comparison for ATLAS high-mass Drell--Yan at $\sqrt{s}=7$ TeV.}
    \label{fig:atlaszhighmass49fb}
\end{figure}

\begin{figure}
    \centering
    \includegraphics[width=0.5\textwidth]{figures/pineappl_CMSDY2D11_bin1}%
    \includegraphics[width=0.5\textwidth]{figures/pineappl_CMSDY2D11_bin2}
    \caption{PineAPPL comparison for CMS 2D Drell--Yan.}
    \label{fig:cmsdy2d11_bins12}
\end{figure}

\begin{figure}
    \centering
    \includegraphics[width=0.5\textwidth]{figures/pineappl_CMSDY2D11_bin3}%
    \includegraphics[width=0.5\textwidth]{figures/pineappl_CMSDY2D11_bin4}
    \caption{PineAPPL comparison for CMS 2D Drell--Yan.}
    \label{fig:cmsdy2d11_bins34}
\end{figure}


\begin{figure}
    \centering
    \includegraphics[width=0.5\textwidth]{figures/pineappl_CMSDY2D11_bin5}%
    \includegraphics[width=0.5\textwidth]{figures/pineappl_CMSDY2D11_bin6}
    \caption{PineAPPL comparison for CMS 2D Drell--Yan.}
    \label{fig:cmsdy2d11_bins56}
\end{figure}


\begin{figure}
    \centering
    \includegraphics[width=0.5\textwidth]{figures/pineappl_ATLAS_TTB_DIFF_8TEV_LJ_TPT}%
    \includegraphics[width=0.5\textwidth]{figures/pineappl_ATLAS_TTB_DIFF_8TEV_LJ_TTM}
    \caption{PineAPPL comparison for ATLAS top pair.}
    \label{fig:cmsdy2d11_bins56}
\end{figure}

\begin{figure}
    \centering
    \includegraphics[width=0.5\textwidth]{figures/pineappl_CMS_Z_13_TEV}
    \caption{PineAPPL comparison for CMS $Z$ $p_T$ distribution.}
    \label{fig:cmsdy2d11_bins56}
\end{figure}

%ERN 7 Apr: there is consensus on the fact that we should present results for the
%following data sets:
%\begin{itemize}
%\item ATLAS high mass DY distributions, 7 TeV~\cite{Aad:2013iua} (CS);
%\item CMS 2D DY distributions, 7 TeV~\cite{Chatrchyan:2013tia} (CS);
%\item ATLAS top pair differential distributions ($m_{t\bar{t}}$ and $p_T^t$),
%8 TeV~\cite{Aad:2015mbv} (ERN);
%\item CMS $Z$ $pT$ distributions, 13 TeV~\cite{Sirunyan:2019bzr} (ERN).
%\end{itemize}
%
%We agree not to display any LHCb measurement, given that they won't add
%further value to our discussion.
%Note added: we might also want to have a look at the ATLAS 2D and 3D DY
%distributions, 8 TeV~\cite{Aad:2016zzw,Aaboud:2017ffb}, if time allows
%us to do so.
%
%CS 25 Jun: We've agreed to show basically two types of plots: 1) technical plots showing the good agreement between the MC compared to the results from the grids, and 2) phenomenological results showing larger EW corrections, for example.
%In the aMCfast paper both is shown in single plot, but since we have more to show, I suggest the following: for the technical plots we show a 2x2 matrix of plots, each showing the difference of the grid result compared to the MC result, in the following fashion:
%\begin{itemize}
%\item NLO QCD with low statistics,
%\item NLO QCD+EW with low statistics,
%\item NLO QCD with high statistics, and finally
%\item NLO QCD+EW with high statistics,
%\end{itemize}
%each showing a few scale variations.
%These plots then clearly show that no matter what corrections you choose, no matter the statistics, and no matter the scale variation, the agreement is always excellent.
%In any case I would like to avoid showing a plot with absolute numbers and low statistics, which looks a bit ridiculous in my opinion (look at figure 1, left side, top plot of the aMCfast paper).
%
%Finally we can show another series of plots, which in my opinion should be very similar to the usual pheno paper plots: absolute numbers with a scale variation band, maybe a few corrections shown in the same plot and then in the bottom relative corrections.
