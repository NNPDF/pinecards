\section{On the consistency of experimental data}
\label{sec:doublecounting}

The ability to include theoretical calculations of EW corrections in
PDF determinations, as has been mentioned in the introduction, is not enough to make a \emph{consistent} fit possible. Indeed, experimental data should
be provided in a format that allows PDF collaborations to easily employ them in these fits in a theoretically sound manner. Before concluding the paper,
we will provide some guidelines to facilitate a consistent presentation of experimental data to that end. We focus on the problem of subtracting (partial) EW effects in the data, which, in PDF fits with theory predictions including them, leads to multiple accounting of EW effects (so-called double counting).

\begin{figure}[ht!]
    \centering
    \begin{overpic}[width=0.6\textwidth, trim=0.cm 11cm 0.cm 10cm, clip=True]{figures/dy-pi.pdf}
        \put (5, 33) {\large $q$}
        \put (20, 37) {\large $q$}
        \put (38, 5) {\large $\ell^+$}
        \put (38, 31) {\large $\ell^-$}
        \put (11, 6) {\large $\gamma$}
        %
        \put (65, 9) {\large $q$}
        \put (73, 30) {\large $q$}
        \put (96, 25) {\large $\ell^-$}
        \put (96, 8) {\large $\ell^+$}
        \put (55, 26) {\large $\gamma$}
    \end{overpic}
    \caption{\label{fig:dy-pi}
    Photon-induced (left) and quark-induced (right) contributions to the Drell-Yan process. In black, the LO process is shown.
    In red, the initial-state splitting leading to the real-emission $q \gamma \to \ell^+ \ell^- q$ is highlighted. Such a 
    real emission enters in the NLO EW corrections.}
\end{figure}
An example is the subtraction of (irreducible) background processes which must not be considered as such. A very blatant case
is neutral-current Drell--Yan, where the signal process is the production of an opposite-sign lepton pair, which starts
at $\mathcal O(\alpha^2)$. Because this process is sometimes thought
as a quark-initiated $s$-channel mechanism ($q\bar q \to \gamma^*/Z \to \ell^+ \ell^-$), in many analyses the photon-induced component,
$\gamma \gamma \to \ell^+ \ell^-$ in the $t$ channel, is considered to be a different process, and therefore as a background and subtracted.
The subtraction from the measured data is done by calculating the theoretical predictions of the double-photon initiated contribution, possibly including (unphysical) higher-order
corrections. For example, in refs.~\cite{Aaboud:2017ffb,Aad:2016zzw} (a similar statement appears also in an older analysis~\cite{Aad:2013iua}), one reads:
\begin{quote}
The photon-induced process, $\gamma\gamma \to \ell \ell$, is simulated at LO using Pythia 8 
and the MRST2004qed PDF set~\cite{Martin:2004dh}. The expected yield for this process also accounts for 
NLO QED/EW corrections from references~\cite{Bardin:2012jk,Bondarenko:2013nu}, which decrease the yield by approximately 30\%.
\end{quote}
Such a distinction, which is unphysical and incorrect in quantum mechanics, may be somehow justified at LO. Beyond this order, it is simply wrong.
Indeed, at $\mathcal O(\alpha^3)$, the reaction $q \gamma \to \ell^+ \ell^- q$ becomes possible, which
includes both kind of topologies discussed above (and needs both in order to yield an IR-finite result), see Fig.~\ref{fig:dy-pi}. While it is common sense that the QCD counterpart of this subtraction should never 
be performed---no one would 
consider to \enquote{subtract} the gluon-initiated contribution to top-pair production in top analyses---seemingly it is not so
when EW corrections are considered.

%%{\bf It looks like CMS does things in a pretty consistent manner~\cite{Sirunyan:2018owv, CMS:2014jea}. Shall we mention? We should
%%not make ATLAS appear as the bad guys, and CMS as the good ones.} 

A second example is related to removing EW effects from data. These can be either the full EW corrections
or just a part. In either case, a comparison between these data and a NLO-EW accurate simulation aimed at the extraction of some parameter would be meaningless, as some effects included in the latter
have been removed from the former. The typical example relevant for LHC is the deconvolution of effects due to multiple-photon radiation
from light particles in the final state. This applies mostly
to processes such as neutral- or charge-current Drell--Yan, specially when electrons are considered. The problem lies in the fact that
 electrons, and to a lesser extent muons, tend to radiate photons, and such photons are typically not accounted
for in QCD matrix-elements. Thus, leptons that are measured in the detector are less energetic, and this fact is compensated for
by inverting the photon shower. The resulting dataset is e.g.\ referred to as \emph{pre-FSR} with observables of \emph{Born-level electrons} (see e.g.\ Ref.~\cite{Aad:2015auj} for
its definition).
These datasets are needed for and correctly used in QCD-only PDF determinations, since the EW corrections to some DY observables can be significant, and excluding them would therefore degrade the fit.
In fits including fixed-order EW corrections the problem with this definition is that the first photon emission is included exactly at the matrix-element level. The inclusion of
subsequent emission would require the matching with the QED shower, which is not yet available for general processes. It is interesting
to note that one can tune the QED parton shower in order to mimic NLO EW effects, so that a prediction only accurate at NLO
QCD displays a remarkable agreement with another at NLO QCD+EW when the photonic shower is included (see example the behaviour of predictions showered with
\textsc{Photos}~\cite{Barberio:1990ms,Barberio:1993qi,Golonka:2005pn}in Ref.~\cite{CarloniCalame:2016ouw}). However, this kind of agreement
always comes \emph{a posteriori}, and cannot be ensured in general. A more physical definition would be either the one of bare leptons (post FSR; \textbf{CS}: post FSR is for recombined leptons, I haven't seen a dataset for bare leptons, but they should exist)
or of recombined ones. The problem with the former is that 
electrons are never measured as bare particles, because of the finite resolution 
of the electromagnetic calorimeter. For what concerns muons, while in principle the concept of a bare muon is physical (\textbf{CS}: cite papers for non-collinear safe observables/subtraction?), it should be kept in
mind that modern, general-purpose codes employed to
compute EW corrections treat leptons as massless, to ensure numerical stability of the matrix elements. In this case, using bare leptons is not collinear safe. Recombined
leptons avoid all these shortcomings, with the further advantage of being 
inclusive on the effect of extra collinear emissions. This fact encourages to explore the possibility of employing a dressed-lepton
definition, regardless of the leptonic flavour. It is reassuring to acknowledge that this practice is already being followed in experimental analyses:
indeed, to mention two examples of analyses discussed in this paper, in Ref.~\cite{Aad:2015auj} data for dressed leptons are published, together with the Born-level
and bare ones (\textbf{CS}: wrong?), while Refs.~\cite{Sirunyan:2019bzr} employs a dressed-lepton definition.

{\bf this section may be troublesome with part of our community, e.g. paper of Tackmann suggests to use born-level electrons https://arxiv.org/pdf/2006.11382.pdf}
