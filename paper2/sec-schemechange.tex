\section{Scheme-change operator}
\label{sec:scheme-change-operator}

In this section, we elaborate on the consistency between the coupling renormalisation in the PDFs (entering the evolution equation) and in 
the partonic matrix-element. {\bf MZ add citations}
At LO accuracy in $\alpha$, the two couplings can be chosen independently, as any ambiguity in the definition of $\alpha$  is higher-order.
While, in general, $\alpha$ is defined in the $\overline{\textrm{MS}}$ scheme in the PDF, the choice for the value (and renormalisation conditions)
in the partonic matrix-element depends on the process at hand. For example, for most LHC processes, 
one may want to use a scheme such as the $G_\mu$ scheme, which includes all-order weak effects. An exception is the case where tagged final-state
photons are present, where one should use the Thomson value $\alpha(0)$ for the vertices where the photons are present. In the counterpart
of this last example where the photons are in the initial state, i.e. they are part of the partonic content of the proton, a sensible choice 
seems the one of employing $\overline{\textrm{MS}}$ scheme, at least for the vertices where the photons eneter, since they come 
directly from the parton evolution.\\

At NLO in $\alpha$, this freedom seems not to be there any longer, at least conceptually: in fixed-order calculations a collinear counter term is present,
which subtracts initial-state collinear singularities and implements the so-called PDF redefinition. This counter-term is
essentially the $\mathcal O(\alpha)$ expansion, or at least it should be, since the two definitions of $\alpha$ may differ. However, any
dfference in the two definitions is $\mathcal O(\alpha^2)$, thus beyond NLO accuracy. This is why the usage of e.g. the $G_\mu$ scheme
is still possible (and in general recommended) when NLO EW corrections are computed. One notable exception
is the case when initial-state photons are present. As argued above, the corresponding coupling should be renormalised in the $\overline{\textrm{MS}}$ scheme.
If one fails to do so, as it is the case for all (to our knowledge) general-purpose programs which compute NLO EW corrections for a generic process, in
principle a mismatch is introduced at NLO. In practice, given that \emph{numerically} photon-initiated processes are of the same order of EW corrections,
one can argue that such a mismatch turns to be subleading for all practical purposes. Still, it is worth to provide some formulas, which can help to gauge the impact of the missing terms.
Let $n_I$ be the number of initial-state photons, and $b$ the power of $\alpha$ entering (a given contribution to) the Born squared matrix-element $\mathcal B$. If
one employs the $G_\mu$ scheme everywhere, then
\begin{equation}
    \mathcal B_{G_\mu}= \alpha_{G_\mu}^b B,
\end{equation}
where $B$ does not contain any power of $\alpha$. If instead the $\overline{\textrm{MS}}$ scheme is employed for initial-state photons, then one has
\begin{equation}
    \mathcal B_{\overline{\textrm{MS}}}= \alpha_{G_\mu}^{b-n_I} \hat \alpha^{n_I} B,
\end{equation}
where we have denoted $\alpha$ in the $\overline{\textrm{MS}}$ scheme as $\hat\alpha$, and we have omitted its dependence on the scale.
Thus, the mismatch between the two cases is
\begin{equation}
    \mathcal B_{G_\mu} - \mathcal B_{\overline{\textrm{MS}}}  = \left[ \alpha_{G_\mu}^b - \alpha_{G_\mu}^{b-n_I} \hat \alpha^{n_I} \right] B =
    n_I \frac{\Delta \alpha_{G_\mu-\overline{\textrm{MS}}}}{\alpha_{G_\mu}} \alpha_{G_\mu}^b B  + \mathcal O (\alpha^2),
\end{equation}
where
\begin{equation}
\frac{\Delta \alpha_{G_\mu-\overline{\textrm{MS}}}}{\alpha_{G_\mu}}(\mu_\mathrm{R}^2) = 
\left(\Delta r^{(1)} - \Delta \hat{\alpha} (\mu_\mathrm{R}^2) + \ldots \right) \text{,} \label{eq:consistency-condition}
\end{equation}
which uses the following definitions,
\begin{equation}
\hat{\alpha} (\mu_\mathrm{R}^2) = \frac{\alpha(0)}{1 - \Delta \hat{\alpha} (\mu_\mathrm{R}^2)} \text{,} \qquad \alpha_{G_\mu} = \alpha(0) \left( 1 + \Delta r^{(1)} \right) + \mathcal{O} (\alpha(0)^3) \text{,}
\end{equation}
which in make use of the fine-structure constant $\alpha(0)$; all quantities are given e.g.\ in ref.~\cite{Denner:2019vbn} and references therein.

{\bf MZ add plot of DeltaAlpha and discuss it a bit}
