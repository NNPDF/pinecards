\section{Validation and interpretation of PineAPPL grids}
\label{sec:results}

In this section we demonstrate the capabilities of \textsc{mg5\_aMC}+\textsc{PineAPPL} by
computing fast-interpolation grids, accurate to NLO QCD and NLO QCD+EW,
for a representative set of processes in which EW corrections may have a
sizeable effect on the accuracy of the theoretical predictions.
In order to consider some realistic kinematics for these
processes, we rely on a set of measurements that are commonly included in PDF
fits. Our aim is twofold. First, we want to validate the results
obtained with \textsc{PineAPPL}; second, we want to assess the
impact of the EW corrections for the specific experimental setups. We describe
the settings employed for the computations, and the corresponding results for
each process. We remind the reader that \textsc{mg5\_aMC} makes it possible
to compute predictions including NLO QCD and EW corrections for arbitrary processes
in an automated manner. It employs the FKS subtraction scheme~\cite{Frixione:1995ms,Frixione:1997np}
as automated in \textsc{MadFKS}~\cite{Frederix:2009yq,Frederix:2016rdc} to deal with IR singularities. One-loop
amplitudes are computed by \textsc{MadLoop}~\cite{Hirschi:2011pa}, which employs different
numerical techniques~\cite{Passarino:1978jh,Davydychev:1991va,Denner:2005nn,Ossola:2006us,Cascioli:2011va,Mastrolia:2012bu} implemented in the corresponding computer libraries~\cite{Ossola:2007ax,Peraro:2014cba,Hirschi:2016mdz,Denner:2016kdg}. Matching
with parton showers is available only for pure-QCD corrections via the MC@NLO method~\cite{Frixione:2002ik}, and 
will not be employed in the following.

\subsection{Computational settings}
\label{subsec:computational_settings}

We generate each process by means of the Universal FeynRules Output
(UFO)~\cite{Degrande:2011ua} model {\tt loop\_qcd\_qed\_sm\_Gmu},
included as standard in {\sc mg5\_aMC}. It contains the UV and $R_2$
counterterms relevant to NLO QCD and EW corrections, the latter in the
$\overline{G}_\mu$ scheme. The model features five massless quark flavours,
sets the CKM matrix equal to the identity, and is compatible with the usage of
the complex mass (CM) scheme~\cite{Denner:1999gp,Denner:2005fg} for all massive particles, see
ref.~\cite{Frederix:2018nkq} for details. We use this scheme
for all processes that involve only massless particles in the final state.
The photon is always considered as part of the proton in the initial state and
of any hadronic jet produced in the final state: 
 in particular, photon-induced (PI) contributions are consistently included at LO and NLO.\footnote{We employ the $\overline{G}_\mu$ scheme also for the QED coupling entering vertices involving initial-state photons, see section~4.3.3 of ref.\cite{Denner:2019vbn}.}
We use a PDF set that contains a photon PDF, namely
{\tt NNPDF31\_nlo\_as\_0118\_luxqed}~\cite{Bertone:2017bme}. We evaluate the PDF
uncertainty associated to the theoretical predictions a posteriori, that is,
we convolve the fast-interpolation grid generated with {\sc PineAPPL} with
each member in the PDF set, and we compute the associated standard deviation. Monte Carlo
weights are stored as Lagrange-interpolation grids.

The central values of the renormalisation and factorisation scales, $\mu_\mathrm{R}$ and
$\mu_\mathrm{F}$, are chosen in a process-specific way, as discussed in sec.~\ref{subsec:processes_and_measurements}. In order to estimate the missing higher-order uncertainty,
we allow the events to be reweighted on-the-fly in the Monte Carlo generation upon scale
variations, with the technique presented in Ref.~\cite{Frederix:2011ss}. To this purpose, we use the default \textsc{mg5\_aMC}
implementation, whereby the factorisation and renormalisation scales
are varied down to a factor $1/2$ and up to a factor $2$, and the envelope
from the nine-point scale variations is constructed,
see equation~\eqref{eq:9pt}. However, we note that
\textsc{PineAPPL} allows the user to determine the envelope with any point
prescription, see appendix~\ref{app:pineappl-demo} for an example.

The values of the relevant physical parameters are chosen as
\begin{equation}
\begin{aligned}
M_\mathrm{W} &= \SI{80.419}{\giga\electronvolt} \text{,} \quad &
M_\mathrm{Z} &= \SI{91.176}{\giga\electronvolt} \text{,} \quad &
m_\mathrm{t} &= \SI{172.5}{\giga\electronvolt} \text{,} \quad \\
\Gamma_\mathrm{W} &= \SI{2.09291}{\giga\electronvolt} \text{,} &
\Gamma_\mathrm{Z} &= \SI{2.49877}{\giga\electronvolt} \text{,} &
G_\mu &= \SI{1.16639e-5}{\per\giga\electronvolt\squared} \text{,}
\end{aligned}
\label{eq:parameters}
\end{equation}
where $M_\mathrm{Z}$, $M_\mathrm{W}$, $m_\mathrm{t}$ are the values of the Z-boson, W-boson, and
top-quark masses, respectively, $\Gamma_\mathrm{Z}$ and $\Gamma_\mathrm{W}$ are the widths of
the Z and W bosons, and $G_\mu$ is value of the Fermi coupling. The value of the strong
coupling is chosen consistently with the PDF set, $\alphas(M_\mathrm{Z})=0.118$.

The definition of observables and cuts is process-specific, and it follows the corresponding experimental measurements,
see section~\ref{subsec:processes_and_measurements}.
When relevant, final-state photons and massless charged fermions (leptons and light quarks) are recombined together 
if they satisfy the condition $\Delta R_{f \gamma}<0.1$, where $\Delta R_{f \gamma}$
is the fermion-photon distance. In this case the
sum of their momenta is assigned to the charged fermion, and the photon is removed
from the event. Kinematic observables and cuts are defined starting from recombined momenta. If we were interested also in jet-related observables,
photons surviving the recombination would have to be clustered together with coloured
partons.\footnote{For issues related to the definition of jets in presence of EW corrections, in particular
    about the fragmentation of partons into photons and vice-versa,
see refs.~\cite{Glover:1993xc,Frederix:2016ost,Denner:2019zfp}.} Finally, although contributions corresponding to the radiation
of a heavy boson are formally of the same perturbative order of the EW corrections, they are not included in our
computations. In fact, while nothing prevents to include these contributions \emph{a posteriori}, as they are finite,
their impact is either smaller than the one of \enquote{standard} EW corrections, or anyway negligible with respect to
the total cross section (see refs.~\cite{Frixione:2014qaa,Frixione:2015zaa,Pagani:2016caq} for some process-specific cases).
    

\subsection{Results for specific processes and measurements}
\label{subsec:processes_and_measurements}

We focus on the following three processes: Drell--Yan lepton-pair production,
top-quark pair production, and Z-boson (lepton-pair) production with non-zero
transverse momentum in proton-proton collisions. In the following, we shall
present the experimental measurements, the process-specific settings, and
the phenomenological results for each of these processes.

When presenting the results, for each of the processes and measurements considered, we compute differential
cross sections for the observables defined in the experimental analyses in two different
ways: directly, by means of \textsc{mg5\_aMC}, and a posteriori, by convolving
the fast-interpolation grid produced by \textsc{PineAPPL} with the PDF set
specified in section~\ref{subsec:computational_settings}. We
refer to the first result as the MC result, and to the second as
the \textsc{PineAPPL} result. We repeat the computation for theories accurate
to NLO QCD and to NLO QCD+EW, respectively. The corresponding orders of the
strong and EW couplings that we consider are specified for each process. In each case, we determine the
PDF uncertainty (coming from the PDF ensemble), the scale uncertainty (coming
from variations of the factorisation and renormalisation scales), and the Monte
Carlo uncertainty (coming from the finite number of events generated). In this
last respect, we consider by default high-statistics computations, whereby we
require a relative Monte Carlo precision of \SI{0.1}{\permille} on the integrated cross section. While
this choice does not affect the validation of the \textsc{PineAPPL} result
against the MC result, it ensures that the statistical uncertainty of
the computation remains negligible in comparison to the PDF and scale
uncertainties, as we will explicitly demonstrate. This is a desirable feature
to correctly interpret the size of the EW corrections. An example that
validates the \textsc{PineAPPL} result in the case of a low-statistic run is
nevertheless provided in appendix~\ref{app:add_plots}.

Our goal is indeed twofold. On the one hand, we aim to validate the
interpolation grids generated with \textsc{PineAPPL}: to this purpose we shall
verify that the MC and the \textsc{PineAPPL} results are identical up to
numerical inaccuracies due to the grid interpolation. This equivalence must
hold for any choice of renormalisation and factorisation scale and should not
depend on the MC uncertainty of the binned cross section. On the other
hand, we aim to study the size of the EW corrections, in particular with
respect to the kinematics of each process, and to three kinds of uncertainties:
the PDF uncertainty, the scale uncertainty, and the uncertainty of the
experimental data.

We present these comparisons in
figures~\ref{fig:atlaszhighmass49fb}, \ref{fig:cmsdy2d11_bins3456}, \ref{fig:atlastop}, and \ref{fig:cmsZ13TeV}.
The format of the plots is the same across all figures. The first panel
displays the relative difference (in per mille) between the {\sc PineAPPL} and
the MC results for the central scale choice and upper/lower edges of the
scale-uncertainty envelope, for
theories accurate to both NLO QCD and NLO QCD+EW. The following three panels
present the theoretical predictions, accurate to either NLO QCD or NLO QCD+EW,
always normalised to the former; on top of the theoretical predictions, the
PDF uncertainty, the scale uncertainty and the Monte Carlo uncertainty are
displayed in turn. The relative uncertainty of the experimental data is
also shown for comparison. We shall now discuss the results for each
process and data set.

%%%%%%%%%%%%%%%%%%%%%%%%%%%%%%%%%%%%%%%%%%%%%%%%%%%%%%%%%%
\subsubsection{Drell--Yan lepton pair production.}
\label{sec:dy-lepton-pair-production}

\paragraph{Experimental measurements and process features.}
We select the single-differential invariant mass distribution of the
lepton pair, $M_{\ell \bar\ell}$, measured by the ATLAS experiment at a centre-of-mass
energy of \SI{7}{\tera\electronvolt} in the high-mass region
($M_{\ell\bar\ell}>\SI{116}{\giga\electronvolt}$)~\cite{Aad:2013iua}.
We also select the single-differential rapidity distribution, $y_{\ell\bar\ell}$, in slices of
the invariant mass of the lepton pair, $M_{\ell\bar\ell}$,
measured by the CMS experiment at a centre-of-mass energy of
\SI{7}{\tera\electronvolt}~\cite{Chatrchyan:2013tia}.
These measurements are currently included as standard in the
NNPDF3.1~\cite{Ball:2017nwa} and MMHT2014~\cite{Harland-Lang:2014zoa} PDF sets,
although with appropriate kinematic cuts that remove the bins at the largest
values of invariant mass, where EW corrections become sizeable. As explained in
section~\ref{sec:pineappl-example}, 
the process has a single LO, $\mathcal{O}(\alpha^2)$; at NLO, the
QCD contribution is $\mathcal{O}(\alphas\alpha^2)$, while the EW contribution
is $\mathcal{O}(\alpha^3)$. Our NLO QCD computation includes the
$\mathcal{O}(\alpha^2)$ and $\mathcal{O}(\alphas\alpha^2)$ contributions, while
our NLO QCD+EW computation includes the $\mathcal{O}(\alpha^2)$, 
$\mathcal{O}(\alphas\alpha^2)$ and $\mathcal{O}(\alpha^3)$ contributions.
Combined QCD--EW corrections occur only at NNLO, and are therefore not
considered here. EW corrections for this process were computed in
refs.~\cite{Baur:2001ze,Arbuzov:2007db,Dittmaier:2009cr,Frederix:2018nkq}. The process receives contributions
from 13 (35) parton luminosities at NLO QCD (NLO QCD+EW),
see appendix~\ref{app:lumis} for details.

\paragraph{Process-specific settings.}
We use a fixed value for the renormalisation and factorisation scales $\mu_\mathrm{R}=\mu_\mathrm{F}=M_\mathrm{Z}$, where $M_\mathrm{Z}$ is
the mass of the Z boson, for the ATLAS measurement, and the scale
$\mu_\mathrm{R}=\mu_\mathrm{F}=M_{\ell\bar\ell}$, where $M_{\ell\bar\ell}$ is the central value of each
invariant mass slice, for the CMS one. In the case of ATLAS, we require $p_\mathrm{T}^\ell>\SI{25}{\giga\electronvolt}$,
$|\eta_\ell|<2.5$ and $\SI{116}{\giga\electronvolt}<M_{\ell\bar\ell}<\SI{1500}{\giga\electronvolt}$ for the transverse
momentum and the pseudorapidity of each lepton and for the invariant mass of the
lepton pair, respectively. Conversely, in the case of CMS, we require $p_\mathrm{T}^{\ell_1}>\SI{14}{\giga\electronvolt}$, $p_\mathrm{T}^{\ell_2}>\SI{9}{\giga\electronvolt}$,
$|\eta_\ell|<2.4$, $|y_{\ell\bar\ell}|<2.4$ and $\SI{20}{\giga\electronvolt}<M_{\ell\bar\ell}<\SI{1500}{\giga\electronvolt}$
for the transverse momentum and the pseudorapidity of each lepton, and for the
rapidity and the invariant mass of the lepton pair.

\paragraph{Numerical results.}
We first consider the single-differential measurement of a lepton-pair
for large invariant masses performed by the ATLAS experiment at \SI{7}{\tera\electronvolt}.
From figure~\ref{fig:atlaszhighmass49fb} we immediately
observe that the validation of the \textsc{PineAPPL} result against the MC
result is successful. The relative difference between the two is of the order of
\SI{0.1}{\permille} at most, with negligible fluctuations across different
invariant mass bins. The agreement is similarly good irrespective of the
perturbative accuracy of the theory (NLO QCD or NLO QCD+EW) or of the scale
choice. As explicitly demonstrated in appendix~\ref{app:add_plots}, the good
agreement is also independent from the numerical precision of the Monte Carlo
run.

%-------------------------------------------------------------------------------
\begin{figure}[!t]
    \centering
    \includegraphics[width=0.5\textwidth]{figures/pineappl_ATLASZHIGHMASS49FB}\\
    \caption{Validation and test of the \textsc{PineAPPL} grid for the ATLAS
      high-mass Drell--Yan lepton pair measurement in the high-mass region at
      a centre-of-mass energy of \SI{7}{\tera\electronvolt}~\cite{Aad:2013iua}. The first panel
      displays the relative difference (in per mille) between the {\sc PineAPPL}
      and the MC results for the central, upper and lower scale choices,
      for theories accurate to both NLO QCD and NLO QCD+EW. The second, third
      and fourth panels present the theoretical predictions, accurate to either
      NLO QCD or NLO QCD+EW, always normalised to the former; on top of the
      theoretical predictions, the PDF uncertainty, the scale uncertainty and
      the Monte Carlo uncertainty are displayed in turn. The relative
      uncertainty of the experimental data is also shown for comparison.}
    \label{fig:atlaszhighmass49fb}
\end{figure}
%-------------------------------------------------------------------------------

The measurement is mainly driven by $\mathrm{q}\bar{\mathrm{q}}$ scattering: specifically, the
leading (next-to-leading) contribution comes from a $\mathrm{u}\bar{\mathrm{u}}$/$\mathrm{c}\bar{\mathrm{c}}$ ($\mathrm{d}\bar{\mathrm{d}}$/$\mathrm{s}\bar{\mathrm{s}}$)
parton luminosity, which accounts for about \SI{55}{\percent} (\SI{49}{\percent}) of the cross section
for the lowest invariant mass bins, and \SI{68}{\percent} (\SI{22}{\percent}) for the largest invariant
mass bins.\footnote{The size of these contributions may depend on the input PDF
  set. Here and in the following, we always quote results obtained from the
  \texttt{NNPDF31\_nlo\_as\_0118\_luxqed} PDF set.}
The PI contribution raises from about \SI{1.3}{\percent} in the lowest bin to
about \SI{3.6}{\percent} in the highest bin.\footnote{For the \texttt{MRST2004qed} PDF set~\cite{Martin:2004dh}, which is used to subtract PI contributions (see section~\ref{sec:doublecounting}), the value in the highest bin is roughly twice as large, \SI{6.9}{\percent} (also in absolute numbers).}
Overall, the EW corrections range between \SI{5}{\percent}
around $M_{\ell\bar\ell}\sim \SI{150}{\giga\electronvolt}$, \SIrange{2}{3}{\percent} for intermediate invariant mass
values, $\SI{150}{\giga\electronvolt}\lesssim M_{\ell\bar\ell}\lesssim \SI{700}{\giga\electronvolt}$, and
\SIrange{15}{20}{\percent} for the largest invariant mass bin, $M_{\ell\bar\ell}>\SI{1000}{\giga\electronvolt}$,
see figure~\ref{fig:atlaszhighmass49fb}. For this reason, the data points with
$M_{\ell\bar\ell}>\SI{210}{\giga\electronvolt}$ were not included in the NNPDF3.1
analysis~\cite{Ball:2017nwa}.

The NLO QCD+EW corrections always lead to a reduction of the cross section in
comparison to the NLO QCD prediction. The size of this shift is comparable to
the data uncertainty at small values of $M_{\ell\bar\ell}$, and rapidly becomes
negligible with respect to it as the value of the invariant mass increases
and the data uncertainty blows up. This fact suggests a couple of observations
in light of the inclusion of EW corrections in a fit of PDFs. First, the
description of the more precise bins in the low invariant mass range is likely
to change, and will possibly become more accurate should the inclusion of EW
corrections improve the data/theory agreement. Second, the kinematic cut that
excludes any data point at large $M_{\ell\bar\ell}$ can be safely removed: any
shift in the predictions induced by the more accurate NLO QCD+EW theory is
likely to be easily accommodated by the large data uncertainty. However, 
this conclusion might not continue to hold: first, data
will become more precise, in particular thanks to the increased Run-II LHC
luminosity. Second, both EW corrections and PI contributions are expected to
grow larger at \SI{13}{\tera\electronvolt}, making their impact more relevant.

In comparison to the PDF uncertainty, the size of the EW corrections is
always larger, especially at the boundaries of the distribution. This fact
suggests that, once included in a global fit, EW corrections will make PDFs
more accurate. In comparison to the scale
uncertainty, the size of the EW correction is similar, except for the four bins
at the largest invariant mass, where the latter is significantly larger than
the former. This fact suggests that the impact of NNLO QCD corrections~\cite{Anastasiou:2003yy,Catani:2009sm,Gavin:2010az,Li:2012wna,Boughezal:2016wmq} is comparable
to the one of NLO QCD+EW, except at very large values of the invariant mass,
where the EW correction still dominates. This result stresses the need to
include the EW corrections in order to obtain an accurate description of the
large invariant mass bins. Finally, the Monte Carlo 
uncertainty remains negligible in comparison to the data, PDF and scale
uncertainties, and to the size of the EW correction. Our conclusions should
therefore not be affected by a generation of too few Monte Carlo events.

We then turn our attention to the double-differential measurement performed by
the CMS experiment at \SI{7}{\tera\electronvolt}. For illustrative purposes, we report only four out
of the six invariant mass bins, respectively below the Z-boson mass peak,
$\SI{45}{\giga\electronvolt}<M_{\ell\bar\ell}<\SI{60}{\giga\electronvolt}$, on the Z-boson mass peak,
$\SI{60}{\giga\electronvolt}<M_{\ell\bar\ell}<\SI{120}{\giga\electronvolt}$, above the mass peak,
$\SI{120}{\giga\electronvolt}<M_{\ell\bar\ell}<\SI{200}{\giga\electronvolt}$, and at very high invariant masses,
$\SI{200}{\giga\electronvolt}<M_{\ell\bar\ell}<\SI{1500}{\giga\electronvolt}$, see figure~\ref{fig:cmsdy2d11_bins3456}.
Analogous plots for the remaining low invariant mass bins are collected in
appendix~\ref{app:add_plots}. From figure~\ref{fig:cmsdy2d11_bins3456},
first of all we validate the \textsc{PineAPPL} result: its relative difference
with respect to the MC result is always below a fraction of per mille,
again irrespective of the accuracy of the theory, of the choice of scale, and
of the kinematic bin considered. 

%-------------------------------------------------------------------------------
\begin{figure}[!p]
    \centering
    \includegraphics[width=0.46\textwidth]{figures/pineappl_CMSDY2D11_bin3}%
    \includegraphics[width=0.46\textwidth]{figures/pineappl_CMSDY2D11_bin4}\\
    \includegraphics[width=0.46\textwidth]{figures/pineappl_CMSDY2D11_bin5}%
    \includegraphics[width=0.46\textwidth]{figures/pineappl_CMSDY2D11_bin6}\\
    \caption{Same as figure~\ref{fig:atlaszhighmass49fb}, but for the CMS
      double-differential Drell--Yan lepton pair measurement at a
      centre-of-mass energy of \SI{7}{\tera\electronvolt}~\cite{Chatrchyan:2013tia}. Displayed are
      only four of the six invariant mass bins available, respectively below the
      Z-boson mass peak, $\SI{45}{\giga\electronvolt}<M_{\ell\bar\ell}<\SI{60}{\giga\electronvolt}$, on the Z-boson mass
      peak, $\SI{60}{\giga\electronvolt}<M_{\ell\bar\ell}<\SI{120}{\giga\electronvolt}$, above the mass peak,
      $\SI{120}{\giga\electronvolt}<M_{\ell\bar\ell}<\SI{200}{\giga\electronvolt}$, and at very high invariant masses,
      $\SI{200}{\giga\electronvolt}<M_{\ell\bar\ell}<\SI{1500}{\giga\electronvolt}$.
      Results for the slices $\SI{45}{\giga\electronvolt}<M_{\ell\bar\ell}<\SI{60}{\giga\electronvolt}$ and $\SI{45}{\giga\electronvolt}<M_{\ell\bar\ell}<\SI{60}{\giga\electronvolt}$ can be found in figure~\ref{fig:cmsdy2d11_bins12}.}
    \label{fig:cmsdy2d11_bins3456}
\end{figure}
%-------------------------------------------------------------------------------

As in the case of the ATLAS high-mass Drell--Yan measurement, the CMS
measurement is also dominated by $\mathrm{q}\bar{\mathrm{q}}$ scattering. The leading
(next-to-leading) contribution to the $\SI{45}{\giga\electronvolt}<M_{\ell\bar\ell}<\SI{60}{\giga\electronvolt}$ invariant
mass bin comes from
the $\mathrm{u}\bar{\mathrm{u}}$/$\mathrm{c}\bar{\mathrm{c}}$ ($\mathrm{d}\bar{\mathrm{d}}$/$\mathrm{s}\bar{\mathrm{s}}$) parton luminosity, which accounts for about \SI{70}{\percent}
(\SI{22}{\percent}) of the double differential cross section, with small fluctuations across
the rapidity range. The PI contribution decreases from about \SI{4}{\percent} at zero
rapidity to \SI{1.5}{\percent} in the largest rapidity bin. The situation is slightly
different in the $\SI{60}{\giga\electronvolt}<M_{\ell\bar\ell}<\SI{120}{\giga\electronvolt}$ invariant mass bin, where the
leading (next-to-leading) contribution comes instead from the $\mathrm{d}\bar{\mathrm{d}}$/$\mathrm{s}\bar{\mathrm{s}}$
($\mathrm{u}\bar{\mathrm{u}}$/$\mathrm{c}\bar{\mathrm{c}}$) parton luminosity, which accounts for about \SI{60}{\percent} (\SI{44}{\percent}) of the
double differential cross section at small rapidities, and for about \SI{56}{\percent} (\SI{50}{\percent})
at large rapidities. In the remaining two invariant mass bins, the leading
(next-to-leading) contribution comes again from the $\mathrm{u}\bar{\mathrm{u}}$/$\mathrm{c}\bar{\mathrm{c}}$ ($\mathrm{d}\bar{\mathrm{d}}$/$\mathrm{s}\bar{\mathrm{s}}$)
parton luminosity, which accounts for about \SIrange{69}{95}{\percent} (\SIrange{38}{30}{\percent}) and
\SIrange{57}{70}{\percent} (\SIrange{48}{34}{\percent}) of the cross section, respectively for
$\SI{120}{\giga\electronvolt}<M_{\ell\bar\ell}<\SI{200}{\giga\electronvolt}$ and $\SI{200}{\giga\electronvolt}<M_{\ell\bar\ell}<\SI{1500}{\giga\electronvolt}$ in the
corresponding rapidity intervals; PI contributions range between \SIrange{3.7}{0.6}{\percent}
and \SIrange{7.3}{1.6}{\percent} in the two invariant mass bins, respectively, for increasing
rapidity.

The way in which NLO QCD+EW corrections affect the theoretical prediction for
the double differential cross section (with respect to its counterpart accurate
to NLO QCD) depends on the invariant mass bin. In the
$\SI{45}{\giga\electronvolt}<M_{\ell\bar\ell}<\SI{60}{\giga\electronvolt}$ region, they enhance the value of the cross
section by about \SI{11}{\percent} across all the rapidity range. This is mostly due to photon-radiation effects
on events with $M_{\ell\bar\ell}\simeq M_\mathrm{Z}$ at the Born, for which the invariant mass is shifted to lower 
values. In the 
$\SI{60}{\giga\electronvolt}<M_{\ell\bar\ell}<\SI{120}{\giga\electronvolt}$ region, EW corrections suppress the value of the cross
section by about \SI{2}{\percent}, again across all the rapidity range; in the
$\SI{120}{\giga\electronvolt}<M_{\ell\bar\ell}<\SI{200}{\giga\electronvolt}$ bin, the suppression is around \SIrange{4}{5}{\percent}; and in
the $\SI{200}{\giga\electronvolt}<M_{\ell\bar\ell}<\SI{1500}{\giga\electronvolt}$ bin, the suppression increases further to
about \SIrange{6}{7}{\percent} for rapidities $y_{\ell\bar\ell}<2.0$ and up to \SIrange{20}{40}{\percent} at forward
rapidities. For this reason, for instance, the data points with
$M_{\ell\bar\ell}>\SI{200}{\giga\electronvolt}$ and $y_{\ell\bar\ell}>2.2$ were not included in the
NNPDF3.1 analysis~\cite{Ball:2017nwa}.

In general, the size of the EW corrections is comparable to or slightly larger
than the data uncertainty, except for the invariant mass bin 
$\SI{45}{\giga\electronvolt}<M_{\ell\bar\ell}<\SI{60}{\giga\electronvolt}$,
where the shift due to the EW correction overshoots the data uncertainty by
about a factor of ten, and at large rapidities, where the shift due to the EW
correction, although it can become large, is always a fraction of the data
uncertainty. Because EW effects are subtracted from the data used
in PDF fits (see section~\ref{sec:doublecounting}), a good agreement between
data and theory is usually achieved without the inclusion of EW corrections.
However, as already observed in the case of the ATLAS measurement, should EW
corrections be included in a fit of PDFs, the latter are likely to become more
accurate: even though the apparent description of the data will not
improve, by including the more precisely predicted bins in the low invariant mass range, PDFs
will resemble more closely the underlying truth. Furthermore, the kinematic cut
that excludes any data point at large invariant mass and/or rapidity can be
safely removed.

In comparison to the PDF uncertainty, the size of the EW corrections is always
larger. We therefore anticipate that, even if the agreement between the more
accurate theory (including EW corrections) and the data will remain the same,
the PDFs will however become overall more accurate. In comparison to the scale
uncertainty, the size of the EW correction is similar, except on the Z-boson
mass peak, $\SI{60}{\giga\electronvolt}<M_{\ell\bar\ell}<\SI{120}{\giga\electronvolt}$, where the scale uncertainty exceeds
the size of the EW correction by about a factor of five. This is due to the choice
of invariant mass window around the Z peak, in which positive and negative EW corrections almost cancel.
Finally, the Monte Carlo uncertainty remains negligible
in comparison to the data, PDF and scale uncertainties, and to the size of the
EW correction, except for a couple of bins at forward/backward rapidity in the highest
invariant mass bins. Improving the Monte Carlo precision will require to
generate a larger number of events, possibly with cuts that select
only the kinematic bins affected by the largest MC uncertainties. If this
turned out to be computationally too expensive, it would be desirable to treat
this uncertainty as an additional theoretical uncertainty in the PDF
fit~\cite{Ball:2018lag}.


%%%%%%%%%%%%%%%%%%%%%%%%%%%%%%%%%%%%%%%%%%%%%%%%%%%%%%%%%%
\subsubsection{Top-quark pair production.}
\label{sec:toppair}

\paragraph{Experimental measurements and process features.}
We select the single-differential distribution in either the transverse
momentum of the top quark, $p_\mathrm{T}^\mathrm{t}$, or the invariant mass of the top-quark
pair, $m_{\mathrm{t}\bar{\mathrm{t}}}$, measured by the ATLAS and CMS experiments at a centre-of-mass
energy of \SI{8}{\tera\electronvolt}~\cite{Aad:2015mbv,Khachatryan:2015oqa}. These measurements have
been extensively studied in the context of PDF fits in
refs.~\cite{Czakon:2016olj,Bailey:2019yze,Amoroso:2020lgh,Kadir:2020yml} (see
also refs.~\cite{Pagani:2016caq,Czakon:2017wor} for studies related to the photon density) and
included by default in the CT18~\cite{Hou:2019efy} analysis.
Because EW corrections are significantly smaller for distributions differential
in the rapidity of either the top quark or the top-quark
pair~\cite{Czakon:2017wor}, these distributions were preferred for inclusion
in the NNPDF3.1 analysis~\cite{Ball:2017nwa}. The process receives
pure QCD contributions at LO, $\mathcal{O}(\alphas^2)$, and
at NLO, $\mathcal{O}(\alphas^3)$. These orders make up our NLO QCD
computation. The NLO QCD+EW computation includes the $\mathcal{O}(\alphas^2)$
and $\mathcal{O}(\alphas^3)$ QCD contributions, the LO contribution
$\mathcal{O}(\alphas\alpha)$ and the NLO contribution
$\mathcal{O}(\alphas^2\alpha)$.
We do not consider the LO contribution $\mathcal{O}(\alpha^2)$ nor the
NLO contributions $\mathcal{O}(\alphas\alpha^2)$ and $\mathcal{O}(\alpha^3)$, which have been shown to be negligible~\cite{Czakon:2017wor,Frederix:2018nkq}.
EW corrections for this process
were computed in refs.~\cite{Bernreuther:2010ny,Hollik:2011ps,Kuhn:2011ri,Bernreuther:2012sx,Pagani:2016caq,Czakon:2017wor,Czakon:2017lgo,Czakon:2017mmr,Frederix:2018nkq,Czakon:2019bcq,Czakon:2019txp}. The process receives contributions from
7 (37) parton luminosities at NLO QCD (NLO QCD+EW),
see appendix~\ref{app:lumis} for details.

\paragraph{Process-specific settings.}
We employ the following functional form for the renormalisation and factorisation scales;
$\mu_\mathrm{R}=\mu_\mathrm{F}=\sqrt{m_\mathrm{t}^2+(p_\mathrm{T}^\mathrm{t})^2}{\Big /}2$ for the distribution differential
in the transverse momentum of the top quark, and $\mu_\mathrm{R}=\mu_\mathrm{F}=H_\mathrm{T}/4$ for the
distribution differential in the invariant mass of the top-quark pair, where
$H_\mathrm{T}=\sqrt{m_\mathrm{t}^2+(p_\mathrm{T}^\mathrm{t})^2}+\sqrt{m_\mathrm{t}^2+(p_\mathrm{T}^{\bar{\mathrm{t}}})}$, with $m_\mathrm{t}$,
$p_\mathrm{T}^\mathrm{t}$ and $p_\mathrm{T}^{\bar{\mathrm{t}}}$ the mass of the top quark and the transverse momenta
of the top and antitop quarks, respectively. These choices were demonstrated
to maximise the convergence of the perturbative expansion~\cite{Czakon:2016dgf}. No cuts are imposed.

\paragraph{Numerical results.}

In figure~\ref{fig:atlastop} we report the
distributions differential in the transverse momentum of the top quark,
$p_\mathrm{T}^\mathrm{t}$, and in the invariant mass of the top-quark pair,
$m_{\mathrm{t}\bar{\mathrm{t}}}$. Analogous plots for the distributions differential
in the rapidity of either the top quark or the top-quark pair are collected in
appendix~\ref{app:add_plots}. From figure~\ref{fig:atlastop}, we immediately
validate the \textsc{PineAPPL} result: its relative difference with respect to
the MC result is at most as large as \SI{0.4}{\permille}, irrespective of the
accuracy of the theory, of the choice of scale and of the distribution
considered.

%-------------------------------------------------------------------------------
\begin{figure}[!t]
    \centering
    \includegraphics[width=0.5\textwidth]{figures/pineappl_ATLAS_TTB_DIFF_8TEV_LJ_TPT}%
    \includegraphics[width=0.5\textwidth]{figures/pineappl_ATLAS_TTB_DIFF_8TEV_LJ_TTM}
    \caption{Same as figure~\ref{fig:atlaszhighmass49fb}, but for the ATLAS
      differential top-quark pair measurement at a centre-of-mass energy of
      \SI{8}{\tera\electronvolt}~\cite{Aad:2015mbv}. Displayed are the distributions in the
      transverse momentum of the top quark $p_\mathrm{T}^\mathrm{t}$ (left), and in the invariant
      mass of the top-quark pair $m_{\mathrm{t}\bar{\mathrm{t}}}$ (right).}
    \label{fig:atlastop}
\end{figure}
%-------------------------------------------------------------------------------

The process receives its leading contribution from the $\mathrm{gg}$ channel,
which varies between \SI{81}{\percent} and \SI{61}{\percent} (\SI{76}{\percent} and \SI{83}{\percent}) of the $p_\mathrm{T}^\mathrm{t}$
($m_{\mathrm{t}\bar{\mathrm{t}}}$) differential cross section as the value of the
transverse momentum of the top quark (the invariant mass of the top-quark pair)
increases; the largest PI contribution for this process comes from $\gamma\mathrm{g}$ scattering,
which accounts for about \SIrange{0.5}{1}{\percent} (\SIrange{0.5}{0.7}{\percent}) of the cross
section, and is almost entirely (\SI{90}{\percent}) due to the LO contribution at $\mathcal{O}(\alphas\alpha)$; the contribution from other PI parton luminosities is comparatively
negligible. Overall, the EW corrections suppress the $p_\mathrm{T}^\mathrm{t}$
($m_{\mathrm{t}\bar{\mathrm{t}}}$) distribution by about \SIrange{0.2}{3.5}{\percent}
(\SIrange{0.5}{0.2}{\percent}) for increasing values of $p_\mathrm{T}^\mathrm{t}$
($m_{\mathrm{t}\bar{\mathrm{t}}}$), except in the first bin of the
$p_\mathrm{T}^\mathrm{t}$ distribution, where they enhance the cross section by
about \SI{1}{\percent}. The size of these shifts, however, remains always significantly
smaller than the data uncertainty.\footnote{In
  figure~\ref{fig:atlastop} the data uncertainty corresponds to the ATLAS
  measurement~\cite{Aad:2015auj}. Similar considerations apply also for the CMS
  measurement~\cite{Khachatryan:2015oaa}.}
As a consequence, we anticipate that the more accurate NLO QCD+EW theory is
likely to be easily accommodated by the large data uncertainty, should the data
be fitted with the inclusion of EW corrections.

The size of the EW correction is comparable to the size of the PDF uncertainty,
except at large values of transverse momentum or invariant mass, where the
former becomes larger than the latter. This fact suggests that, once included in
a global fit, EW corrections can improve the accuracy of the PDFs. In comparison
to the scale uncertainty, the size of the EW corrections remains negligible:
despite the fact that the choice of factorisation and renormalisation scales
have been devised to optimise the convergence of the perturbative expansion,
NNLO QCD corrections remain large, as expected in a process mostly initiated
by gluons. Their inclusion is therefore mandatory in a fit of PDFs. Finally,
the Monte Carlo statistical uncertainty remains negligible in comparison to the
data, PDF and scale uncertainties, and to the size of the EW correction. Our
conclusions are therefore not affected by Monte Carlo inefficiencies.

%%%%%%%%%%%%%%%%%%%%%%%%%%%%%%%%%%%%%%%%%%%%%%%%%%%%%%%%%%
\subsubsection{Z-boson production with non-zero transverse momentum.}
\label{sec:Zpt}

\paragraph{Experimental measurements and process features.}
We select the single-differential transverse momentum distribution of the
sum of the two leptons (the \enquote{Z boson}), $p_\mathrm{T}^{\ell \bar{\ell}}$, measured by the CMS experiment at a
centre-of-mass energy of \SI{13}{\tera\electronvolt}~\cite{Sirunyan:2019bzr}.
This measurement, which has not been included in any PDF determination so far,
shows very low experimental uncertainties (at the percent level or below).
EW corrections are therefore expected to be essential to achieve a good
description of it, and to constrain accurately the PDFs, together with
NNLO QCD corrections, which are already well
known~\cite{Boughezal:2015ded,Boughezal:2016isb,Boughezal:2016yfp,Ridder:2015dxa,Ridder:2016nkl,Gehrmann-DeRidder:2017mvr,Bizon:2019zgf}
Analogous measurements,
from the ATLAS~\cite{Aad:2015auj} and CMS~\cite{Khachatryan:2015oaa}
experiments at a centre-of-mass energy of \SI{8}{\tera\electronvolt},
were partly included (upon
selection of an appropriate kinematic cut that excluded bins with large EW
corrections) in a dedicated study~\cite{Boughezal:2017nla},
in the NNPDF3.1 PDF set~\cite{Ball:2017nwa} and in variants of
the CT18 PDF set~\cite{Hou:2019efy}. In the QCD computation, we consider a
single LO contribution $\mathcal{O}(\alphas\alpha^2)$ and a single NLO
contribution $\mathcal{O}(\alphas^2\alpha^2)$. In the NLO QCD+EW computation,
we supplement these with another LO and NLO, which are
$\mathcal{O}(\alpha^3)$ and $\mathcal{O}(\alphas\alpha^3)$;
contributions of the order $\mathcal{O}(\alpha^4)$ are not considered (see ref.~\cite{Denner:2019zfp}).
EW corrections for this process were computed in
refs.~\cite{Kuhn:2005az,Denner:2011vu,Hollik:2015pja,Kallweit:2015dum,Frederix:2018nkq}.
The process receives contributions from 101 (166) parton luminosities,
see appendix~\ref{app:lumis}.

\paragraph{Process-specific settings.}

As in the case of Drell-Yan, we use a fixed value for the renormalisation and
factorisation scales $\mu_\mathrm{R}=\mu_\mathrm{F}=M_\mathrm{Z}$, where
$M_\mathrm{Z}$ is the Z-boson mass. Consistently with the
experimental analysis, we require $p_\mathrm{T}^\ell>\SI{25}{\giga\electronvolt}$,
$|\eta_\ell|<2.4$, $|M_{\ell\bar\ell} - M_\mathrm{Z}| < \SI{15}{\giga\electronvolt}$,
$|y_{\ell\bar\ell}|<2.4$ and $\SI{20}{\giga\electronvolt}<p_\mathrm{T}^{\ell\bar\ell}<\SI{1500}{\giga\electronvolt}$ for the
transverse momentum and pseudorapidity of each lepton, and for the invariant
mass, pseudorapidity and transverse momentum of the lepton pair.
We finally discard all bins with
$p_\mathrm{T}^{\ell \bar{\ell}}<\SI{20}{\giga\electronvolt}$ to avoid a kinematic
region where resummation effects are sizeable.

\paragraph{Numerical results.} In figure~\ref{fig:cmsZ13TeV} we report the
distribution differential in the transverse momentum of the Z
boson. Also in this case, the \textsc{PineAPPL} result is well validated, as it
differs from the MC result by $\SI{0.5}{\permille}$ at most. The accuracy of the
theory or the choice of scale do not alter this conclusion.

%-------------------------------------------------------------------------------
\begin{figure}[!t]
    \centering
    \includegraphics[width=0.5\textwidth]{figures/pineappl_CMS_Z_13_TEV}
    \caption{Same as figure~\ref{fig:atlaszhighmass49fb} but for the 
      CMS differential Z $p_\mathrm{T}$ measurement at a centre-of-mass energy of
      \SI{13}{\tera\electronvolt}~\cite{Sirunyan:2019bzr}.}
    \label{fig:cmsZ13TeV}
\end{figure}
%-------------------------------------------------------------------------------

The process receives its leading contribution from $\mathrm{qg}$- and
$\bar{\mathrm{q}}\mathrm{g}$-initiated channels, which account for about \SI{65}{\percent}
of the cross section, with some variations in the relative contributions from
individual quark and antiquark flavours as the transverse momentum of the
Z boson varies; the PI contribution always remain negligible.
Overall, the EW corrections suppress the theoretical predictions by about
\SIrange{2}{10}{\percent} as the transverse momentum of the Z boson increases.
The size of this shift is as large as the data uncertainty up to
$p_{\mathrm{T}}^{\ell\bar\ell}\sim \SI{200}{\giga\electronvolt}$, and 
exceeds it by about \SI{60}{\percent} at larger values of
$p_\mathrm{T}^{\ell \bar{\ell}}$. As a consequence,
we anticipate the inclusion of EW corrections to be relevant for an accurate
fit of this data.

The size of the EW correction is between four and twenty times larger than the
size of the PDF uncertainty: as previously noted in the other cases, this fact
suggests that, once included in a PDF fit, EW corrections can improve the
accuracy of the PDFs. In comparison to the scale uncertainty, the size of the
EW corrections remains negligible at small values of $p_{\mathrm{T}}^{\ell\bar\ell}$,
roughly $p_{\mathrm{T}}^{\ell\bar\ell}\lesssim\SI{400}{\giga\electronvolt}$, while it
becomes up to four times larger than it in the two bins at the largest value of
$p_{\mathrm{T}}^{\ell\bar\ell}$. In this kinematic region, NLO EW corrections might
therefore become even more relevant than NNLO QCD corrections, and should
therefore be mandatorily included in a fit of PDFs to this data set.
Finally, the Monte Carlo uncertainty is well under control, as it remains
mostly negligible in comparison to the PDF, scale and data uncertainty, and to
the size of the EW correction.

