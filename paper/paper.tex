\documentclass[a4paper,11pt]{article}
\pdfoutput=1

\usepackage{jheppub}
\usepackage[T1]{fontenc}

\newcommand{\alphas}{\alpha_{S}}

\title{Including EW and mixed QCD-EW corrections into PDF fits}

\author[a,b]{S. Carrazza,}
\author[b]{E. Nocera,}
\author[a]{C. Schwan}
\author[a,b]{and M. Zaro}

\affiliation[a]{Tif Lab, Dipartimento di Fisica, Universit\`a di Milano and INFN, Sezione di Milano, 20133 Milano, Italy}
\affiliation[b]{Nikhef Theory Group, Science Park 105, 1098 XG Amsterdam, The Netherlands}

\emailAdd{stefano.carrazza@mi.infn.it}
\emailAdd{e.nocera@nikhef.nl}
\emailAdd{christopher.schwan@mi.infn.it}
\emailAdd{marco.zaro@mi.infn.it}

\abstract{}

\begin{document}

\maketitle
\flushbottom

\section{Introduction}
\label{sec:introduction}

ERN

The impact of EW corrections to LHC data.\\
How to do that consistently, include systematically EW corrections, blabla.\\
Outline a possible strategy to deal with EW correction in LHC observables (FSR, PHOTOS, dressed/born, ...).\\
Pinappl kills amcfast and mcgrid, replace applgrid and fastnlo.\\

\section{Technical aspects}
\label{sec:technical-aspects}

\subsection{PineAPPL}
\label{sec:pineappl}

In this paper we introduce a new program called \textsc{PineAPPL}, which solves the same problem as \textsc{APPLgrid}~\cite{} and \textsc{fastNLO}~\cite{}, but also supports EW corrections, which are the main interest here.
In particular, the following features distinguish it from the two aforementioned programs:
\begin{itemize}
\item support for arbitrary fixed-order calculations in $\alpha$ and $\alpha_\mathrm{s}$ with
\item support for scale variations in multiples of the central renormalization and factorization scale,
\item support for all-order predictions from resummation and photon-/parton-showers,
\item a simple \textsc{C}-interface, with a wrapper for \textsc{Fortran}
\item and support for it implemented in \textsc{mg5\_aMC@NLO},
\item converters to convert \textsc{APPLgrids} and \textsc{fastNLO} tables.
\end{itemize}

\subsubsection{Cross sections in a multi-coupling expansion}
The structure of the cross-section weights needed by \textsc{PineAPPL} follows the scheme outlined in
Refs.~\cite{Frederix:2011ss, Bertone:2014zva}, embedded in a mixed-coupling expansion, see Ref.~\cite{Frederix:2018nkq} and Refs.CITE for
specific examples. Starting
from the latter, an observable $\Sigma(\alphas, \alpha)$ is written as
\begin{equation}
    \Sigma(\alphas, \alpha) = \alpha^c \alphas^{c_S} \sum_{a_S, a} \alphas^{a_S} \alpha^a \Sigma_{a_S, a}\,.
\end{equation}
$c$ and $c_S$ are process-dependent; the contributions $\Sigma_{a_S, a}$ are in general non-zero for $a_S, a \ge 0$ and $a_S + a > q$, where also $q$ is a process-dependent
uantity. For example,
for (stable) top-pair production, $c=c_S=0$ and $q=2$; for Drell-Yan production, $c=2$, $c_S=0$, and $q=0$. Terms
with  $a_S + a = q + k$ correspond to different N$^k$LO contributions to $\Sigma$, usually
labeled as N$^k$LO$_i$, $i =1,2, \ldots$, with $i=1$ assigned to the term with the largest power of $\alphas$. Given the hierarchy of the couplings,
one expects that
\begin{equation}
 \textrm{N}^k\textrm{LO}_1 \gg \textrm{N}^k\textrm{LO}_2 \gg\ldots \, ,
\end{equation}
however such a relation is not always respected, and sometimes blatantly violated~CITE.

Given the perturbative order (LO, NLO, \ldots), different weights $\cal W$, with different kinematics expressed by the set $\cal P$, enter the various
terms $\Sigma_{a_S, a}$. One can write
\begin{equation}
    \Sigma_{a_S, a}= \sum_{l\in \mathcal P} f_1(x_1^{(l)},\mu_F^{(l)}) \,f_2(x_2^{(l)},\mu_F^{(l)}) \mathcal W^{(l)}_{a_S, a}
    d \textrm{PS}\,,
\end{equation}
where we have introduced the parton distributions $f_{1,2}$ and the phase-space measure $d$PS. The structure of the weights $ \mathcal W^{(l)}_{a_S, a}$ is 
identical to the case when only correction of QCD origin are treated. In particular:
\begin{itemize}
    \item At LO, $\mathcal P = \{B\}$, $B$ being the Born kinematics, and the weight structure is trivial,
    \begin{equation}
        \mathcal W^{(l)}_{a_S, a} = {\mathcal W^{(l),0}_{a_S, a}}\,.
    \end{equation}
    \item At NLO, $\mathcal P = \{E, S, C, SC\}$ with, $E$ being the event (or resolved) kinematics, and $S$, $C$, $SC$ being
        respectively the soft, collinear, and soft-collinear kinematics. The weight structure includes three terms at NLO:
    \begin{equation}
        \mathcal W^{(l)}_{a_S, a} = {{\mathcal W}^{(l),0}_{a_S, a}} +
                                {\mathcal W^{(l),R}_{a_S, a}} \log\left(\frac{{\mu_R^{(l)}}^2}{Q^2}\right) +
                                {\mathcal W^{(l),F}_{a_S, a}} \log\left(\frac{{\mu_F^{(l)}}^2}{Q^2}\right)
    \end{equation}
    where, besides the renormalisation and factorisation scales $\mu_{R,F}$, we have introduced the Ellis-Sexton scale $Q$
\end{itemize}


\subsection{Grid representation}
\label{sec:grid-representation}

\subsection{Accuracy and performance}
\label{sec:accuracy-and-performance}

CS, MZ

How to integrate in MC codes, how this works in aMCblast (setup, MG5 version, urls).\\
Appendix: how to create and fill grids, + simple code.


\section{Results}

SC

For datasets in 3.1qed (+DY2), compute pure NLO QCD (born) vs NLO QCD + NLO EW (dressed) FKtables and:\\
- compute data/prediction plots\\
- compute chi2\\

\section{Double counting}

Count von Count, ALL

Explain the problem to experimentalists.\\

\section{Outlook}

ALL

\cite{Carli:2010rw}
\cite{Bertone:2014zva}

\appendix

\acknowledgments

C.S. is supported by the European Research Council under the European Unions Horizon 2020 research and innovation Programme (grant agreement no. 740006).

\bibliographystyle{JHEP}
\bibliography{paper}

\end{document}
