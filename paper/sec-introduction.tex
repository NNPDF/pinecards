\section{Introduction}
\label{sec:introduction}

With the recent completion of Run II, the Large Hadron Collider (LHC) has 
accumulated data from an integrated luminosity of approximately 
\SI{150}{\per\femto\barn}~\cite{Mangano:2020icy}. This represents only a small fraction of
the anticipated \SI{3000}{\per\femto\barn} that will eventually be recorded in the
forthcoming twenty years of LHC operation. Nevertheless the statistical 
uncertainty of the data has already shrunk to unprecedented small values,
typically \SI{1}{\percent} or less, a fact that will allow for precision tests of the
Standard Model (SM) and for indirect searches of New Physics only if 
theoretical predictions become comparatively precise. This entails the 
computation of additional higher-order contributions to the fixed-order 
coupling-constant perturbative expansion, on the one hand, and an increasingly 
sophisticated determination of the Parton Distribution Functions (PDFs) of the 
proton~\cite{Gao:2017yyd,Ethier:2020way}, on the other hand. 

In the first respect, because Quantum Chromodynamics (QCD) dominates the 
interactions occurring within colliding protons at the LHC, much effort 
has been devoted to the computation of higher-order QCD corrections: 
fully-differential next-to-leading order (NLO) results, possibly matched to the
parton shower, are currently automated in various general-purpose
Monte Carlo generators~\cite{Gleisberg:2008ta,Alwall:2014hca,Bellm:2015jjp}
(see also ref.~\cite{Buckley:2011ms} for a review),
while an increasing number of next-to-next-to-leading order (NNLO) predictions 
are becoming available for processes with various degrees of inclusiveness
(see e.g.\ ref.~\cite{Amoroso:2020lgh} and references therein). The computation
of higher-order corrections in the electroweak (EW) and mixed QCD+EW theory has
also witnessed a comprehensive progress. Frameworks 
were developed in which the QCD and EW couplings are simultaneously treated as 
small parameters in the perturbative expansion, and the computation of 
theoretical predictions, accurate to NLO in both (including mixed-coupling 
QCD--EW terms), is automated~\cite{Kallweit:2014xda,Biedermann:2017yoi,Frederix:2018nkq}. For an extensive and recent review, see ref.~\cite{Denner:2019vbn}.

In the second respect, contemporary PDF
sets~\cite{Harland-Lang:2014zoa,Ball:2017nwa,Hou:2019efy}
incorporate a significant amount of LHC data, which is analysed with NNLO QCD 
theory by default. No EW corrections are systematically included in the 
theoretical description of the experimental observables to which PDFs are 
optimised, except for QED effects if a photon PDF is 
determined~\cite{Schmidt:2015zda,Manohar:2016nzj,Manohar:2017eqh,Bertone:2017bme,Harland-Lang:2019pla}.
Data points in kinematic regions where EW corrections are expected to affect
the accuracy of the theoretical predictions are usually removed from the fits.
The resulting relative PDF uncertainty --- which accounts only for the
uncertainty of the data and of residual methodological inefficiencies inherent to 
each PDF determination --- can be as low as \SI{1}{\percent} at the EW
scale~\cite{Ball:2017nwa}. Theoretical uncertainties, possibly of comparable 
size (e.g.\ from missing higher-order terms in the QCD perturbative
expansion), have started to be included in a PDF determination and represented 
into PDF uncertainties only very 
recently~\cite{AbdulKhalek:2019bux,AbdulKhalek:2019ihb}. This has led to an 
improvement of PDF accuracy, in terms of an overall
better statistical quality of the fit, at the price of a modest deterioration 
of PDF precision, in terms of a slight inflation of PDF uncertainties.

The two respects are intertwined: PDFs provide a way of obtaining a prediction
for a given process in terms of other processes, therefore the accuracy and 
precision with which PDFs are able to make predictions crucially depend on 
the accuracy of the perturbative computation used in the PDF determination
(matched to the precision of the data). Electroweak (including mixed QCD+EW) 
effects should therefore naturally be included in a PDF determination. The 
reason is twofold. First, one expects NNLO QCD and NLO EW corrections to be of 
comparable size because, at the EW scale, the QCD and EW running couplings 
become similar, $\alphas^2\sim \alpha$. If NNLO QCD corrections are included
by default in PDF determinations, NLO EW and NLO QCD+EW corrections should be
taken into account on the same footing. Second, the virtual exchange of soft or 
collinear weak bosons leads to Sudakov 
logarithms~\cite{Denner:2000jv,Denner:2001gw},
which can make the coefficients of the EW series grow faster than 
their QCD counterparts. Such a behaviour becomes relevant in
phase-space regions associated with large mass scales (roughly of the order
of the \si{\tera\electronvolt}), where several LHC data sets (e.g.\ Z-boson
transverse-momentum distributions) enter both the determination of PDFs and the search
for New Physics.

The systematic inclusion of EW corrections in a PDF determination entails
the solution of two separate problems. First, a problem of efficiency:
fast-interpolation grids should be constructed, whereby partonic matrix 
elements, accurate to NLO QCD+EW, are precomputed in such a way that the 
numerical convolution with generic input PDFs can be efficiently approximated
by means of interpolation techniques. Such grids are a fundamental ingredient
in any PDF fit, where the evaluation of the hadronic cross section needs to 
be performed a large number of times. While two default formats exist for
these grids, \textsc{APPLgrid}~\cite{Carli:2010rw} and
\textsc{fastNLO}~\cite{Kluge:2006xs}, none of them supports the inclusion of EW
corrections nor the interface to a Monte Carlo generator accurate to 
NLO QCD+EW\@. Second, a problem of consistency: the way in which EW effects may
(or may not) be folded into the data varies across different experimental 
analyses, a fact that challenges their consistent theoretical interpretation. 
Examples are the subtraction of background processes which should not be 
considered as such (e.g.\ the $t$-channel photon-induced component in
neutral-current Drell--Yan, which is not a separate processes beyond the leading
order) or of just a part of the EW effects (e.g.\ multiple-photon
radiation from light particles in the final state of neutral- or charged-current
Drell--Yan, especially with electrons). Be that as it may, if EW effects are
systematically included in theoretical predictions, they should not be subtracted
from experimental results, otherwise they will be double counted.

In this paper we address the first of these two problems. Specifically, 
we develop \textsc{PineAPPL}, a new computer library that provides a direct interface
from parton-level Monte Carlo event generators accurate up to NLO QCD+EW to
fast-interpolation grids. The library supports variations of the factorisation
and renormalisation scales, resummation, and matching with a photon- and/or
parton-shower. The library is included by default in 
\textsc{MadGraph5\_aMC@NLO} (\textsc{MG5\_aMC} henceforth) within which it has
been designed and tested. However, it is suitable for use as a plugin in any 
Monte Carlo generator that supports NLO QCD+EW computations, such as 
\textsc{SHERPA}~\cite{Biedermann:2017yoi}. In this respect, \textsc{PineAPPL}
supersedes \textsc{aMCfast}~\cite{Bertone:2014zva} and complements
\textsc{MCgrid}~\cite{DelDebbio:2013kxa,Bothmann:2015dba}.

The paper is organised as follows. In section~\ref{sec:pineappl} we introduce
\textsc{PineAPPL}, we describe its features, we illustrate its operation, and we
assess its performance. In section~\ref{sec:results} we validate
\textsc{PineAPPL} and demonstrate its capabilities by computing fast-interpolation grids, accurate to NLO QCD and NLO QCD+EW, for a representative
set of processes, usually included in PDF fits, for which EW corrections may
have a sizeable effect on the accuracy of the theoretical predictions.
In section~\ref{sec:doublecounting} we use these results to formulate more
comprehensively the second problem sketched above, the solution of which, 
however, remains beyond the scope of the current work. We provide our
conclusions and an outlook in section~\ref{sec:conclusion}. Examples of
installation and usage of \textsc{PineAPPL} are provided in
appendix~\ref{app:pineappl}; appendix~\ref{app:lumis} collects the parton
luminosities for each process considered in section~\ref{sec:results}; and
appendix~\ref{app:add_plots} complements some of the results presented
in section~\ref{sec:results}.

