\section{More technical details about \texorpdfstring{\textsc{PineAPPL}}{PineAPPL}}

%\subsection{Efficiency of the grid representation}
%\label{app:pineappl-representation-efficiency}
%
%Using a similar notation introduced in eq.~\eqref{eq:expansion}, we single out a specific NLO correction:
%\begin{equation}
%\begin{split}
%\alphas^k \bigl( \mu_\mathrm{R}^2 \bigr) \alpha^l \Bigl[ \log \bigl( \mu_\mathrm{R}^2 \bigr) &W_{ab}^{(k,l,1,0)} (x_1, x_2) \\
%+ \log \bigl( \mu_\mathrm{F}^2 \bigr) &W_{ab}^{(k,l,0,1)} (x_1, x_2) \\
%&W_{ab}^{(k,l,0,0)} (x_1, x_2) \Bigr] \text{.}
%\end{split}
%\end{equation}
%If we assume the case of a dynamic scale choice, i.e.\ $\mu_\mathrm{R}^2 \equiv \mu_\mathrm{R}^2 \bigl( x_1, x_2 \bigr)$ and $\mu_\mathrm{F}^2 \equiv \mu_\mathrm{F}^2 \bigl( x_1, x_2 \bigr)$,

\subsection{Installation}
\label{app:installation}

\textsc{PineAPPL} currently consists of three parts: 1) the library itself, which is a dependency for the other parts, 2) the helper program \texttt{pineappl}, which allows to read \textsc{PineAPPL} grids from the command line and make predictions with it, and finally 3) the C interface, which is intended to be used in Monte Carlo integrators to generate the grids.

\lstset{
  basicstyle=\ttfamily,
  showstringspaces=false,
  commentstyle=\color{red},
  keywordstyle=\color{blue}
}

All parts are written in Rust and therefore need a Rust compiler and other tools.
On computers with a \texttt{bash} shell the installation is as simple as
\begin{lstlisting}[language=bash]
curl --proto '=https' --tlsv1.2 -sSf https://sh.rustup.rs | sh
\end{lstlisting}
which downloads the compiler \texttt{rustc}, the package manager \texttt{cargo}, and a few other helpful tools.
When the installation has completed make sure to read and follow the instructions printed before.
See also \url{https://www.rust-lang.org/tools/install} for more details and for installation instructions for other systems.

The command-line program \texttt{pineappl} is compiled and installed using
\begin{lstlisting}[language=bash]
cargo install pineappl_cli
\end{lstlisting}
This program also needs \texttt{LHAPDF} installed.
For usage instructions simply type \texttt{pineappl} in your shell and read the help message.

For the C interface you need to first install \texttt{cargo-c}, download the \textsc{PineAPPL} repository, and compile and finally install into it into a directory \texttt{\$prefix} as follows:
\begin{lstlisting}[language=bash]
cargo install cargo-c
git clone https://github.com/N3PDF/pineappl/
cd pineappl_capi/
cargo cinstall --release --prefix=${prefix}
\end{lstlisting}
The last line will install the C header \texttt{pineappl\_capi.h}, the library, and a pkg-config\footnote{A standard way on Linux to express how dependencies are compiled/linked against, see \url{https://www.freedesktop.org/wiki/Software/pkg-config/}} file (\texttt{pineappl\_capi.pc}).
Make sure that the environment variables \texttt{PATH}, \texttt{LD\_LIBRARY\_PATH}, and \texttt{PKG\_CONFIG\_PATH} are properly set.
The latter is needed for \texttt{pkg-config -{}-cflags -{}-libs pineappl\_capi} to work, which prints the necessary compiler/linker flags.
See App.~\ref{app:example-program} for an example.

Updated installation instructions are kept in the file \texttt{README.md} in \textsc{PineAPPL}'s repository at \url{https://github.com/N3PDF/pineappl/}.

\subsection{Drell--Yan \texorpdfstring{$n$}{n}-tuple storage size}
\label{app:drell-yan-storage}

TODO

\subsection{Example Monte Carlo program}
\label{app:example-program}

\definecolor{mygreen}{rgb}{0,0.6,0}
\definecolor{mygray}{rgb}{0.5,0.5,0.5}
\definecolor{mymauve}{rgb}{0.58,0,0.82}

\lstset{
  belowcaptionskip=1\baselineskip,
  breaklines=true,
  frame=L,
  numbers=left,
  xleftmargin=\parindent,
  showstringspaces=false,
  basicstyle=\footnotesize\ttfamily,
  keywordstyle=\bfseries\color{green!40!black},
  commentstyle=\itshape\color{purple!40!black},
  identifierstyle=\color{blue},
  stringstyle=\color{orange},
}

The following listing shows how to setup \textsc{PineAPPL} using its C interface in a simple Monte Carlo integrator for calculating the double-photon contribution to Drell--Yan lepton-pair production at the LHC.
All \textsc{PineAPPL} functions have the prefix \texttt{pineappl\_}.
The full example can be found at TODO.
\begin{lstlisting}[language=C++,mathescape=true]
// create a new luminosity function for the $\gamma\gamma$ initial state
auto* lumi = pineappl_lumi_new();
int32_t pdg_ids[] = { 22, 22 };
double ckm_factors[] = { 1.0 };
pineappl_lumi_add(lumi, 2, pdg_ids, ckm_factors);

// only LO $\alpha_\mathrm{s}^0 \alpha^2 \log^0(\xi_\mathrm{R}) \log^0(\xi_\mathrm{F})$
uint32_t orders[] = { 0, 2, 0, 0 };

// we bin in rapidity from 0 to 2.4 in steps of 0.1
double bins[] = {
    0.1, 0.2, 0.3, 0.4, 0.5, 0.6, 0.7, 0.8, 0.9, 1.0, 1.1, 1.2,
    1.3, 1.4, 1.5, 1.6, 1.7, 1.8, 1.9, 2.0, 2.1, 2.2, 2.3, 2.4
};

// create the PineAPPL grid with default interpolation and binning parameters
auto* keyval = pineappl_keyval_new();
auto* grid = pineappl_grid_new(lumi, 1, orders, 24, bins, keyval);

// now we no longer need `keyval` and `lumi`
pineappl_keyval_delete(keyval);
pineappl_lumi_delete(lumi);

// fill the grid with phase-space points
fill_grid(grid, 10000000);

// perform a convolution of the grid with PDFs
auto* pdf = LHAPDF::mkPDF("NNPDF31_nlo_as_0118_luxqed", 0);
auto xfx = [](int32_t id, double x, double q2, void* pdf) {
    return static_cast <LHAPDF::PDF*> (pdf)->xfxQ2(id, x, q2);
};
auto alphas = [](double q2, void* pdf) {
    return static_cast <LHAPDF::PDF*> (pdf)->alphasQ2(q2);
};

std::vector<double> dxsec(24);
pineappl_grid_convolute(grid, xfx, xfx, alphas, pdf, nullptr,
    nullptr, 1.0, 1.0, dxsec.data());

// print the results
for (std::size_t i = 0; i != 24; ++i) {
    std::printf("%.1f %.1f %.3e\n", bins[i], bins[i + 1], dxsec[i]);
}

// write the grid to disk
pineappl_grid_write(grid, "DY-LO-AA.pineappl");

// destroy the object
pineappl_grid_delete(grid);
\end{lstlisting}

\subsection{Demonstration of \texorpdfstring{\texttt{pineappl}}{pineappl}}
\label{app:pineappl-demo}

TODO

\subsection{Sample runcard for \texorpdfstring{\textsc{mg5\_aMC@NLO}}{mg5\_aMC@NLO}}
\label{app:sample-runcard}

TODO
