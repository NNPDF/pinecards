\section{PDF-independent storage of phase-space weights with \texorpdfstring{\textsc{PineAPPL}}{PineAPPL}}
\label{sec:pineappl}

In this paper we present a new library called \textsc{PineAPPL}, which stores phase-space weights of a Monte Carlo (MC) integration of a fixed-order calculation independently from the chosen PDFs.
This separates the task of producing predictions for observables into two steps: 1) the generation of the \emph{grids}, which is how the generated files are called, and 2) the convolution of those grids with a set of PDFs.
The advantage of this method is that the time-consuming step 1) has to be done only once and, therefore, allows in step 2) for very quick convolutions of a given grid with multiple PDF sets.

The convolution is typically done in a few seconds or less, which offers at least two applications:
\begin{enumerate}
\item the study of PDF-dependence of observables; e.g.\ PDF set comparisons, PDF uncertainties for different PDF set, $\alphas$ variations, etc., and
\item the determination of PDF sets themselves; the grids together with the corresponding experimental data constitute two important ingredients for a PDF fit.
\end{enumerate}
This puts \textsc{PineAPPL} in the same category as \textsc{APPLgrid}~\cite{Carli:2010rw} and \textsc{fastNLO}~\cite{Kluge:2006xs,Wobisch:2011ij,Britzger:2012bs}.
With respect to the capabilities of these computer codes, \textsc{PineAPPL} makes it possible for the first time to include also higher-order corrections due to electroweak (EW), and in general mixed QCD--EW, effects.
Documenting this extension, and how to interface \textsc{PineAPPL} with a general-purpose matrix-element generator, is the main interest of this paper.

In particular, the following features are supported:
\begin{itemize}
\item The inclusion of perturbative corrections (fixed-order, i.e.\ without parton-shower matching) with any given set of powers of $\alpha$, $\alphas$, in particular including mixed QCD--EW corrections.
\item The support for non-coloured initial-state partons, such as photon-initiated contributions.
In fact, \textsc{PineAPPL} allows arbitrary initial-state combinations, e.g.\ leptonic initial states~\cite{Bertone:2015lqa,Buonocore:2020nai}.
\item The estimate of theory uncertainty via variations of the renormalisation and factorisation scale (the electroweak coupling is assumed to be scale-independent, consistently with the commonest renormalisation schemes).
%\item Support for all-order predictions coming from a resummation calculation or a photon-/parton-shower, which are important for some observables (see section~\ref{sec:results}),
\end{itemize}
On a more technical level, we further point out the following differences and improvements in \textsc{PineAPPL}:
\begin{itemize}
\item \textsc{APPLgrid} typically required a substantial amount of memory (close to \SI{120}{\giga\byte} for Drell--Yan), if a grid for a differential distributions with more than 10 bins was generated.
The memory usage is substantially reduced (roughly \SI{1}{\giga\byte}) after \enquote{optimisation} of the grids, i.e.\ an optimisation of the grid representation in memory.
However, this required a two-step procedure: first, to produce an unoptimised grid in order to identify those parts where the cross section is either zero or extremely suppressed; second, after those parts are removed, to fill an optimised grid with a small number of bins.
\textsc{PineAPPL} avoids this by using a more space-efficient representation from the start.
\item \textsc{PineAPPL} offers an easy-to-use interface written in the C programming language, to allow MC integrators to read and write \textsc{PineAPPL} grids.
C was chosen because it can be easily interfaced with both Fortran and C++, in which most MC integrators are written.
The interface consists of roughly 30 functions, among which only a handful are needed in practice.
See appendix~\ref{app:example-program} for an example.
\item The shell command \texttt{pineappl}, which performs convolutions on the command line (see appendix~\ref{app:pineappl-demo}), without requiring the user to write a new program.
In addition to convolutions, it can also print how the luminosity function is constructed, which perturbative orders are stored, their size, the size of each partonic channel, etc., separately for each bin.
%\item Possibility to import \textsc{APPLgrids} and \textsc{fastNLO} tables.
\end{itemize}

We have interfaced \textsc{PineAPPL} with the new version (v3+) of \textsc{mg5\_aMC@NLO}~\cite{Alwall:2014hca,Frederix:2018nkq}\footnote{It can be downloaded from \url{https://code.launchpad.net/~amcblast/+junk/3.0.2}, and will soon be merged in the official code release.}, where it will replace the \textsc{aMCfast}~\cite{Bertone:2014zva} interface.
Its usage is very similar to that of \textsc{mg5\_aMC v2}+\textsc{aMCfast}+\textsc{APPLgrid}.
In practice, using \textsc{PineAPPL} leads to substantially faster run times, in particular for simple processes, because the grids do not need to be optimised 
and their combination is faster.
See appendix~\ref{app:sample-runcard} for an example of a runcard.

Building an interface for other MC generators poses no difficulties, see appendix~\ref{app:example-program} for a small example program.

\subsection{Cross sections in a multi-coupling expansion}
\label{sec:multi-coupling-expansion}

Fixed-order partonic cross sections $a + b \to X$ supported by \textsc{PineAPPL} are expansions in powers of the strong coupling $\alphas$, the electromagnetic coupling $\alpha$, and the logarithms of $\xi_\mathrm{R} = \mu_\mathrm{R} / Q$ and $\xi_\mathrm{F} = \mu_\mathrm{F} / Q$,
\begin{multline}
\frac{\mathrm{d} \sigma_{ab}}{\mathrm{d} \mathcal{O}} (x_1, x_2, \mathcal{O}, \xi_\mathrm{R}, \xi_\mathrm{F}) \\
= \sum_{k,l,m,n} \alphas^k \left( \xi_\mathrm{R}^2 Q^2 \right) \alpha^l \log^m ( \xi_\mathrm{R}^2 ) \log^n ( \xi_\mathrm{F}^2 ) W_{ab}^{(k,l,m,n)} \left( x_1, x_2, Q^2, \mathcal{O} \right) \text{.}
\label{eq:expansion}
\end{multline}
The cross section shown above is differential w.r.t.\ the observable $\mathcal{O}$, which, in general, is a function of phase space and subject to the usual conditions (soft- and collinear safety, etc.).

In experiments, but also for many calculations where the phase-space integration is performed using MC techniques, finite statistics does not allow us to reconstruct the exact dependence of the cross section on the observable $\mathcal{O}$.
Instead, it it sufficient to approximate the derivative using a piecewise-constant function,
\begin{equation}
W_{ab}^{(k,l,m,n)} \left( x_1, x_2, Q^2, \mathcal{O} \right) \approx \sum_{o=1}^M \frac{\Theta (\mathcal{O}_o^\mathrm{min} \le \mathcal{O} < \mathcal{O}_o^\mathrm{max})}{\mathcal{O}_o^\mathrm{max} - \mathcal{O}_o^\mathrm{min}} w_{ab}^{(k,l,m,n,o)} \left( x_1, x_2, Q^2 \right) \text{,}
\end{equation}
which uses $M$ bins with limits $\{ \mathcal{O}_o^\mathrm{min}, \mathcal{O}_o^\mathrm{max} \}_{o=1}^M$ to partition a finite range of the observable,
\begin{equation}
\mathcal{O}_0^\mathrm{min} < \mathcal{O}_0^\mathrm{max} = \mathcal{O}_1^\mathrm{min} < \ldots < \mathcal{O}_{M-1}^\mathrm{max} = \mathcal{O}_M^\mathrm{min} < \mathcal{O}_M^\mathrm{max} \text{.}
\label{eq:bins-of-diff-xsection}
\end{equation}

If chosen dynamically the Ellis--Sexton scale $Q^2$ depends on the phase space, but we assume the fractions $\xi_\mathrm{R}$ and $\xi_\mathrm{F}$ to be constants of phase space in any case.
This allows variations around the central scale choice $\xi_\mathrm{R} = \xi_\mathrm{F} = 1$, but not changing the scale arbitrarily.
The terms with powers $m > 0$ and $n > 0$ vanish for the central scale choice and are only required for variations of the factorisation and renormalisation scales.
To estimate the perturbative QCD uncertainty---no EW uncertainty is covered by this method---one typically uses a 7-point scale variation, which evaluates the cross section using the following values,
\begin{equation}
(\xi_\mathrm{R}, \xi_\mathrm{F}) \in \left\{ \bigl( 1, 1 \bigr), \bigl( \tfrac{1}{2}, \tfrac{1}{2} \bigr), \bigl( 2, 2 \bigr), \bigl( \tfrac{1}{2}, 1 \bigr), \bigl( 1, \tfrac{1}{2} \bigr), \bigl( 2, 1 \bigr), \bigl( 1, 2 \bigr) \right\} \text{.}
\end{equation}
The (asymmetric) uncertainties are then given as the minimum and maximum value (the envelope), measured from the central value $(1, 1)$.
As is clear from eq.~\eqref{eq:expansion}, the EW coupling $\alpha$ is assumed not to be a dynamically varying coupling, but instead a constant over phase space.
This, however, includes the most common choices of the coupling, which are (not necessarily in this order), $\alpha (0)$, $\alpha (M_\mathrm{Z})$, and $\alpha_{G_\mu}$.
The task that \textsc{PineAPPL} solves can now be described: Approximately reconstruct the functions
\begin{equation}
w_{ab}^{(k,l,m,n,o)} \left( x_1, x_2, Q^2 \right)
\label{eq:weight-map}
\end{equation}
from a set of $N$ function evaluations for specific momentum fractions, scales, and values of the observable:
\begin{equation}
\left\{ x_1^{(i)}, x_2^{(i)}, Q^2_i, \mathcal{O}_i \right\}_{i=1}^N \text{,} \label{eq:phase-space-weights}
\end{equation}
given by the MC integrator.

Using eq.~\eqref{eq:expansion} and
\begin{multline}
\frac{\mathrm{d} \sigma}{\mathrm{d} \mathcal{O}} (\mathcal{O}, \xi_\mathrm{R}, \xi_\mathrm{R}) \\
= \sum_{a,b} \int_0^1 \mathrm{d} x_1 \int_0^1 \mathrm{d} x_2 \int_{Q^2_\mathrm{min}}^{Q^2_\mathrm{max}} \mathrm{d} Q^2 \, f_a (x_1, \xi_\mathrm{F}^2 Q^2) f_b (x_2, \xi_\mathrm{F}^2 Q^2) \sigma_{ab} (x_1, x_2, Q^2, \xi_\mathrm{R}, \xi_\mathrm{R}) \text{,}
\label{eq:pineappl-convolution}
\end{multline}
\textsc{PineAPPL} can then quickly calculate hadronic cross sections for arbitrarily many PDF sets and perform scale variations.

Note that in this section we have omitted the dependence of the weights $w$, the observable $\mathcal{O}$, and the scales $\mu_\mathrm{F}, \mu_\mathrm{R}, Q^2$ on the specific kinematics for which they are computed.
Indeed, beyond LO, different kinematic contributions have to be considered (in ref.~\cite{Bertone:2014zva}, for example, they are labelled with an index $\alpha$, see eq.~(12) therein).
In the FKS subtraction scheme~\cite{Frixione:1995ms,Frixione:1997np} employed in \textsc{mg5\_aMC@NLO} maximally one type of kinematics for each counterterm (soft, collinear, and soft-collinear) is needed, but this is not the general case.
In Catani--Seymour subtraction~\cite{Catani:1996jh}, for example, different dipoles have different phase spaces and therefore different scales.
However, \textsc{PineAPPL} remains completely blind to this fact, and consistent treatment is ensured by correctly using the interface between \textsc{PineAPPL}, which fills each event into a grid using the \emph{numerical value} of $Q^2$.

\subsubsection{General characteristics}

In general, we define leading order (LO) as the set of all possible initial states $a b$ which lead to the same final state $X$, for which the sum of the coupling exponents in eq.~\eqref{eq:expansion} is smallest, i.e.\ $k + l = p$, where $p = \min (k+l)$.
This number is process dependent and usually determined by the number of external particles.
For many processes there is only one LO, but when a process has multiple quark lines, colourless (photons, \dots) and coloured particles (gluons, \ldots) can be exchanged between them, making it possible to have more than one LO.
An important example at the LHC is di-jet production, which has three different LOs: $\mathcal{O} (\alpha^2)$, $\mathcal{O} (\alphas \alpha)$, and $\mathcal{O} (\alpha^2)$.
Each LO receives a higher-order correction with an additional power of $\alphas$ or $\alpha$, which in general leads to at least two next-to-leading order (NLO) corrections.
The correction with the largest power in $\alphas$ can be unambiguously called \enquote{the} QCD correction, and the one with the largest power in $\alpha$ \enquote{the} EW correction.
All remaining corrections are of mixed type, meaning that they, in general, cannot be attributed to either one of strong or electroweak origin.
%Sometimes there are higher orders that cannot directly be understood as a correction to a LO, e.g.\ $\mathrm{p} \mathrm{p} \to \ell \bar{\ell} + \mathrm{jet} + X$, which has one LO, $\mathcal{O} (\alphas \alpha^2)$, but three NLOs, one of which is $\mathcal{O} (\alpha^4)$~TODO.

Due to the typical sizes of the couplings $\alphas^2 \sim \alpha$, it is naively expected that within the same order, i.e.\ for fixed $k + l$, terms with larger powers $\alphas^k$ dominate over those with smaller powers.
In practice, however, this naive expectation is not always true due to dynamic effects.
Some examples are vector-boson scattering processes~\cite{Biedermann:2017bss,Denner:2019tmn}, top-pair production with a W boson and four-top production~\cite{Frederix:2017wme}, and Higgs production with a bottom-pair~\cite{Pagani:2020rsg}.

\subsubsection{Example: Drell--Yan lepton-pair production at the LHC}
\label{sec:pineappl-example}

To give an example of eq.~\eqref{eq:expansion}, the following shows Drell--Yan lepton-pair production up to terms at NLO (with some arguments suppressed for the phase-space weights):
\begin{equation}
\begin{split}
\sigma_{ab}
    &= \alpha^2 W_{ab}^{(0,2,0,0)} \\
    &+ \alphas \left( \xi_\mathrm{R}^2 Q^2 \right) \alpha^2 W_{ab}^{(1,2,0,0)} (Q^2) + \alphas \left( \xi_\mathrm{R}^2 Q^2 \right) \log (\xi_\mathrm{F}^2) \alpha^2 W_{ab}^{(1,2,0,1)} \\
    &+ \alpha^3 W_{ab}^{(0,3,0,0)} (Q^2) + \log (\xi_\mathrm{F}^2) \alpha^3 W_{ab}^{(0,3,0,1)} \text{.}
\end{split}
\end{equation}
The first term with index $(0,2,0,0)$ is the LO term, the following line shows the NLO QCD correction, and the final line the NLO EW correction.
Note that all terms depend on the renormalisation scale only indirectly through $\alphas$, because 1) higher-order terms in $\alpha$ never generate a renormalisation scale dependence (in the $\alpha$ schemes that are valid according to section~\ref{sec:multi-coupling-expansion}) and 2) higher-order QCD corrections only introduce an explicit renormalisation scale dependence in counterterms with vertices with more than two gluons.
At NLO these terms are not present for this process so that terms proportional to $\log (\xi_\mathrm{R}^2)$ vanish.
Both NLOs, however, have contributions from a collinear counterterm that depends on the factorisation scale.

Since this process has a single LO, mixed QCD--EW corrections first appear at next-to-next-to-leading order (NNLO), which include the QCD correction at $\mathcal{O} (\alphas^2 \alpha^2)$, the EW correction $\mathcal{O} (\alpha^4)$, and a single mixed correction at $\mathcal{O} (\alphas \alpha^3)$.

Note that all initial states have to be taken into account that lead to the same final state.
This includes the photon--photon initial state, which appears already at LO.
In the corresponding Feynman diagrams all particles are colourless, so that this subprocess only receives EW and mixed QCD--EW corrections, but never (pure) QCD corrections.
The EW corrections also introduce quark--photon contributions, in analogy of QCD corrections introducing quark--gluon contributions.

%\subsubsection{Example: top-pair production}
%
%\begin{equation}
%\begin{split}
%\sigma_{ab}
%&= \alpha^2                                                                       W_{ab}^{(0,2,0,0)}
% + \alphas   \left( \xi_\mathrm{R}^2 Q^2 \right) \alpha                           W_{ab}^{(1,1,0,0)}
% + \alphas^2 \left( \xi_\mathrm{R}^2 Q^2 \right)                                  W_{ab}^{(2,0,0,0)} \\
%&+ \alphas^3 \left( \xi_\mathrm{R}^2 Q^2 \right)                                  W_{ab}^{(3,0,0,0)} \\
%&+ \alphas^3 \left( \xi_\mathrm{R}^2 Q^2 \right)          \log (\xi_\mathrm{F}^2) W_{ab}^{(3,0,0,1)} (Q^2)
% + \alphas^3 \left( \xi_\mathrm{R}^2 Q^2 \right)          \log (\xi_\mathrm{R}^2) W_{ab}^{(3,0,1,0)} (Q^2) \\
%&+ \alphas^2 \left( \xi_\mathrm{R}^2 Q^2 \right) \alpha                           W_{ab}^{(2,1,0,0)} \\
%&+ \alphas^2 \left( \xi_\mathrm{R}^2 Q^2 \right) \alpha   \log (\xi_\mathrm{F}^2) W_{ab}^{(2,1,0,1)} (Q^2)
% + \alphas^2 \left( \xi_\mathrm{R}^2 Q^2 \right) \alpha   \log (\xi_\mathrm{R}^2) W_{ab}^{(2,1,1,0)} (Q^2) \\
%&+ \alphas   \left( \xi_\mathrm{R}^2 Q^2 \right) \alpha^2                         W_{ab}^{(1,2,0,0)} \\
%&+ \alphas   \left( \xi_\mathrm{R}^2 Q^2 \right) \alpha^2 \log (\xi_\mathrm{F}^2) W_{ab}^{(1,2,0,1)} (Q^2)
% + \alphas   \left( \xi_\mathrm{R}^2 Q^2 \right) \alpha^2 \log (\xi_\mathrm{R}^2) W_{ab}^{(1,2,1,0)} (Q^2) \\
%&+           \left( \xi_\mathrm{R}^2 Q^2 \right) \alpha^3                         W_{ab}^{(0,3,0,0)} \\
%&+           \left( \xi_\mathrm{R}^2 Q^2 \right) \alpha^3 \log (\xi_\mathrm{F}^2) W_{ab}^{(0,3,0,1)} (Q^2)
% +           \left( \xi_\mathrm{R}^2 Q^2 \right) \alpha^3 \log (\xi_\mathrm{R}^2) W_{ab}^{(0,3,1,0)} (Q^2)
%\end{split}
%\end{equation}

\subsection{Grid representations}
\label{sec:grid-representation}

In this subsection we explain the details of how the phase-space weights $w_{ab}$ in eq.~\eqref{eq:bins-of-diff-xsection} are represented.

\subsubsection{4-tuples}

A straightforward representation are 4-tuples, which for each phase-space point saves a list of the momentum fractions $x_1$ and $x_2$, the scale $Q^2$, and the phase-space weight $w$; this is sufficient to reconstruct a total cross section.

For each combination $(a,b,k,l,m,n,o)$ we save the following 4-tuples,
\begin{equation}
\left\{ x_1^i, x_2^i, Q^2_i, w^{(k,l,m,n,o)}_{ab} (x_1^i, x_2^i, Q^2_i, \mathcal{O}_i) \right\}_{i=1}^N \text{.} \label{eq:four-tuples}
\end{equation}
The reconstruction of the differential cross section is then done by simply multiplying the phase-space weights $w$ with PDFs evaluated with the correct arguments given in the 4-tuple and summing over all indices $a$, $b$, $k$, $l$, $m$, $n$, $o$, and $i$.

Using 4-tuples has the clear advantage of being very easy to implement and test.
Furthermore, they reproduce the exact numerical value that is also calculated by the MC integrator.
However, the price one has to pay is the size of the 4-tuples.
To give a concrete example, NLO QCD+EW Drell--Yan lepton-pair production (see section~\ref{sec:pineappl-example} and section~\ref{sec:atlas-high-mass-dy}) needs \SI{159}{\giga\byte} of storage for a target precision of \SI{1}{\percent} of the integrated cross section with this precision.
While this is an acceptable size, increasing the precision by an order of magnitude would require roughly 100 times the size, due to the Monte Carlo convergence that goes as $1/\sqrt{N}$ with $N$ being the number of 4-tuples.
With increasing size also the speed of the convolution degrades, because it basically becomes bound by the speed with which the 4-tuples can be read from disk.
However, due to the uncompressed nature of this representation it can serve as a intermediate format to develop and quickly cross check more space-efficient representations, which we will discuss next.

\subsubsection{Lagrange-interpolation grid}

A different strategy is to partition the $(x_1, x_2, Q^2)$ space,
\begin{equation}
H = [x_\mathrm{min},x_\mathrm{max}]^2 \times [Q^2_\mathrm{min}, Q^2_\mathrm{max}] \ni (x_1, x_2, Q^2)
\end{equation}
along each axis into a small numbers of bins and to insert the phase-space weights $w$ into the corresponding discrete bin.
Using the bin centres and their values one already has a straightforward implementation of eq.~\eqref{eq:weight-map}, but given a finite number of bins usually yields an insufficient approximation for the cross section.
Increasing the number of bins improves the precision, but it also increases the space requirements.
Interpolation methods offer more precision using the same number of bins.

We use the same \enquote{Lagrange-interpolation grid} presented in ref.~\cite{Carli:2010rw} with the parameters published in ref.~\cite{Bertone:2014zva}, which give sufficient precision (see section~\ref{sec:results}).

This method first maps $(x_1, x_2, Q^2) \mapsto (y_1, y_2, \tau)$, with
\begin{equation}
y(x) = 5 (1-x) - \log x \text{,} \qquad \tau (Q^2) = \log \log \frac{Q^2}{(\SI{0.25}{\giga\electronvolt})^2} \text{.}
\label{eq:maps}
\end{equation}
The function $y(x)$ maps events with large $x$ effectively linearly and small $x$ effectively logarithmically onto $y$.
This reflects our knowledge of PDFs, which behave differently in those regions, and thereby increases the precision of the interpolation.
For the convolution of a grid with a PDF set also the inverse functions are needed, which are
\begin{equation}
x(y) = \frac{1}{5} \operatorname{W}_0 (5 \exp (5-y)) \text{,} \qquad Q^2 (\tau) = (\SI{0.25}{\giga\electronvolt})^2 \exp (\exp (\tau)) \text{,}
\end{equation}
where $\operatorname{W}_0 (x)$ is (the principle branch of) the Lambert W function or product logarithm, which satisfies the relation $\operatorname{W} (x) \exp (\operatorname{W} (x)) = x$.

Following ref.~\cite{Carli:2010rw} (see eq.~(17) therein), we furthermore divide the weights, before filling them into the grid, by the function
\begin{equation}
\omega (x_1, x_2) = \left( \frac{\sqrt{x_1}}{1 - 0.99 x_1} \right)^3 \left( \frac{\sqrt{x_2}}{1 - 0.99 x_2} \right)^3 \text{.}
\end{equation}
This flattens the interpolated function in the region $x \to 1$, where the PDFs are small and tend towards zero, and enhances the function in the small-$x$ region.
The effect of this step is an improvement of the precision that depends on the initial states and the process, but it can be as large as a factor of \numrange{10}{100} (one or two more correct digits compared to the MC result).
Before performing a convolution this step is inverted by simply multiplying the interpolated grid values with $\omega (x_1, x_2)$.

The final step is filling the weights into the grid, which maps the variables $(y_1, y_2, \tau)$ onto the 3-dimensional Lagrange-interpolation grid with $N_y = 50$ points in each $y$ direction and $N_\tau = 30$ points in $\tau$ direction.
The interpolation orders $s_y$ and $s_\tau$ are 3 for each dimension, and only the subspace $[\num{2e-7},1] \times [\num{2e-7},1] \times [\num{e2},\num{e6}] \subset H$\footnote{In ref.~\cite{Bertone:2014zva} the upper limit for $Q$ is given as \SI{3162}{\giga\electronvolt} (which corresponds to $Q_\mathrm{max}^2 \approx \SI{e7}{\giga\electronvolt}$), but in the code we found the value $Q_\mathrm{max}^2 = \SI{e6}{\giga\electronvolt\squared}$.} is mapped.

\begin{figure}
\centering
\parbox{0.6\textwidth}{\includegraphics{figures/grid-insertion}}
\begin{tabular}{ll}
\toprule
$u_1$/$u_2$ & $x_1$/$x_2$ \\
\midrule
0 & \num{1.00e0} \\
1 & \num{6.36e-1} \\
2 & \num{3.20e-1} \\
3 & \num{9.96e-2} \\
4 & \num{1.57e-2} \\
5 & \num{1.74e-3} \\
6 & \num{1.81e-4} \\
7 & \num{1.87e-5} \\
8 & \num{1.93e-6} \\
9 & \num{2.00e-7} \\
\bottomrule
\end{tabular}
\caption{Example for a 2-dimensional $10 \times 10$ grid, which is being filled at the location marked with the small black square at $(5.8,4.8)$.
Each side of the grey square starting at $k_i = 4$ and $k_j = 3$ has a length of $N_y + 1 = 4$.
This square marks the grid values (grey dots) that are being updated using eq.~\eqref{eq:interpolation}.
The table on the right-hand side gives the parton-momentum fractions for each grid point according to eq.~\eqref{eq:maps}.
Note that for $u \in [0, 3]$ the values are roughly linearly distributed, then logarithmically.}
\label{fig:grid}
\end{figure}

To illustrate the filling step we give an example in figure~\ref{fig:grid}, where, for simplicity, we assume we have chosen a static scale, so that we do not need to interpolate in $\tau$ direction, and where we also limited the number of grid points to $N_y = 10$.
Each grid point has a numerical value $a_{i,j}$ associated, and the set of all numerical values $\{ a_{i,j} \}$ for all grid indices $(i,j) \in [0,N_y) \times [0,N_y)$ constitute \enquote{the grid}.
Inserting a specific weight $W = w (x_1, x_2) / \omega(x_1, x_2)$ into the grid is shown in figure~\ref{fig:grid} as a small black square, inside the larger grey one.
We have defined the grid points at specific positions, but the points given by the MC will land somewhere between them\footnote{The technical term for this interpolation problem is sometimes called \emph{subtabulation}.}.
The interpolation order $s_y$ then defines a square with length $s_y + 1$ around its centre $(u_1(y_1(x_1)),u_2(y_2(x_2)))$, given by the MC.
All grid points with indices $(i,j)$ covered by the grey square are then updated according to the following formula:
\begin{equation}
a_{i,j} = a_{i,j} + I_i(u(y(x_1))) I_j(u(y(x_2))) W(x_1,x_2) / \omega(x_1,x_2)
\label{eq:interpolation}
\end{equation}
with the Lagrange basis functions
\begin{equation}
I_i (u) = \prod_{\substack{k=k_i \\ k \neq i}}^{k_i + s_y} \frac{u-k}{i-k}
\end{equation}
where the product runs over all indices of the grid points covered by the grey square in figure~\ref{fig:grid}, starting the smallest index in the square, $k_i$ and $k_j$.
Finally, we remapped
\begin{equation}
u(y) = \frac{y-y_\text{min}}{\Delta y} \text{,} \quad
\end{equation}
using $y_\text{min} = y(x_\text{max})$ and $y_\text{max} = y(x_\text{min})$ and the grid spacing $\Delta y = (y_\text{max} - y_\text{min})/(N_y-1)$, so that the integer part of $u(y)$ gives the grid index, e.g.\ $u(y_\text{min}) = 0$ and $u(y_\text{max}) = N_y - 1$, and the fractional part gives the location between the nearest grid points.
