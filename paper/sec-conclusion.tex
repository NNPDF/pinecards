\section{Conclusions}
\label{sec:conclusion}

The systematic inclusion of EW corrections in accurate theoretical computations
for several LHC processes is becoming more and more important in order to
match the increasing precision of the data. In this paper we simplified the
computational aspect of this task, building upon the automation of QCD and EW
computations pioneered in recent years~\cite{Kallweit:2014xda,Biedermann:2017yoi,Frederix:2018nkq}.
Specifically we developed \textsc{PineAPPL}, a new library that stores perturbative calculations from an external Monte Carlo generator in a PDF-independent way using interpolation grids.
This offers the advantage of fast a posteriori convolutions with PDFs, for example to study the uncertainties coming from different PDF sets and/or the strong coupling constant $\alphas$, and to determine the PDFs themselves, a task for
which fast-interpolation grids are fundamental.
We tested \textsc{PineAPPL} together with \textsc{mg5\_aMC} and found a precision of \numrange{e-4}{e-5} relative to the MC result, which is excellent for all
foreseeable practical purposes.
Although we used \textsc{mg5\_aMC}, we note that \textsc{PineAPPL} is not tied in any way to a specific Monte Carlo generator, and can be easily interfaced with any of them.

We emphasise that a distinguishing feature of \textsc{PineAPPL} is the support for arbitrary coupling orders not only in the strong, but also in the electroweak coupling.
This enables us to generate, for the first time, NLO EW and NLO combined QCD--EW interpolation grids.
Using \textsc{mg5\_aMC} we calculated and showcased the impact of these corrections for specific measurements of some representative LHC processes: Drell--Yan lepton-pair production, top-pair production, and Z-boson production with non-zero transverse momentum.

Finally, we discussed the issue of subtracting EW corrections in experimental data, which becomes important when theoretical predictions including EW corrections are compared to experimental data. In particular, with the
development of \textsc{PineAPPL}, all technical requirements are fulfilled for producing the first PDF fit of LHC data including EW and combined QCD--EW corrections.
This will have at least two advantages: in PDF fits phase-space regions are usually cut away if they exhibit large EW corrections; including them therefore increases the number of data points in a fit and therefore indirectly enlarges a PDF set's interpolation region.
Secondly, this makes it possible to use experimental data that are closer to the actual measurement, without the need to compensate for missing EW corrections.
We plan to address this task in a future work.

\vspace{0.5cm}
\hrule
\begin{center}
The \textsc{PineAPPL} library is available at \url{https://n3pdf.github.io/pineappl}.
\end{center}
